\documentclass[12pt]{article}

\usepackage{mathptmx}
\usepackage{mathpazo}
\usepackage{mathrsfs}

\usepackage{amsfonts}
\usepackage{amsmath}
\usepackage{amssymb}
\usepackage[version-1-compatibility]{siunitx}
\usepackage{fixltx2e}
\usepackage{multirow}
\usepackage{dsfont}

\newcommand{\inv}{^{-1}}
\newcommand{\n}{\noindent}
\newcommand{\conj}{^*}
\newcommand{\sq}{^2}

\topmargin -0.5in %topmargin=1in+\topmargin
\textheight 9in % default is letter % so 11-2=9in
\textwidth 6.5in % and 8.5-2in=6.5in
\oddsidemargin 0in %left margin = % 1in + oddsidemargin
\footskip 1cm % pagenumber to text

\setcounter{secnumdepth}{0}

\title{Math 206}
\date{\normalsize Fall 2015}


\begin{document}
\maketitle

\section{8/26}

\subsection{Definitions}

\noindent
A \textit{norm} on a vector space X (over $\mathds{F}$) is a function $\| \|:X \rightarrow \mathds{R}^+$ such that

$\|x\| = 0$ iff $x = 0$

$\|\alpha x\| = |\alpha|\|x\|$ (for $\alpha \in F$)

$\|x + y\| \leq \|x\| + \|y\|$

\noindent
An \textit{algebra} $\mathscr{A}$ over $\mathds{F}$ is a vector space with distributive $\cdot$ satisfying

$cx \cdot y = c(x \cdot y)$

$x \cdot cy = c(x \cdot y)$ for all $c \in F$


\noindent
A \textit{normed algebra} over $\mathds{R}$ or $\mathds{C}$ is an algebra $\mathscr{A}$ equipped with (vector space) norm satisfying

$\|ab\| \leq \|a\|\|b\|$ for all $a,b \in \mathscr{A}$

\noindent
A norm on $\mathscr{A}$ induces a metric 

$d(a, b) = \|a - b\|$ on $\mathscr{A}$ and therefore a topology

if $\mathscr{A}$ is complete for this norm, it is a \textit{Banach algebra}

\subsection{To figure out (use https://www.math.ksu.edu/~nagy/real-an/2-05-b-alg.pdf)}

\noindent
Supposing $\mathscr{A}$ is not necessarily complete

$\|ab\| \leq \|a\|\|b\|$ gives uniform continuity on the product

hence the norm can be extended to the completion $\bar{\mathscr{A}}$ to form a Banach algebra

\noindent
A metric space M is complete if all Cauchy sequences converge to an element of M

\noindent
The completion M is all equivalence classes of Cauchy sequences where 

$\{a_n\} \sim \{b_n\}$ iff $\lim\limits_{x \to \infty}d(a_n - b_n) = 0$

\subsection{Examples}

\noindent
For M a compact space, $C(M)$

the set of continuous $\mathds{R}/\mathds{C}$-valued functions on M

pointwise operations

$\|f\|_\infty = sup \{|f(x)| : x \in M\}$

\noindent
For M locally compact, $C_\infty(M)$

the set of continuous $\mathds{R}/\mathds{C}$-valued functions on M vanishing at $\infty$

vanishing at $\infty$: $\forall \epsilon \exists$ a compact subset of M, outside of which $f < \epsilon$

note that this is non-unital (lacks an identity)

\noindent
For $\mathscr{O} \subset \mathds{C}^n$ open

$H^\infty(\mathscr{O})$ the set of all bounded holomorphic functions on $\mathscr{O}$

\noindent
(M, d) metric space and $f \in C(M)$

Lipschitz constant (which can be $+\infty$) $L_d(f) = sup\{\frac{|f(x) - f(y)|}{d(x, y)} : x, y \in M, x \neq y\}$

The \textit{Lipschitz functions} $\mathscr{L}_d(M, d) = \{f: L(f) < \infty\}$

These form a dense subalgebra of C(M) and are in fact a *-subalgebra

$\|f\|_d$ := $\|f\|_\infty + L_d(f)$, can be shown as a normed-algebra norm

$L_d(M, d)$ is complete for this norm

so $L_d(M, d)$ is a Banach algebra

$L_d$ is a seminorm on $\mathscr{L}_d(M, d)$ since it takes value 0 on the constant functions

can recover d from $L_d$

\noindent
M a differentiable manifold (e.g. $T = \mathds{R}\slash\mathds{Z}$ the circle)

C(M) $\supseteq C^{(1)}(M)$ the singly-differentiable functions

$f \in C^{(2)}(T) \to Df: T_xM \to \mathds{R}, \mathds{C}$

with $Df$ the derivative and $T_x$ the tangent space

If we put on a Riemmannian metric, define $\|f\|^{(1)} = \|f\|_\infty + \|Df\|_\infty$

If $f \in C^{(1)}(T): \|f\|^{(1)} = \|f\|_\infty + \|f'\|_\infty$

Banach algebra norm, for which this space of functions is complete

\noindent
For the circle, $C^{(2)}(T) \to \|f\|^{(2)} = \|f\|_\infty + \|f'\|_\infty + \frac{1}{2}\|f''\|_\infty$

the factor $\frac{1}{2}$ ensures that this satisfies the normed algebra condition

$C^{(n)}(T) = \sum\limits_{k=0}^n\frac{1}{k!}\|f^{(k)}\|_\infty$

For $C^\infty(T)$ using the collection of norms $\{\|\|^{(n)}\}^\infty_{n=1}$ yields a Fr\'{e}chet algebra

A Fr\'{e}chet algebra has a topology defined by a countable family of seminorms

that respect the algebra structure and is complete \textbf{(clarify)}

\noindent
non-commutative algebras

X a Banach space

$\mathscr{B}(X)$ the algebra of bounded operators on X

$\|\|$ operator norm $\to$ Banach algebra

Any closed subalgebra of $\mathscr{B}(X)$ is a Banach algebra

\section{8/28}

\noindent
Sketch of the course

\noindent
X a Banach space, $B(X)$ bounded functions on the space

\noindent
$\mathscr{H}$ a Hilbert space, $\mathscr{B}(\mathscr{H})$ bounded operators on the space

for $T \in \mathscr{B}(\mathscr{H}) \exists$ adjoint operator $T^* \in \mathscr{B}(\mathscr{H})$

$<T\xi, \eta> = <\eta, T^*\xi>$ for $\xi, \eta \in \mathscr{H}$

adjoint is additive, conjugate linear, $T^{**} = T$, $(ST)^* = T^*S^*$

\noindent
An algebra $A$ over $\mathds{R}$ or $\mathds{C}$ is a \textit{*-algebra} if it has a $*: A \to A$ satisfying

certain properties (look up)

\noindent
A \textit{*-normal algebra} is a normal *-algebra such that

$(\forall a \in A)\|a^*\| = \|a\|$

\noindent A \textit{Banach *-algebra} is a *-normal algebra that is a Banach algebra.

\noindent
For any $T \in \mathscr{B}(\mathscr{H})$, have $\|T^*T\| = \|T\|^2$ \textbf{(check: parse through defns)}

\noindent
For M a locally compact space, $A = C_\infty(M, \mathds{C})$, $f^*:= \bar{f}$ is a Banach *-algebra

Also have $\|f^*f\| = \|f\|^2$ \textbf{(verify: should be easier than the other)}

\noindent Little Gelfand-Naimark theorem:

Let $A$ be a commutative Banach *-algebra satisfying $\|a^*a\| = \|a\|^2$.

Then $A \cong C_\infty(M)$ for some locally compact M.

\noindent
One view of the ``spectral theorem''

Let $T \in \mathscr{B}(\mathscr{H})$ with $T^* = T$

Let $A$ be the closed subalgebra of $\mathscr{B}(\mathscr{H})$ generated by T and I (i.e. $p(T) := \Sigma \alpha_kT^K$)

Polynomials closed or stable under *

If $S \in A$ then $S^* \in A$ (i.e. $A$ is a *-subalgebra of $\mathscr{B}(\mathscr{H})$

So $A$ is a Banach *-suubalgebra satisfying $\|S^*S\| = \|S\|^2$

Moreover, $A$ is commutative.  (unital, since generated by I)

Then by the Little Gelfand-Naimark theorem, $A \cong C(M)$

Indeed $M \subset \mathds{R}$, the spectrum of T

If $\mathscr{H}$ is finite dimensional, then $M$ is the set of eigenvalues of T

T is normal if $TT^* = T^*T$

\noindent
A C*-algebra is a Banach *-algebra over $\mathds{C}$ satisfying

$\|a^*a\| = \|a\|^2$

\noindent
Theorem: A commutative C*-algebra is $\cong C_\infty(M)$.

\noindent
Big Gelfand-Naimark Theorem: (\textit{Math 208, C*-algebras})

Any C*-algebra is $\cong$ to a closed *-subalgebra of $\mathscr{B}(\mathscr{H})$ for some Hilbert space $\mathscr{H}$.

\noindent
Tangent

algebraic topology, differential geometry, Riemann manifolds, ``non-commutative geometry'' (Connes)

\noindent
A \textit{von-Neumann algebra} is a *-subalgebra of $\mathscr{B}(\mathscr{H})$

which is closed under the strong operator topology.

Every commutative von-Neumann algebra is $\cong L^\infty(X, S, \mu)$ (measure spaces) acting on $L^2(X, S, \mu)$ by positive sldkjfalksdjf

\noindent
For group G, $\alpha: G \to Auto(X) \subseteq \mathscr{B}(X)$

$Auto(X)$ a Banach space

Look at subalgebra of $\mathscr{B}(X)$ generated by $\alpha(G)$.

Leads to considering $l'(G)$ with product $(f \star g)(x) = \Sigma f(y)g(y^{-1}x)$ convolution

$f^*(x) = \overline{f(x^{-1})}$

Banach *-algebra, G commuatative $\to$ Fourier transform

\section{8/31}

\noindent
K a field, X a set, $\mathscr{F}(X, K)$ the set of all K-valued functions on X with pointwise operations

Given $f \in \mathscr{F}(X, K)$.

Let $\lambda \in K$.  Then $\lambda \in$ range(f) exactly if $(f - \lambda 1)$ is not invertible.

\noindent
For any $a \in A$, the \textit{spectrum} of a is $\{\lambda \in K: a - \lambda 1_A$ is not invertible in A\}.

The spectrum depends on the containing algebra

Assuming that this algebra A  (over the field K) has an identity $1_A$

\noindent
Example: Let A = C([0, 1]), and B = polynomials, viewed as a dense subalgebra of A.

Let p be a polynomial of degree $\geq 2$.  Then

$\sigma_a(p) = p([0, 1])$.

$\sigma_B(p) = \mathds{R}, \mathds{C}$

\noindent
Let A be a Banach algebra with 1 ($\|1\| = 1$), and $a \in A$.

If $\|a\| < 1$, then 1 - a is invertible, and $\|(1 - a)^{-1}\| \leq \frac{1}{1 - \|a\|}$

\noindent
Proof:

$\frac{1}{1 - a} = \sum_{n = 0}^{\infty} a^n$ ($a^0 := 1_A$)

For any $n > 0$, let $s_n = \sum_{k = 0}^na^k$.

Show that $\{s_n\}$ is a Cauchy sequence.

If $n > m$, $\|s_n - s_m\| = \|\sum_{k = m + 1}^na^k\| \leq \sum_{k = m + 1}^n \|a\|^k$.

Given $\epsilon > 0 \exists N$ such that if $m, n \geq N$ then $\sum_{k = m + 1}^n \|a\|^k \leq \epsilon$

So $\{s_n\}$ is a Cauchy sequence.

By completeness there is a $b \in A$ with $s_n \to b$ as $n \to \infty$.

Want to show $b = (1 - a)^{-1}$.

$b(1 - a) = lim_{n \to \infty}(s_n(1 - a))$

$ = lim_{n \to \infty}(1 + a + a^2 + a^3 + \cdots+a^n - (a + a^2 + a^3 + \cdots + a^{n + 1}))$

$= lim_{n \to \infty}(1 - a^{n + 1}) = 1$

Then $1 - (1 - a)$ is invertible, i.e. a is invertible.

$\|(1 - a)^{-1}\| = \lim\|s_n\| \leq \lim\sum_{k = 0}^n\|a^k\| = \frac{1}{1 - \|a\|}$  ($\|1\| = 1$)

\noindent
$\|ab\| \leq \|a\|\|b\|$: can very easily check that multiplication is cts (do this?)

\noindent
Corollary: If $a \in A$ and $\|1 - a\| < 1$ then a is invertible, and $\|a^{-1}\| \leq \frac{1}{1 - \|1 - a\|}$

I.e. the open unit ball about 1 consists of invertible elements.

\noindent
Let $a \in A$. Let $L_a$, $R_a$ be the operators of left and right multiplication by a on A.

$a \to L_a$ is an algebra homomorphism of A into $\mathscr{L}(A)$ (linear operators on A)

$L_aL_b = L_{ab}$, $R_aR_b = R_{ba}$ (R is an antihomomorphism)

$1 \in A$

If a is invertible, then so is $L_a$, $L_aL_{a^-1} = I_a$

Then if A is a normed algebra, $\|L_a\| = \|a\|$

$\|L_{ab}\| = \|ab\| \leq \|a\|\|b\|$

$\|L_a1_a\| = \|a\|$

so if $a \in A$ is invertible, then $L_a$ is a homeomorphism of A onto itself.

Thus if $A$ is a Banach algebra, with 1, and a is invertible:

$\{L_{a}b : \|1 - b\| < 1\}$ is an open neighborhood of a consisting of invertible elements

\noindent
Let GL(A) be the set of invertible elements of A. (general linear group)

Then (for A a unital Banach algebra) GL(A) is an open subset of A.

(Fails for Poly $\subseteq$ C([0, 1]))

\noindent
Two Fr\'{e}chet algebras, for one, GL(A) is an open subset, for another it isn't.

\textbf{ask about this?}: not sure what he was talking about

$C^\infty(T)$, $\|f^{(n)}\|$

C(R) cont fns on $\mathds{R}$ (or $\mathds{C}$) maybe unbounded

For each n let $\|f\|_n = $ sup$\{|f(t)|: |t| \leq n\}$


\noindent
Corollary: For A a Banach algebra with 1 and $a \in A$, $\sigma(a)$ is a closed subset of $\mathds{C}$

\section{9/2}

\noindent
Proposition: Let A be a unital Banach algebra and $a \in A$.

Then $\sigma(a)$ is a closed subset of $\mathds{C}$ or $\mathds{R}$.  If $\lambda \in \sigma(a)$ then $\|\lambda\| \leq \|a\|$.

\noindent
Proof: $\sigma(a) = \{\lambda : (a - \lambda)$ is not invertible\}

Its complement, the \textit{resolvant set}, of a is $\{ \lambda: (a - \lambda) \in GL(A)\}$, is open.

If $|\lambda| > \|a\|$ then $(\lambda - a) = \lambda(1 - \frac{a}{\lambda})$, $\|a\slash \lambda \| < 1$

so $(\lambda - a)$ is invertible, ie $\lambda \in \sigma(a)$.

Over $\mathds{R}$, can have $\sigma(a) = \emptyset$, e.g. $\begin{pmatrix} 0 & -1 \\ 1 & 0 \\ \end{pmatrix}$

\noindent
``If $a \in GL(A)$ and b is close to a then $b^{-1}$ is not much bigger than $a^{-1}$''.

Let $\mathscr{O} = \{c : \|1 - c\| < 1\slash 2 \}$

So c is invertible, and $\|c^{-1}\| \leq \frac{1}{1 - \|1-c\|} \leq 2\}$

Let $b \in a\mathscr{O}$, so $b = ac$ for $c \in \mathscr{O}$, then $\|b^{-1}\| = \|c^{-1}a^{-1}\| \leq 2\|a^{-1}\|$.

For $a, b \in GL(A)$.

$b^{-1} - a^{-1} = b^{-1}(a - b)a^{-1}$

Thus $\|b^{-1} - a^{-1}\| \leq \|b^{-1}\|\|a - b\|\|a^{-1}\|$.

So $b \to b^{-1}$ is continuous for the norm.

So $GL(A)$ is a topological group for topology from norm.

$b^{-1} = (1 + b^{-1}(a - b))a^{-1}$

\noindent
On $\rho (a)$ (the resolvant set, complement of the spectrum) define the resolvant of a

This is the function $R(a, \lambda) = (\lambda - a)^{-1}$

\noindent
$R(a, \lambda)$ is an analytic function on $\rho (a)$.

\noindent
Proof: Let $f(z) = R(a, z)$.

$f'(z) = \lim_{h \to 0} \frac{f(z + h) - f(z)}{h} = \frac{(z + h - a)^{-1} - (z - a)^{-1}}{h}$ 

$ = \lim_{h \to 0}\frac{1}{h} (z + h - a)^{-1} ((z - a) - (z + h - a))(z - a)^{-1}$

$= \lim_{h \to 0} -(z + h - a)^{-1}(z - a)^{-1} = -(z - a)^{-2}$

$f'' = +z(z - a)^{-3}$

Given $z_0 \in \rho (a)$

Will use $b^{-1} = (1 + b^{-1}(a - b))a^{-1}$ and $f(z) = (z - a)^{-1} = \sum c_n(z - z_0)^n$

$f(z) = (z - a)^{-1}$

$b \to z - a$

$f(z) = (1 + (z - a)^{-1}((z_0 - a) - (z - a)) (z_0 - a)^{-1}$

$=(1 + (z - a)^{-1}(z_0 - z))(z_0 - a)^{-1}$

where $(z - a)^{-1}(z_0 - z) \leq 1$ then the above

$=\sum(-1)^{n}(z - a)^{-n}(z - z_0)^n = \sum(-1)^n(z-a)^{-n-1}(z-z_0)^n$

a proper power series expansion.

\noindent
Examine $R(a, z)$ at $\infty$.

$R(a, z^{-1}) = (z^{-1} - a)^{-1} = \frac{1}{z^{-1} - a}$ 

$ =z(1 - za)^{-1}$ (for small z, ie $\|za\| < 1$)

$R(a, z^{-1})$ approaches 0 as $z \to 0$.

defn $R(a, 0^{-1}) = 0$, see $R(a, z)$ is analytic at $\infty$.

\noindent
Theorem: For a Banach algebra over $\mathds{C}$ with 1, and for any $a \in A$,

$\sigma(a) \neq \emptyset$, that is, the spectrum is non-empty.

\noindent
Proof: Suppose that $\sigma(a) = \emptyset$.

Then $R(a, z)$ is defined on all of $\mathds{C}$ and is bounded.

By Liouville's, $R(a, z)$ is constant, $= 0$, $(a - z)^{-1} = 0$ $ \forall z$

Why can we use Liouville's in this Banach space case?

Let A' be the dual Banach space to A.

For $\varphi \in A'$, $z \mapsto \varphi(R(a, z))$ is a $\mathds{C}$-valued analytic function.

So set $\varphi(R(a, z)) = 0$ $\forall z$, $\forall \varphi$

so $R(a, z) = 0$.

Knowing that there is anything in here is the Hahn-Banach Theorem, depending on the axiom of choice.

\noindent
Theorem (Gelfand-Mazer)

Let A be a unital Banach algebra over $\mathds{C}$.

If every nonzero element is invertible, then $z \to z1_A$ is an isomorphism from $\mathds{C}$ onto A.

\noindent
Proof:

Given $a \in A$ let $z \in \sigma (a) \neq \emptyset$.

So $(z - a)$ is not invertible so $z - a = 0$.

Fails over $\mathds{R}$ since have $\mathds{R}$, $\mathds{C}$, quaternions

\section{9/4}

Let A be an algebra or ring.  Ideals $I$, left, right, 2-sided.

$A\slash I$, for a left ideal get a left A-module, right ideal get a right A-module

Two-sided get an algebra or ring.

\noindent
Let A be a normed algebra, and if $I$ is an ideal in A.

Then $\overline{I}$ (the closure) is again an ideal in A.

$\{a_n\} \subset I$, $a_n \to c \in A$ then $ba_n \to bc$

\noindent
Proposition: If A is a unital Banach algebra and if I is an proper ideal in A.

Then $\overline{I}$ is proper.

\noindent
Proof:

Use GL(A) is open.  The original ideal cannot contain any invertible elements.

The complement of the invertible elements is going to be closed.

The ideal is in its closure; the closure is in the complement of the invertible elements.

So the closure will not contain any invertible elements.

\noindent
Counter-example: C locally compact, $C_c(\mathds{R}) \subset C_\infty(\mathds{R})$

In fact $C_c(\mathds{R})$ is the minimal dense ideal in $C_\infty(\mathds{R})$

Lack of identity element.

\noindent
Counter-example: Look at A the polynomials viewed as a subset of $C([0, 1])$

Using the sup-norm.

Lack of completeness.

Let $I = \{p: p(2) = 0\}$ is an ideal, in fact a maximal proper ideal.

This ideal of polynomials will be dense inside A and dense inside all polynomials.

\noindent
Counter-example: A = $C(\mathds{R})$, including unbounded, $I = C_C(\mathds{R})$, compact open topology

Compact in here for the Fr\'{e}chet open topology.

\noindent
Corollary: Let A be a unital Banach algebra.

Then every maximal ideal is closed.

\noindent
Taking ideals to form quotients.

Recall that if X is a normed vector space and if Y is a vector subspace of X:

Then we can form the quotient vector space $X \slash Y$.

Have the evident $\pi: X \to X \slash Y$ by $\pi (x) = x + Y$.

Set $\|\pi(x)\| = inf\{\|x - y\| : y \in Y\}$ i.e. the distance from x to the subspace Y.

It is easily seen that this is a seminorm.

Problem: if Y is not closed, then it is not a norm, since $\|\pi(y)\| = 0$ for $y \in \overline{Y}$.

Therefore, if Y is closed, then $\|\|_{X \slash Y}$ is a norm.

\noindent
(Should know from $202B$) If X is a Banach space and Y is a closed subspace.

Then $X \slash Y$ with the norm defined above is a Banach space.

Trickier to prove, but a true statement.

\noindent
Proposition: Let A be a normed algebra and let I be a closed ideal in A.

(can be left or right or two-sided)

Then if I is a two-sided ideal, then $A \slash I$ with $\|\|_{A \slash I}$ is a normed algebra.

That is $\|\pi(a)\pi(b)\| \leq \|\pi(a)\|\|\pi(b)\|$.

Then if I is a left ideal, so that $A \slash I$ with $\|\|_{A \slash I}$ is a left A-module.

$\|a\pi(b)\|_{A \slash I} \leq \|a\|_A\|\pi(b)\|_{A\slash I}$

And if I is a right ideal, so that $A \slash I$ with $\|\|_{A \slash I}$ is a right A-module, similar.

\noindent
Proof:

For I a two-sided ideal.

If $c, d \in I$, then $\|\pi(a)\pi(b)\|_{A \slash I} \leq \|(a - c)(b - d)\| = \|(ab - (cb + ad - cd)\|$

where $cb + ad - cd \in I$.

Take inf over $c, d \in I$.

$\leq \|a - c\| \|b - d\|$

\noindent
Proposition: If a is a Banach algebra and I is a closed 2-sided ideal, then $A\slash I$ is a Banach algebra.

\noindent
An algebra or ring is \textit{simple} if it contains no proper 2-sided ideals.

e.g. $M_n(\mathds{R})$, $M_n(\mathds{C})$, $M_n(K)$ for $K$ a field.

\n
Proposition: Let A be an algebra or ring and let I be a maximal 2-sided ideal in A.

The $A \slash I$ is a simple algebra/ring.

\n
Corollary: If A is a Banach algebra and if I is a maximal closed 2-sided ideal.

Then $A \slash I$ is a simple Banach algebra.

\n
Let A be a commutative algebra/ring with 1.

If $a \in A$ and a is not invertible, then $Aa$ is a two-sided ideal.

And $1 \not\in Aa$ (else it would be invertible).

Thus $Aa$ is a proper ideal.

\n
Corollary: If A is a commutative algebra/ring with 1 and if A is simple, then A is a field.

A is simple: every nonzero element is invertible.

Since if one weren't we would have a proper ideal.

\n
Theorem: Let A be a commutative Banach algebra with 1.

Let I be a maximal ideal in A (which of course is necessarily closed).

Then $A \slash I \cong$ either $\mathds{C}$ (isomorphic in every sense).

\n
Proof:

Banach-Mazer theorem applied to the previous discussion.

\n
$c \in A\slash I$, $z \in \sigma(c)$, $c - z1_{A}$ is noninvertible by def'n spectrum.

Since it is a field, $c - z1_{A} = 0$.

\n
From a maximal ideal, $I$, get a homomorphism $\phi: A \to \mathds{C}$, unital

($\phi(1) = 1$) such that $I = ker(\phi)$

\n
Let A be a Banach algebra with 1 and let $\phi: A \to \mathds{C}$ be a homomorphism.

Then $\phi$ is continuous and $ker(\phi)$ is a maximal 2-sided ideal in A.

$\|\phi\| = 1$

\n
Lemma: For any given $a \in A$, $\phi(a) \in \sigma(a)$

Proof: $\phi(a - \phi(a)1_A) = 0$ so $a - \phi(a)1$ is not invertible.

so $\phi(a) \in \sigma(a)$

Then $\|\phi(a)\| \leq \|a\|$, so $\|phi\| \leq 1$ but $\phi(1) = 1$.

\section{9/9}

\noindent
\textit{thanks for notes from Roy}\\

\noindent
Let A be a commutative Banach algebra over $\mathds{C}$ with 1.

$\exists$ natural bijection b/t the maximal idea of A and the $\not \equiv 0$ homomorphisms $A \to \mathds{C}$.  

Given a homomorphism $\varphi$, then $I_\varphi = ker(\varphi)$.

\noindent
Notation: $\hat{A}$ denotes the set of maximal ideals

equivalently, the set of $\not \equiv 0$ hom $\varphi: A \to \mathds{C}$, with multiplicative linear functionals

$\hat{A}$ is also the maximal ideal space of A\\

\noindent
For $\varphi \in \hat{A}$, $\phi(a) \in \sigma(a)$.

Corollary: $\|\varphi\|$ = 1.\\

\noindent
$\hat{A} \subset A'$.  in fact $\hat{A} \subset$ unit ball of $A'$.

On $A'$, have the weak-* topology.  Alogue's Thm $\to$ unit ball of G' is compact.\\

\noindent
Proposition: $\hat{A}$ is closed under the weak-* topology and thus compact.
\noindent
Proof:

Let $\{\varphi_x\}$ be a net of elements of $\hat{A}$ which converges to $\psi \in A'$.

Then for $a, b \in A$, we have:

$\psi(ab) = \lim \phi_\alpha(ab) = \lim \phi_\alpha(a)\phi_\alpha(b) = \phi(a)\phi(b)$

$\psi(1a) = \lim \phi_\alpha(1a) = \lim 1 = 1$.\\

\noindent
Gelfand transform

For all $a \in A$, define $\hat{a} \in C(\hat{A})$ by $\hat{A}(\varphi) := \varphi(a)$.

$\hat{A}$ is continuous since if $\{\varphi_\alpha\} \to \varphi$ for weak-* topology,

then $\hat{A}(\varphi) = \varphi(a) = \lim \varphi_\alpha(a) = \lim \hat{A}(\varphi_\alpha)$.

Also $a \mapsto \hat{a}$ is a unital hom of A into $C(\hat{A})$.

Indeed for $a \cdot b \in A$, $\hat{(ab)} \phi = \phi(ab) = \phi(a)\phi(b) = \hat{a}(\phi)\hat{b}(\phi)$.

Therefore $\hat{(ab)} = \hat{a}\hat{b}$.

$\|\hat{A}\|_\infty \leq \|A\|$, since $\forall \varphi$, $\hat{a}(\varphi) = \varphi(a) \in \sigma(a)$.

$|\hat{a}(\phi)| = |\phi(a)| \leq \|a\|$.

\noindent
Definition: The mapping $A \to \hat{A}$ is called the Gelfand transformation for A.\\

\noindent
Note for $A = C_\infty(\mathds{Z})$.  Define $\varphi_n(f) = f(n)$ for $n \in \mathds{Z}$.  $\lim_{n \to \infty} = ?$ in weak-*.\\

\noindent
Let X be a Banach space.  Define a product on X by setting all products = 0.
\noindent
Proposition:

The image of A under Gelfand transform separates the points of $\hat{A}$.

i.e. if $\hat{a}(\varphi) = \hat{a}(\varphi ')$ for $a \in A$, $\varphi = \varphi '$, $\varphi(a) = \varphi ' (a)$.

\noindent
Suppose that a is generated by one element $a_0$ (+1), i.e. poly($a_0$) is dense in A.

Then $\hat{A} \cong \sigma(a_0)$ homomorphic.

\noindent
Proof:

Consider $\hat{a_0} : \hat{a} \to \sigma(a_0)$, $\hat{a_0}(\varphi) = \varphi(a_0) \in \sigma(a_0)$.

If $\varphi, \psi \in \hat{A}$ and $\varphi(a_0) = \psi(a_0)$, then $\varphi(p(a_0)) = \psi(p(a_0))$ for $p \in $poly.

As the poly($a_0$) are dense $\varphi = \psi$.

Therefore $\hat{a_0}: \hat{a} \to \sigma(a_0)$ is inj.

Let $Z \in \sigma(a_0)$.  Then $a_0 - Z$ is not invertible.

Thus $(a_0 - Z)A$ is a proper ideal.  Let I be its closure.

Claim: I is maximal.

Consider $A/I$.  $C_{1a + I} \supset $poly$(a_0)$.

Use $(a_0 - Z) A \supset (a_0 - z)$poly.

$C_1$ = poly($a_0$).

Let $\varphi$ be the multiplicative linear functional such that $\varphi(a_0) = Z$.

image of $C_1 + I$ in $A/I$. $C_1 + I$ dense in $A$.

Therefore $A/I \cong \mathds{C}$.

\section{9/11}

X a compact (hausd) space and $A = C(X)$.  What is $\hat{A}$? (the maximal ideal space)

Each $x \in X$, $\phi_x(f) = f(x)$.  Multiplicative linear functional.

Get map $X \to \hat{A}$, $x \mapsto \phi_x$.  Question: is this onto?

\noindent
Proposition: Let I be a proper ideal in $A = C(X)$.

There is at least one $x \in X$ such that $f(x) = 0$ for all $f \in I$.

\noindent
Proof:

Suppose no such x exists. (Goal: show that $I = A$.)

Then for each $x \in X$ is $f_x \in I$ with $f_x(x) \neq 0$.

Multiply it by $\overline{f_x}$, then we have $\overline{f_x}f_x$ non-negative and real.

Let $U_x = \{y : f_x(y) > 0 \}$ open, $x \in U_x$.

The $U_x$'s cover X, so there is a finite subcover, corresponding to $\{x_1, \cdots, x_n\}$

Then let $f = \sum_{j = 1}^nf_{x_j}$.  Then $f(x) > 0$ all $x \in X$.

Certainly $f \in I$.  In A, f is invertible.  Hence I = A.

\noindent
We can conclude that $X \to \overline{A}$ is onto, one-to-one, continuous (easy to see).

Compact space to a Hausdorff space this is a homeomorphism.

\noindent
Proposition: $\overline{A}$ ``='' X.\\

\noindent
A commutative Banach algebra with 1.

Have the Gelfand transform: $A \to C(\hat{A})$, where $a \mapsto \hat{a}$.

\noindent
If a is nilpotent, $a^n = 0$, for any $\phi \in \hat{A}$:

$0 = \phi(a^n) = (\phi(a))^n$, so $\phi(a) = 0$.  Thus $\hat{a} \equiv 0$.  (0 on all elements)\\

\noindent
A comm. Banach, $1 \in A$. ($\mathds{C}$)

We have ssen that $a \in A$, range($\hat{a}) \subset \sigma(a)$.

\noindent
Is the converse true? Yes, observe:

If $\lambda \in \sigma(a)$, then $\lambda 1_a - a$ is not invertible so $(\lambda - a)A$ is a proper ideal (2-sided).

From algebra: every ideal is contained in a maximal ideal; call it $M$.

So there is $\phi \in \hat{A}$ having this maximal ideal as its kernel.

Then $\phi(\lambda - a) = 0$ and $\phi(a) = \lambda$.

\noindent
Proposition: range$(\hat{a}) = \sigma_a(a)$.\\

\noindent
Let $A = C_b(\mathds{Z}^-)$ be the algebra of all bounded $\mathds{C}$-valued sequences $\| \|_\infty$.

Let $I = C_\infty(\mathds{Z}^+)$ an ideal. (proper, norm-closed)

Then $I$ is contained in a maximal ideal of A, so there is a $\phi \in \hat{A}$ with $\phi(I) = \{0\}$.

Such maximal ideals are ``not constructive''.  (logician term)

Consequence of the use of Zorn's Lemma to say that a maximal ideal exists.

Let R be any unital ring, e.g. finite field, and $\mathscr{R} = \prod_{n=1}^\infty R$, $I = \bigoplus_{n=1}^\infty \mathscr{R}$.

Then exists maximal ideal of R containing I, same difficulty/situation.

I.e. does not have anything to do with Banach algebras.

$\hat{A} = \beta \mathds{Z}^+$ the Stone-C\^{e}ch compactification of the positive integers\\

\noindent
The \textit{expected radius} of a is max$\{|\lambda|: \lambda \in \sigma(a)\}$

Because of range$(\hat{a}) = \sigma(a)$ this is the same as $\|\hat{a}\|_\infty$.

This definition works also for a non-commutative algebra A.

But consequence applies only to commutative algebras. (Gelfand transform def'd)\\

\noindent
For A commutative Banach algebra with 1 and any $a, b \in A$.

$r(ab) \leq r(a)r(b)$ and $r(a + b) \leq r(a) + r(b)$. ($ab \neq ba$ $\to$ can fail)

\noindent
Proof:

$r(ab) = \|\hat{(ab)}\|_\infty = \|\hat{a}\hat{b}\|_\infty \leq \|\hat{a}\|_\infty\|\hat{b}\|_\infty = r(a)r(b)$.

\noindent
Note: the spectral radius is independent of the containing algebra.\\

\noindent
First form of ``holomorphic functional calculus''.

\noindent
Given $a \in A$ Banach algebra with 1.

Let f be a function holomorphic on an open subset of $\mathds{C}$ containing $\{z : |z| \leq \|a\|\}$.

Thus f has a power series expansion $f(z) = \sum_{n = 0}^\infty \alpha_nz^n$ that converges absolutely and uniformly on $\{z : |z| \leq \|a\|\}$.

Thus can define $f(a) := \sum_{n = 0}^\infty\alpha_n a^n$.

\noindent
Proposition (proof next time): If $\lambda \in \sigma(a)$, then $f(\lambda) \in \sigma(f(a))$.

\section{9/14}

Let A be unital Banach algebra, let $a \in A$.

\noindent
Proposition: let f be analytic function with power series expansion that converges on some open subset of $\mathds{C}$ that contains $\{z : |z| \leq \|a\|\}$.  Then $f(a) = \sum \alpha_n a^n \in A$ and if $\lambda \in \sigma(a)$ then $f(\lambda) \in \sigma(f(a))$.

\noindent
Proof:

$f(\lambda) = f(a) = \sum_{n=0}^\infty \alpha_n\lambda^n - \sum_{n=0}^\infty \alpha_na^n = \sum_{n = 1}^\infty \alpha_n(\lambda^n - a^n)$\\

$ = \sum_{n = 1}^\infty \alpha_n(\lambda - a)(\lambda^{n - 1} + \lambda^{n - 2}a + \cdots + \lambda a^{n - 2} + a^{n - 1})$\\

Call the telescoping sum $P_n(\lambda, a)$.\\

Then $\|P_n(\lambda, a)\| \leq n\|a\|^{n - 1}$

Continue, equals $(\lambda - a) \sum \alpha_n P_n(\lambda, a)$

$f'(z) = \sum_{n = 1}^\infty \alpha_n n z^{n - 1}$, converges absolutely uniformly for $|z| \leq \|a\|$.

$\sum_{n = 1}^\infty \alpha_nP_n(\lambda, a) = b \in A$.

So $f(\lambda) - f(a) = (\lambda - a)b$

If $(f(\lambda) - f(a))$ has inverse $c$, then $1 = (\lambda - a)bc$, so $\lambda - a)$ is invertible.

Thus since $\lambda \in \sigma(a)$, $f(\lambda) = f(a)$ is not invertible so $f(\lambda) \in \sigma(f(a))$.

\noindent
This proof above is the beginnings of the spectral mapping theorem.\\

\noindent
Consider $f(z) = z^n$.

Then if $\lambda \in \sigma(a)$ then $\lambda^n \in \sigma(a^n)$.

Thus $|\lambda^n| \leq \|a^n\|$, so $|\lambda| \leq \|a^n\|^{1/n}$.

Thus $|\lambda| \leq inf_n\{\|a^n\|^{1/n}\}$.

\noindent
Corollary: $r(a) \leq inf_n\{\|a^n\|^{1/n}\}$.

This expression doesn't depend on the containing algebra.\\

\noindent
Consider the resolvant of a, at $\infty$.

$R(a, z^{-1}) \frac{1}{z - a} = z(1 - az)\inv = z\sum_{n = 0}^\infty a^nz^n$ converges for $\|az\| < 1$ i.e. $|z| < \|a\|^{-1}$\\

\noindent
However since $R(a, z)$ is analytic for $|z| > r(a)$, $R(a, z^{-1})$ is analytic for $|z| < r(a)\inv$.

Power series converges in the largest circle in which it's analytic.

Should converge in the larger circle.  Note we are in Banach-algebra-valued functions.

Right now we assume we don't know that, so use method from before.

Composing with linear operators on the algebra.

For any $\varphi \in A'$ the dual space of the algebra, $f_\varphi(z) = \varphi(z(1 - az)\inv)$.

Then $f_\varphi$ is an ordinary holomorphic function, holomorphic in $\{z: |z| < r(a)\inv\}$.

So the power series expansion of $f_\varphi$ about 0 converges for $|z| < r(a)\inv$.

But the power series for $f_\varphi$ is $z\sum_{n = 0}^\infty \varphi(a^n)z^n$.

This will converge for $|z| < r(a)\inv$.

Thus for any $r > r(a)$, $z\sum_{n = 0}^\infty \varphi(a^n)z^n$ will converge absolutely and uniformly for $|z| \leq r\inv$.

So there is $M_\varphi$ such that $|\phi(a^n)||z^n| \leq M_\varphi \forall n$, $\forall z$ with $|z| \leq r\inv$.

So $|\varphi(a^n)r^{-n}| \leq M_\varphi$ for all $n$.

For each $n$ define $F_n \in A''$ by $F_n(\varphi) = \varphi(a^n)r^{-n}$.

Thus $|F_n(\varphi)| \leq M_\varphi$ for all $\varphi$.

Recall the uniform boundedness theorem (consequence of the Baire category theorem).

Says here that $\exists M$ such that $\|F_n\| \leq M$ $\forall n$.

But it is clear that $\|F_n\| = \|a^n\|r^{-n}$.

Thus $\|a^n\|r^{-n} \leq M$ i.e. $\|a^n\| \leq Mr^n \to \|a^n\|^{1/n} \leq M^{1/n}r$

As $n \to \infty$, $M^{1/n} \to 1$.

So the $\limsup_{n \to \infty} \|a^n\|^{1/n} \leq r$ $\forall r > r(a)$ (Box it!)

Thus $\limsup_{n \to \infty}\|a^n\|^{1/n} \leq r(a)$.

And we have $r(a) \leq \inf_n\{\|a^n\|^{1/n}\}$.

\noindent
Thus, Theorem: $r(a) = \lim\|a^n\|^{1/n}$.  In particular, this limit exists.

(Gelfand's spectral radius formula).

Just needed a unital Banach algebra over $\mathds{C}$.

\noindent
Corollary: $r(a)$ does not depend on the containing algebra.\\

\noindent
Corollary: Let A be a commutative Banach algebra with 1 over $\mathds{C}$.

Then the Gelfand transform $a \mapsto \hat{a}$ from $A$ to $C(\hat{A})$ is isometric exactly if $\|a^2\| = \|a\|^2$ for all $a \in A$.

\section{9/16}

\noindent
Proposition: Let A be a commutative unital Banach algebra.

The Gelfand transform is isometric iff $\|a^2\| = \|a\|^2$ for all $a \in A$.

\noindent
Proof:

Forward: If equality holds then $\|a^4\| = \|a^2\|^2 = \|a\|^4$.

$\|a^{2^n}\| \cdots = \|a\|^{2^n}$. So $\|a^{2^n}\|^{1\slash{2^n}} = \|a\|$.

So $r(a) = \|a\|$, $r(a) \to \|\hat{a}\|_\infty$.

Conversely: If for some a we have $\|a^2\| = s^2 \leq \|a\|^2$, then

$\|a^4\| \leq \|a^2\|^2 \leq s^4$.

$\|a^{2^n}\|\cdots \leq s^{2^n}$ so $\|a^{2^n}\| \leq s < \|a\|$ and $r(a) < \|a\|$\\

\noindent
Let $\mathcal{O}$ be an open bounded region in $\mathds{C}$ and A be the bounded holomorphic fns on $\mathcal{O}$

That's a unital Banach algebra.  Pretty clear that $\|f^2\|_\infty = \|f\|_\infty^2$.

The Gelfand transform will be isometric.

Will take functions to themselves... well there will be some things on the boundary.

Let's add the requirement that the function is continuous on the boundary.

i.e. consider continuous functions on boundary $\overline{\mathcal{O}}$ holorphic on the interior

Gelfand transform isometric but not onto $C(\mathcal{O})$.\\

\noindent
Let A be an algebra over $\mathds{C}/\mathds{R}$.

An involution on A is a map $a \to a^*$.

$(a + b)^* = a^* + b^*$

$(\alpha a)^* = \overline{\alpha}a^*$

$(ab)^* = b^*a^*$

$(a^*)^* = a$

Given $a$, set $a = \frac{a + a^*}{2} + i \frac{a - a^*}{2i}$ with real and imaginary part, which are self-adjoint

\noindent
A *-algebra is going to be an algebra equipped with an involution

\noindent
A normed *-algebra is a *-algebra that is a normed algebra, with $\|a^*\| = \|a\|$.

\noindent
And similarly Banach *-algebra\\

\noindent
A *-algebra is symmetric if for any $a \in A$ with $a^* = a$ (self-adjoint), $\sigma(a) \subset \mathds{R}$.

\noindent
A commutative algebra is semi-simple if the intersection of all its maximal ideals is $\{0\}$

\noindent
Proposition: Let A be a unital Banach algebra, commutative.

Then the Gelfand transform for A is one-to-one, exactly if A is semi-simple.
No real content in the proposition, just relating the terminology.

\noindent
Let A be a unital Banach *-algebra, commutative.

If it is symmetric, then the image of A under the Gelfand transform is a dense *-subalgebra of $C(\hat{A})$.

\noindent
Proof

Always the image of A separates the points of $\hat{A}$.  Image contains 1.

By separating the the image is closed under complex conjugation, because if 

$a = b + ic$, $b, c$ self-adjoint

$\hat{a} = \hat{b} + i\hat{c}$, $\hat{b}, \hat{c}$ are real-valued

$a^* = b - ic$, $\hat{a^*} = \hat{b} - i\hat{c} = (\hat{a})^-.$

Apply Stone-Weierstrass.

\noindent
If, in addition, $\|a^2\| = \|a\|^2$ for all $a \in A$:

then the Gelfand transform is an isometric *-isomorphism of A onto $C(\hat{A}$.\\

\noindent
continuous functions on a compact hausdorff space is an example\\

\noindent
By a C*-algebra we mean a Banach *-algebra such that $\|a^*a\\ = \|a\|^2$ for all $a \in A$.

e.g. $C(X)$, $C_\infty(X)$ for $X$ locally compact.

\noindent
Proposition: Let $\mathcal{H}$ be a Hilbert space, $T \in B(\mathcal{H})$, then $\|T^*T\| = \|T\|^2$.

\noindent
Proof:

$\|T^*T\| \leq \|T^*\|\|T\| = \|T\|^2$.

For any $\xi \in \mathcal{H}$, $\|T\xi\|^2 = \langle T\xi, T\xi \rangle = \langle T^*T\xi, \xi \rangle \leq$ (Cauchy-Schwartz) $\|T^*T\|\|\xi\|^2$

so $\|T\| \leq \|T^*T\|^{1\slash{2}}$.\\

\noindent
Any closed *-subalgebra of $B(\mathcal{H})$ is a C*-algebra. (C from closed, * from *).\\

\noindent
Proposition: Let A be a unital C*-algebra.  Let $a \in A$, with $a^* = a$, then $\|a^{2^n}\| = \|a\|^{2^n}$, $r(a) = \|a\|$.

\noindent
Corollary: For a unital C*-algebra, its norm is determined by its *-algebra structure.

\noindent
Proof:

Given $a \in A$, $\|a\|^2 = \|a^*a\| = r(a^*a)$.

\noindent
The C*-algebras are by far the nicest-behaving Banach algebras.

Tied into this correspondence between *-algebra structure and norm.\\

\noindent
Let V be a finite-dimensional vector space over $\mathds{C}$.

Then $\mathcal{L}(V)$ the algebra of linear operators on V.

For each inner product on V, get a * on A and a norm on V and so a norm on A.

C*-algebra.

\section{9/21}

Let A be a C*-algebra with 1.

Then A is symmetric, i.e. if $a \in A$ and $a^* = a$, then $\sigma(a) \subset \mathds{R}$.

(in C*-subalgebra generated by a and $1_a$)

\noindent
Proof: (Arens' trick)

Let $a \in A$, $a\conj = a$, let ($r, s \in mathds{R}$) $r + is \in \sigma(a)$.

Need to show $s = 0$.  Given $t \in \mathds{R}$.  Let $b = a + it$.

Then $b\conj b = (a - it)(a + it) = a^2 + t^t$

Thus $\|b\|^2 = \|b\conj b\| \leq \|a\sq\| + t\sq$.

$r + i(s + t) \in \sigma(b)$.  Thus $|r + i(s + t)|\sq \leq \|b\|\sq$

$r\sq + (s + t)\sq \leq \|b\|\sq \leq \|a\sq\| + t\sq$

$r\sq + s\sq + 2st + t\sq$

$\leq \|a\sq\|$ $\forall t \in \mathds{R}$

So $s = 0$.

See the power of this little identity $\|a\conj a\|=\|a\|\sq$\\

\noindent
Theorem (Gelfand-Naimark ``little'')

Let A be a unital commutative C*-algebara.  Then the Gelfand transform is an isometric *-isomorphism of A onto $C(\hat{A})$.

\noindent
Proof:

Now have all the pieces we need to apply the Stone-Weierstrass theorem.\\

\noindent
An element a in a *-algebra is normal if $a\conj$ commutes with a.

(e.g. a is unitary, i.e. $a\conj a = 1 = aa\conj$)

\noindent
Continuous functional calculus:

Let A be a unital C*-algebra (e.g. C*-subalgebra of some $B(\mathcal{H}) \ni T$)

Let $a \in A$, and let a be normal.

Let $f \in C(\sigma(a))$.

Let B be the C*-subalgebra generated by a and 1 (so $a\conj \in B$).

If $\phi$ is a multiplicative linear functional on a then $\phi(a) \in \sigma(a)$

$\hat{B} "=" \sigma(a)$ by $\phi \mapsto \phi(a)$

\end{document}
