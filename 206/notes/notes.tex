\documentclass[12pt]{article}

\usepackage{mathptmx}
\usepackage{mathpazo}
\usepackage{mathrsfs}

\usepackage{amsfonts}
\usepackage{amsmath}
\usepackage{amssymb}
\usepackage[version-1-compatibility]{siunitx}
\usepackage{fixltx2e}
\usepackage{multirow}
\usepackage{dsfont}

\newcommand{\inv}{^{-1}}
\newcommand{\n}{\noindent}

\topmargin -0.5in %topmargin=1in+\topmargin
\textheight 9in % default is letter % so 11-2=9in
\textwidth 6.5in % and 8.5-2in=6.5in
\oddsidemargin 0in %left margin = % 1in + oddsidemargin
\footskip 1cm % pagenumber to text

\setcounter{secnumdepth}{0}

\title{Math 206}
\date{\normalsize Fall 2015}


\begin{document}
\maketitle

\section{8/26}

\subsection{Definitions}

\noindent
A \textit{norm} on a vector space X (over $\mathds{F}$) is a function $\| \|:X \rightarrow \mathds{R}^+$ such that

$\|x\| = 0$ iff $x = 0$

$\|\alpha x\| = |\alpha|\|x\|$ (for $\alpha \in F$)

$\|x + y\| \leq \|x\| + \|y\|$

\noindent
An \textit{algebra} $\mathscr{A}$ over $\mathds{F}$ is a vector space with distributive $\cdot$ satisfying

$cx \cdot y = c(x \cdot y)$

$x \cdot cy = c(x \cdot y)$ for all $c \in F$


\noindent
A \textit{normed algebra} over $\mathds{R}$ or $\mathds{C}$ is an algebra $\mathscr{A}$ equipped with (vector space) norm satisfying

$\|ab\| \leq \|a\|\|b\|$ for all $a,b \in \mathscr{A}$

\noindent
A norm on $\mathscr{A}$ induces a metric 

$d(a, b) = \|a - b\|$ on $\mathscr{A}$ and therefore a topology

if $\mathscr{A}$ is complete for this norm, it is a \textit{Banach algebra}

\subsection{To figure out (use https://www.math.ksu.edu/~nagy/real-an/2-05-b-alg.pdf)}

\noindent
Supposing $\mathscr{A}$ is not necessarily complete

$\|ab\| \leq \|a\|\|b\|$ gives uniform continuity on the product

hence the norm can be extended to the completion $\bar{\mathscr{A}}$ to form a Banach algebra

\noindent
A metric space M is complete if all Cauchy sequences converge to an element of M

\noindent
The completion M is all equivalence classes of Cauchy sequences where 

$\{a_n\} \sim \{b_n\}$ iff $\lim\limits_{x \to \infty}d(a_n - b_n) = 0$

\subsection{Examples}

\noindent
For M a compact space, $C(M)$

the set of continuous $\mathds{R}/\mathds{C}$-valued functions on M

pointwise operations

$\|f\|_\infty = sup \{|f(x)| : x \in M\}$

\noindent
For M locally compact, $C_\infty(M)$

the set of continuous $\mathds{R}/\mathds{C}$-valued functions on M vanishing at $\infty$

vanishing at $\infty$: $\forall \epsilon \exists$ a compact subset of M, outside of which $f < \epsilon$

note that this is non-unital (lacks an identity)

\noindent
For $\mathscr{O} \subset \mathds{C}^n$ open

$H^\infty(\mathscr{O})$ the set of all bounded holomorphic functions on $\mathscr{O}$

\noindent
(M, d) metric space and $f \in C(M)$

Lipschitz constant (which can be $+\infty$) $L_d(f) = sup\{\frac{|f(x) - f(y)|}{d(x, y)} : x, y \in M, x \neq y\}$

The \textit{Lipschitz functions} $\mathscr{L}_d(M, d) = \{f: L(f) < \infty\}$

These form a dense subalgebra of C(M) and are in fact a *-subalgebra

$\|f\|_d$ := $\|f\|_\infty + L_d(f)$, can be shown as a normed-algebra norm

$L_d(M, d)$ is complete for this norm

so $L_d(M, d)$ is a Banach algebra

$L_d$ is a seminorm on $\mathscr{L}_d(M, d)$ since it takes value 0 on the constant functions

can recover d from $L_d$

\noindent
M a differentiable manifold (e.g. $T = \mathds{R}\slash\mathds{Z}$ the circle)

C(M) $\supseteq C^{(1)}(M)$ the singly-differentiable functions

$f \in C^{(2)}(T) \to Df: T_xM \to \mathds{R}, \mathds{C}$

with $Df$ the derivative and $T_x$ the tangent space

If we put on a Riemmannian metric, define $\|f\|^{(1)} = \|f\|_\infty + \|Df\|_\infty$

If $f \in C^{(1)}(T): \|f\|^{(1)} = \|f\|_\infty + \|f'\|_\infty$

Banach algebra norm, for which this space of functions is complete

\noindent
For the circle, $C^{(2)}(T) \to \|f\|^{(2)} = \|f\|_\infty + \|f'\|_\infty + \frac{1}{2}\|f''\|_\infty$

the factor $\frac{1}{2}$ ensures that this satisfies the normed algebra condition

$C^{(n)}(T) = \sum\limits_{k=0}^n\frac{1}{k!}\|f^{(k)}\|_\infty$

For $C^\infty(T)$ using the collection of norms $\{\|\|^{(n)}\}^\infty_{n=1}$ yields a Fr\'{e}chet algebra

A Fr\'{e}chet algebra has a topology defined by a countable family of seminorms

that respect the algebra structure and is complete \textbf{(clarify)}

\noindent
non-commutative algebras

X a Banach space

$\mathscr{B}(X)$ the algebra of bounded operators on X

$\|\|$ operator norm $\to$ Banach algebra

Any closed subalgebra of $\mathscr{B}(X)$ is a Banach algebra

\section{8/28}

\noindent
Sketch of the course

\noindent
X a Banach space, $B(X)$ bounded functions on the space

\noindent
$\mathscr{H}$ a Hilbert space, $\mathscr{B}(\mathscr{H})$ bounded operators on the space

for $T \in \mathscr{B}(\mathscr{H}) \exists$ adjoint operator $T^* \in \mathscr{B}(\mathscr{H})$

$<T\xi, \eta> = <\eta, T^*\xi>$ for $\xi, \eta \in \mathscr{H}$

adjoint is additive, conjugate linear, $T^{**} = T$, $(ST)^* = T^*S^*$

\noindent
An algebra $A$ over $\mathds{R}$ or $\mathds{C}$ is a \textit{*-algebra} if it has a $*: A \to A$ satisfying

certain properties (look up)

\noindent
A \textit{*-normal algebra} is a normal *-algebra such that

$(\forall a \in A)\|a^*\| = \|a\|$

\noindent A \textit{Banach *-algebra} is a *-normal algebra that is a Banach algebra.

\noindent
For any $T \in \mathscr{B}(\mathscr{H})$, have $\|T^*T\| = \|T\|^2$ \textbf{(check: parse through defns)}

\noindent
For M a locally compact space, $A = C_\infty(M, \mathds{C})$, $f^*:= \bar{f}$ is a Banach *-algebra

Also have $\|f^*f\| = \|f\|^2$ \textbf{(verify: should be easier than the other)}

\noindent Little Gelfand-Naimark theorem:

Let $A$ be a commutative Banach *-algebra satisfying $\|a^*a\| = \|a\|^2$.

Then $A \cong C_\infty(M)$ for some locally compact M.

\noindent
One view of the ``spectral theorem''

Let $T \in \mathscr{B}(\mathscr{H})$ with $T^* = T$

Let $A$ be the closed subalgebra of $\mathscr{B}(\mathscr{H})$ generated by T and I (i.e. $p(T) := \Sigma \alpha_kT^K$)

Polynomials closed or stable under *

If $S \in A$ then $S^* \in A$ (i.e. $A$ is a *-subalgebra of $\mathscr{B}(\mathscr{H})$

So $A$ is a Banach *-suubalgebra satisfying $\|S^*S\| = \|S\|^2$

Moreover, $A$ is commutative.  (unital, since generated by I)

Then by the Little Gelfand-Naimark theorem, $A \cong C(M)$

Indeed $M \subset \mathds{R}$, the spectrum of T

If $\mathscr{H}$ is finite dimensional, then $M$ is the set of eigenvalues of T

T is normal if $TT^* = T^*T$

\noindent
A C*-algebra is a Banach *-algebra over $\mathds{C}$ satisfying

$\|a^*a\| = \|a\|^2$

\noindent
Theorem: A commutative C*-algebra is $\cong C_\infty(M)$.

\noindent
Big Gelfand-Naimark Theorem: (\textit{Math 208, C*-algebras})

Any C*-algebra is $\cong$ to a closed *-subalgebra of $\mathscr{B}(\mathscr{H})$ for some Hilbert space $\mathscr{H}$.

\noindent
Tangent

algebraic topology, differential geometry, Riemann manifolds, ``non-commutative geometry'' (Connes)

\noindent
A \textit{von-Neumann algebra} is a *-subalgebra of $\mathscr{B}(\mathscr{H})$

which is closed under the strong operator topology.

Every commutative von-Neumann algebra is $\cong L^\infty(X, S, \mu)$ (measure spaces) acting on $L^2(X, S, \mu)$ by positive sldkjfalksdjf

\noindent
For group G, $\alpha: G \to Auto(X) \subseteq \mathscr{B}(X)$

$Auto(X)$ a Banach space

Look at subalgebra of $\mathscr{B}(X)$ generated by $\alpha(G)$.

Leads to considering $l'(G)$ with product $(f \star g)(x) = \Sigma f(y)g(y^{-1}x)$ convolution

$f^*(x) = \overline{f(x^{-1})}$

Banach *-algebra, G commuatative $\to$ Fourier transform

\section{8/31}

\noindent
K a field, X a set, $\mathscr{F}(X, K)$ the set of all K-valued functions on X with pointwise operations

Given $f \in \mathscr{F}(X, K)$.

Let $\lambda \in K$.  Then $\lambda \in$ range(f) exactly if $(f - \lambda 1)$ is not invertible.

\noindent
For any $a \in A$, the \textit{spectrum} of a is $\{\lambda \in K: a - \lambda 1_A$ is not invertible in A\}.

The spectrum depends on the containing algebra

Assuming that this algebra A  (over the field K) has an identity $1_A$

\noindent
Example: Let A = C([0, 1]), and B = polynomials, viewed as a dense subalgebra of A.

Let p be a polynomial of degree $\geq 2$.  Then

$\sigma_a(p) = p([0, 1])$.

$\sigma_B(p) = \mathds{R}, \mathds{C}$

\noindent
Let A be a Banach algebra with 1 ($\|1\| = 1$), and $a \in A$.

If $\|a\| < 1$, then 1 - a is invertible, and $\|(1 - a)^{-1}\| \leq \frac{1}{1 - \|a\|}$

\noindent
Proof:

$\frac{1}{1 - a} = \sum_{n = 0}^{\infty} a^n$ ($a^0 := 1_A$)

For any $n > 0$, let $s_n = \sum_{k = 0}^na^k$.

Show that $\{s_n\}$ is a Cauchy sequence.

If $n > m$, $\|s_n - s_m\| = \|\sum_{k = m + 1}^na^k\| \leq \sum_{k = m + 1}^n \|a\|^k$.

Given $\epsilon > 0 \exists N$ such that if $m, n \geq N$ then $\sum_{k = m + 1}^n \|a\|^k \leq \epsilon$

So $\{s_n\}$ is a Cauchy sequence.

By completeness there is a $b \in A$ with $s_n \to b$ as $n \to \infty$.

Want to show $b = (1 - a)^{-1}$.

$b(1 - a) = lim_{n \to \infty}(s_n(1 - a))$

$ = lim_{n \to \infty}(1 + a + a^2 + a^3 + \cdots+a^n - (a + a^2 + a^3 + \cdots + a^{n + 1}))$

$= lim_{n \to \infty}(1 - a^{n + 1}) = 1$

Then $1 - (1 - a)$ is invertible, i.e. a is invertible.

$\|(1 - a)^{-1}\| = \lim\|s_n\| \leq \lim\sum_{k = 0}^n\|a^k\| = \frac{1}{1 - \|a\|}$  ($\|1\| = 1$)

\noindent
$\|ab\| \leq \|a\|\|b\|$: can very easily check that multiplication is cts (do this?)

\noindent
Corollary: If $a \in A$ and $\|1 - a\| < 1$ then a is invertible, and $\|a^{-1}\| \leq \frac{1}{1 - \|1 - a\|}$

I.e. the open unit ball about 1 consists of invertible elements.

\noindent
Let $a \in A$. Let $L_a$, $R_a$ be the operators of left and right multiplication by a on A.

$a \to L_a$ is an algebra homomorphism of A into $\mathscr{L}(A)$ (linear operators on A)

$L_aL_b = L_{ab}$, $R_aR_b = R_{ba}$ (R is an antihomomorphism)

$1 \in A$

If a is invertible, then so is $L_a$, $L_aL_{a^-1} = I_a$

Then if A is a normed algebra, $\|L_a\| = \|a\|$

$\|L_{ab}\| = \|ab\| \leq \|a\|\|b\|$

$\|L_a1_a\| = \|a\|$

so if $a \in A$ is invertible, then $L_a$ is a homeomorphism of A onto itself.

Thus if $A$ is a Banach algebra, with 1, and a is invertible:

$\{L_{a}b : \|1 - b\| < 1\}$ is an open neighborhood of a consisting of invertible elements

\noindent
Let GL(A) be the set of invertible elements of A. (general linear group)

Then (for A a unital Banach algebra) GL(A) is an open subset of A.

(Fails for Poly $\subseteq$ C([0, 1]))

\noindent
Two Fr\'{e}chet algebras, for one, GL(A) is an open subset, for another it isn't.

\textbf{ask about this?}: not sure what he was talking about

$C^\infty(T)$, $\|f^{(n)}\|$

C(R) cont fns on $\mathds{R}$ (or $\mathds{C}$) maybe unbounded

For each n let $\|f\|_n = $ sup$\{|f(t)|: |t| \leq n\}$


\noindent
Corollary: For A a Banach algebra with 1 and $a \in A$, $\sigma(a)$ is a closed subset of $\mathds{C}$

\section{9/2}

\noindent
Proposition: Let A be a unital Banach algebra and $a \in A$.

Then $\sigma(a)$ is a closed subset of $\mathds{C}$ or $\mathds{R}$.  If $\lambda \in \sigma(a)$ then $\|\lambda\| \leq \|a\|$.

\noindent
Proof: $\sigma(a) = \{\lambda : (a - \lambda)$ is not invertible\}

Its complement, the \textit{resolvant set}, of a is $\{ \lambda: (a - \lambda) \in GL(A)\}$, is open.

If $|\lambda| > \|a\|$ then $(\lambda - a) = \lambda(1 - \frac{a}{\lambda})$, $\|a\slash \lambda \| < 1$

so $(\lambda - a)$ is invertible, ie $\lambda \in \sigma(a)$.

Over $\mathds{R}$, can have $\sigma(a) = \emptyset$, e.g. $\begin{pmatrix} 0 & -1 \\ 1 & 0 \\ \end{pmatrix}$

\noindent
``If $a \in GL(A)$ and b is close to a then $b^{-1}$ is not much bigger than $a^{-1}$''.

Let $\mathscr{O} = \{c : \|1 - c\| < 1\slash 2 \}$

So c is invertible, and $\|c^{-1}\| \leq \frac{1}{1 - \|1-c\|} \leq 2\}$

Let $b \in a\mathscr{O}$, so $b = ac$ for $c \in \mathscr{O}$, then $\|b^{-1}\| = \|c^{-1}a^{-1}\| \leq 2\|a^{-1}\|$.

For $a, b \in GL(A)$.

$b^{-1} - a^{-1} = b^{-1}(a - b)a^{-1}$

Thus $\|b^{-1} - a^{-1}\| \leq \|b^{-1}\|\|a - b\|\|a^{-1}\|$.

So $b \to b^{-1}$ is continuous for the norm.

So $GL(A)$ is a topological group for topology from norm.

$b^{-1} = (1 + b^{-1}(a - b))a^{-1}$

\noindent
On $\rho (a)$ (the resolvant set, complement of the spectrum) define the resolvant of a

This is the function $R(a, \lambda) = (\lambda - a)^{-1}$

\noindent
$R(a, \lambda)$ is an analytic function on $\rho (a)$.

\noindent
Proof: Let $f(z) = R(a, z)$.

$f'(z) = \lim_{h \to 0} \frac{f(z + h) - f(z)}{h} = \frac{(z + h - a)^{-1} - (z - a)^{-1}}{h}$ 

$ = \lim_{h \to 0}\frac{1}{h} (z + h - a)^{-1} ((z - a) - (z + h - a))(z - a)^{-1}$

$= \lim_{h \to 0} -(z + h - a)^{-1}(z - a)^{-1} = -(z - a)^{-2}$

$f'' = +z(z - a)^{-3}$

Given $z_0 \in \rho (a)$

Will use $b^{-1} = (1 + b^{-1}(a - b))a^{-1}$ and $f(z) = (z - a)^{-1} = \sum c_n(z - z_0)^n$

$f(z) = (z - a)^{-1}$

$b \to z - a$

$f(z) = (1 + (z - a)^{-1}((z_0 - a) - (z - a)) (z_0 - a)^{-1}$

$=(1 + (z - a)^{-1}(z_0 - z))(z_0 - a)^{-1}$

where $(z - a)^{-1}(z_0 - z) \leq 1$ then the above

$=\sum(-1)^{n}(z - a)^{-n}(z - z_0)^n = \sum(-1)^n(z-a)^{-n-1}(z-z_0)^n$

a proper power series expansion.

\noindent
Examine $R(a, z)$ at $\infty$.

$R(a, z^{-1}) = (z^{-1} - a)^{-1} = \frac{1}{z^{-1} - a}$ 

$ =z(1 - za)^{-1}$ (for small z, ie $\|za\| < 1$)

$R(a, z^{-1})$ approaches 0 as $z \to 0$.

defn $R(a, 0^{-1}) = 0$, see $R(a, z)$ is analytic at $\infty$.

\noindent
Theorem: For a Banach algebra over $\mathds{C}$ with 1, and for any $a \in A$,

$\sigma(a) \neq \emptyset$, that is, the spectrum is non-empty.

\noindent
Proof: Suppose that $\sigma(a) = \emptyset$.

Then $R(a, z)$ is defined on all of $\mathds{C}$ and is bounded.

By Liouville's, $R(a, z)$ is constant, $= 0$, $(a - z)^{-1} = 0$ $ \forall z$

Why can we use Liouville's in this Banach space case?

Let A' be the dual Banach space to A.

For $\varphi \in A'$, $z \mapsto \varphi(R(a, z))$ is a $\mathds{C}$-valued analytic function.

So set $\varphi(R(a, z)) = 0$ $\forall z$, $\forall \varphi$

so $R(a, z) = 0$.

Knowing that there is anything in here is the Hahn-Banach Theorem, depending on the axiom of choice.

\noindent
Theorem (Gelfand-Mazer)

Let A be a unital Banach algebra over $\mathds{C}$.

If every nonzero element is invertible, then $z \to z1_A$ is an isomorphism from $\mathds{C}$ onto A.

\noindent
Proof:

Given $a \in A$ let $z \in \sigma (a) \neq \emptyset$.

So $(z - a)$ is not invertible so $z - a = 0$.

Fails over $\mathds{R}$ since have $\mathds{R}$, $\mathds{C}$, quaternions

\section{9/4}

Let A be an algebra or ring.  Ideals $I$, left, right, 2-sided.

$A\slash I$, for a left ideal get a left A-module, right ideal get a right A-module

Two-sided get an algebra or ring.

\noindent
Let A be a normed algebra, and if $I$ is an ideal in A.

Then $\overline{I}$ (the closure) is again an ideal in A.

$\{a_n\} \subset I$, $a_n \to c \in A$ then $ba_n \to bc$

\noindent
Proposition: If A is a unital Banach algebra and if I is an proper ideal in A.

Then $\overline{I}$ is proper.

\noindent
Proof:

Use GL(A) is open.  The original ideal cannot contain any invertible elements.

The complement of the invertible elements is going to be closed.

The ideal is in its closure; the closure is in the complement of the invertible elements.

So the closure will not contain any invertible elements.

\noindent
Counter-example: C locally compact, $C_c(\mathds{R}) \subset C_\infty(\mathds{R})$

In fact $C_c(\mathds{R})$ is the minimal dense ideal in $C_\infty(\mathds{R})$

Lack of identity element.

\noindent
Counter-example: Look at A the polynomials viewed as a subset of $C([0, 1])$

Using the sup-norm.

Lack of completeness.

Let $I = \{p: p(2) = 0\}$ is an ideal, in fact a maximal proper ideal.

This ideal of polynomials will be dense inside A and dense inside all polynomials.

\noindent
Counter-example: A = $C(\mathds{R})$, including unbounded, $I = C_C(\mathds{R})$, compact open topology

Compact in here for the Fr\'{e}chet open topology.

\noindent
Corollary: Let A be a unital Banach algebra.

Then every maximal ideal is closed.

\noindent
Taking ideals to form quotients.

Recall that if X is a normed vector space and if Y is a vector subspace of X:

Then we can form the quotient vector space $X \slash Y$.

Have the evident $\pi: X \to X \slash Y$ by $\pi (x) = x + Y$.

Set $\|\pi(x)\| = inf\{\|x - y\| : y \in Y\}$ i.e. the distance from x to the subspace Y.

It is easily seen that this is a seminorm.

Problem: if Y is not closed, then it is not a norm, since $\|\pi(y)\| = 0$ for $y \in \overline{Y}$.

Therefore, if Y is closed, then $\|\|_{X \slash Y}$ is a norm.

\noindent
(Should know from $202B$) If X is a Banach space and Y is a closed subspace.

Then $X \slash Y$ with the norm defined above is a Banach space.

Trickier to prove, but a true statement.

\noindent
Proposition: Let A be a normed algebra and let I be a closed ideal in A.

(can be left or right or two-sided)

Then if I is a two-sided ideal, then $A \slash I$ with $\|\|_{A \slash I}$ is a normed algebra.

That is $\|\pi(a)\pi(b)\| \leq \|\pi(a)\|\|\pi(b)\|$.

Then if I is a left ideal, so that $A \slash I$ with $\|\|_{A \slash I}$ is a left A-module.

$\|a\pi(b)\|_{A \slash I} \leq \|a\|_A\|\pi(b)\|_{A\slash I}$

And if I is a right ideal, so that $A \slash I$ with $\|\|_{A \slash I}$ is a right A-module, similar.

\noindent
Proof:

For I a two-sided ideal.

If $c, d \in I$, then $\|\pi(a)\pi(b)\|_{A \slash I} \leq \|(a - c)(b - d)\| = \|(ab - (cb + ad - cd)\|$

where $cb + ad - cd \in I$.

Take inf over $c, d \in I$.

$\leq \|a - c\| \|b - d\|$

\noindent
Proposition: If a is a Banach algebra and I is a closed 2-sided ideal, then $A\slash I$ is a Banach algebra.

\noindent
An algebra or ring is \textit{simple} if it contains no proper 2-sided ideals.

e.g. $M_n(\mathds{R})$, $M_n(\mathds{C})$, $M_n(K)$ for $K$ a field.

\n
Proposition: Let A be an algebra or ring and let I be a maximal 2-sided ideal in A.

The $A \slash I$ is a simple algebra/ring.

\n
Corollary: If A is a Banach algebra and if I is a maximal closed 2-sided ideal.

Then $A \slash I$ is a simple Banach algebra.

\n
Let A be a commutative algebra/ring with 1.

If $a \in A$ and a is not invertible, then $Aa$ is a two-sided ideal.

And $1 \not\in Aa$ (else it would be invertible).

Thus $Aa$ is a proper ideal.

\n
Corollary: If A is a commutative algebra/ring with 1 and if A is simple, then A is a field.

A is simple: every nonzero element is invertible.

Since if one weren't we would have a proper ideal.

\n
Theorem: Let A be a commutative Banach algebra with 1.

Let I be a maximal ideal in A (which of course is necessarily closed).

Then $A \slash I \cong$ either $\mathds{C}$ (isomorphic in every sense).

\n
Proof:

Banach-Mazer theorem applied to the previous discussion.

\n
$c \in A\slash I$, $z \in \sigma(c)$, $c - z1_{A}$ is noninvertible by def'n spectrum.

Since it is a field, $c - z1_{A} = 0$.

\n
From a maximal ideal, $I$, get a homomorphism $\phi: A \to \mathds{C}$, unital

($\phi(1) = 1$) such that $I = ker(\phi)$

\n
Let A be a Banach algebra with 1 and let $\phi: A \to \mathds{C}$ be a homomorphism.

Then $\phi$ is continuous and $ker(\phi)$ is a maximal 2-sided ideal in A.

$\|\phi\| = 1$

\n
Lemma: For any given $a \in A$, $\phi(a) \in \sigma(a)$

Proof: $\phi(a - \phi(a)1_A) = 0$ so $a - \phi(a)1$ is not invertible.

so $\phi(a) \in \sigma(a)$

Then $\|\phi(a)\| \leq \|a\|$, so $\|phi\| \leq 1$ but $\phi(1) = 1$.


\end{document}
