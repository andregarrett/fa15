\documentclass[12pt]{article}

\usepackage{mathptmx}
\usepackage{mathpazo}

\usepackage{amsfonts}
\usepackage{amsmath}
\usepackage{amssymb}
\usepackage[version-1-compatibility]{siunitx}
\usepackage{fixltx2e}
\usepackage{multirow}
\usepackage{dsfont}

\topmargin -0.5in %topmargin=1in+\topmargin
\textheight 9in % default is letter % so 11-2=9in
\textwidth 6.5in % and 8.5-2in=6.5in
\oddsidemargin 0in %left margin = % 1in + oddsidemargin
\footskip 1cm % pagenumber to text

\setcounter{secnumdepth}{0}

\title{Math 250A}
\date{\normalsize Fall 2015}


\begin{document}
\maketitle

\section{8/27}

\subsection{Group Action}
\noindent
A group G acts on a set S:

$G \times S \to S$

$(g, s) \mapsto g \cdot s$

$e \cdot s = s$

$(g g') \cdot s = g \cdot (g' \cdot s)$

\noindent
Alternatively,

$\phi: G \to Perm(S)$

$\phi$ is a homomorphism (gives the corresponding properties)

$(\phi(g))(s) = g \cdot s$

\subsection{Examples of Group Actions}

\noindent
The trivial action:

$G \to Perm(S)$ where $g \mapsto e_{Perm(S)}$

\noindent
G acting on self by left/right translation, conjugation

\noindent
G acting on the set of subgroups of G by conjugation:

$g \cdot H = gHg^{-1} = \{ghg^{-1} | h \in H\}$

\noindent
Normal subgroup $N \trianglelefteq G$

G acting on N, $g \cdot n := gng^{-1} \in N$

\noindent
$G = S_3$ where S is the set of subgroups of G of order 2.

S = \{\{1, (1 2)\}, \{1, (1 3)\}, \{1, (2 3)\}\}

\noindent recall
$\sigma (a_1, a_2, a_3, ... a_k) \sigma^{-1} = (\sigma a_1, \sigma a_2, \sigma a_3, ... \sigma a_k)$

\noindent
V vector space over a field K

G = GL(V) = group of invertible linear maps $V \to V$

e.g. if $V = K^n$ then G = GL(n, K)

G acts on V (rather simply) by $L \cdot v = L(v)$

\subsection{Orbits and Stabilizers}

\noindent
Given G acting on S by $G \times S \to S$ there is an obvious relation on S:

s, s': $s \thicksim s' \leftrightarrow \exists g \in G, s' = gs $

the orbit of s is just the equivalence class of s under this relation

i.e., $G \cdot s = \{g \cdot s | g \in G\}$

\noindent
The conjugacy classes of s are the orbits of S under the group action of G by conjugation

the orbit of s, $O(s) = \{s\} \leftrightarrow s = gsg^{-1} \forall g$

$\leftrightarrow (\forall g)gs = sg$

$\leftrightarrow s \in Z(G)$ the center of the group

\noindent
Example, for $G = S_3$

the orbit of 1 is \{1\}

the orbit of (1 2) = \{(1 2), (1 3), (2 3)\}

the orbit of (1 2 3) = \{(1 2 3), (1 3 2)\}

\noindent
Stabilizer (isotropy group) of a given element $s \in S := G_s$

$G_s = \{g \in G |g \cdot s = s\}$

stabilizer is closed under inverses: $g \in G_s \to g \cdot s = s \to g^{-1}gs = g^{-1}s \to s = g^{-1}s$

\subsection{large stabilizer$\leftrightarrow$ small orbit}

\noindent
there exists a natural bijection $\alpha: G/G_s \to O(s)$ defined $gG_s \mapsto g \cdot s$


\noindent
well-definition:

if $g_1G_s = g_2G_s$
then $\exists g \in G_s, g_1 = g_2g$
and $\alpha(g_1G_s) = g_1 \cdot s = g_2 g s = g_2 s = \alpha(g_2G_s)$

\noindent
injectivity:

if $\alpha(g_1G_s) = g_1 \cdot s = g_2 \cdot s = \alpha(g_2G_s)$
then $g_2^{-1}g_1 \cdot s = s$, $g_2^{-1}g_1 \in G_s$ and $g_1G_s = g_2G_s$

\section{Lang 1.1-1.5}

\subsection{1.1: Monoids}

\noindent
A \textit{monoid} is a set with associative binary operation and unit element.

Abelian $\leftrightarrow$ commutative

\noindent
A \textit{submonoid} is a subset of a monoid with identity and closure under the operation

Such a submonoid is, itself, a monoid

\subsection{1.2: Groups}

\noindent
A \textit{group} is a monoid with inverses for each element

\noindent
The \textit{permutation group} of S is the set of all bijections $S \to S$ (with composition as product)

\noindent
A direct product of groups has product defined componentwise

\noindent
A \textit{subgroup} of a group is a subset closed under composition and inverse

\noindent
$S \subset G$ \textit{generates} G if $\forall g \in G, g = \prod s_i$, where $s_i \in S$ or $s_i^{-1} \in S$

$G = \langle S \rangle$

\noindent
The \textbf{group of symmetries of the square} is a non-abelian group of order 8

generated by $\sigma$, $\tau$ such that $\sigma^4 = \tau^2 = e$ and $\tau \sigma \tau^{-1} = \sigma^3$

\noindent
The \textbf{quaternions} are a non-abelian group of order 8

generated by i, j where defining $k = ij$, $m = i^2$

$i^4 = j^4 = k^4 = e$, $i^2 = j^2 = k^2 = m$, and $ij = mji$

\noindent
A \textit{monoid-homomorphism} $f: G \to G'$ satisfies $f(xy) = f(x)f(y)$ and $f(e_G) = e_{G'}$

If G and G' are groups, f is a group homomorphism ($f(x^{-1}) = f(x)^{-1}$ is implied)

\noindent
An \textit{isomorphism} is a bijective homomorphism.

An \textit{automorphism} or \textit{endomorphism} of G is an isomorphism $\varphi: G \to G$

\noindent
The group \textbf{Aut(G)} is the set of all automorphisms of G

\noindent
The \textit{kernel} of a homomorphism $f: G \to G'$ is $\{g \in G : f(g) = e_{G'}\}$

the kernel and the image f are subgroups of their respective groups

\noindent
An \textit{embedding} is a homomorphism $f: G \to G'$ where $G \cong Im(f)$.

\noindent
Fact: A homomorphism with trivial kernel is injective.

Forward is obvious.

Supposing trivial kernel: $f(x) = f(y) \leftrightarrow f(x)[f(y)]^{-1} = e \leftrightarrow f(xy^{-1}) = e \leftrightarrow xy^{-1} = e$

\noindent
For G a group, and $H, K \leq G$ such that $H \cap K = e$, $HK = G$, and $xy=yx$ $\forall x \in H$ $\forall y \in K$

The map $H \times K \to G$ defined $(x, y) \mapsto xy$ is an isomorphism

This generalizes to finitely many such subgroups by induction

\noindent
A \textit{left coset} of H in G ($H \leq G$) is $aH = \{ax : x \in H\} \leq G$

$x \mapsto ax$ gives bijection between cosets of H, are all of equal cardinality

The \textit{index} of H in G (G : H) is the number of cosets of H in G (right or left)

The \textit{order} of G is the index (G : 1) of its trivial subgroup

\noindent
For any subgroup H of G, G is the disjoint union of its cosets in H

\noindent
For $H \leq G$, $(G : H)(H : 1) = (G : 1)$, holding if at least two are finite

If (G : 1) is finite, the order of H divides the order of G.

\noindent
Given:

$H, K \leq G$, $K \subset H$

$\{x_i\}$ a set of coset representatives of K in H

$\{y_i\}$ a set of coset representatives of H in G

\noindent
Then:

$\{y_jx_i\}$ is a set of coset representatives of K in G.

\noindent
Therefore the above can be generalized to $(G : K) = (G : H)(H : K)$

\noindent
Conclusion: groups of prime order are cyclic.

\noindent
$J_n = \{1, ... , n\}$, $S_n = Perm(J_n)$

$\tau \in s_n$ is a \textbf{transposition} if $\exists r \neq s \in J_n$, $\tau(r) = s$, $\tau(s) = r$, $\tau(k) = k$ $\forall k \neq r, s$

The set of transpositions generate $S_n$

Consider $H \leq S_n$ those which leave $n$ fixed. Then $H \cong S_{n-1}$.

Now if $\sigma_i \in S_n$ for $1 \leq i \leq n$ are defined with $\sigma_i(n) = i$, $\{\sigma_i\}$ are coset reps for H

Hence $(S_n : 1) = n(H : 1) = n!$. 

\subsection{1.3: Normal subgroups}

\noindent
For H the kernel of $f: G \to G'$ a group-homomorphism, $xH = f^{-1}(f(x)) = Hx$

Such a relation is equivalent to e.g. $xH \subset Hx$ and $H \subset xHx^{-1}$

A subgroup $H \trianglelefteq G$ (satisfying $xHx^{-1} = H$ $\forall x \in G$) is termed \textit{normal}

H is normal $\leftrightarrow$ H is the kernel of some homomorphism

\noindent
The \textit{factor group} of G by $H \trianglelefteq G$ is the group of cosets, denoted $G\slash H$

$f: G \to G\slash H$ defined $x \mapsto xH$ is the canonical map for H

\noindent
The \textit{normalizer} $N_S$ of $S \subset G$ is $\{x \in G | xSx^{-1} = S \}$

The normalizer of H is the largest subgroup of G in which H is normal

\noindent
The \textit{centralizer} $Z_S$ of S is $\{x \in G|xyx^{-1} = y$ $\forall y \in S\}$

The centralizer of G is called its \textit{center}; its elements commute with all others in G

\noindent
The \textbf{special linear group} is the kernel of the determinant (a homomorphism)

\noindent
G is the \textit{semidirect product} of N and H if $G=NH$ and $H\cap N = \{e\}$

\noindent
An \textit{exact} sequence $G' \xrightarrow{f} G \xrightarrow{g} G''$ satisfies $Im(f) = Ker(g)$.

Can extend to larger sequences as long as each triple satisfies the above

\noindent
Some canonical homomorphisms, given $f: G \to G'$

$H = ker(f)$ $\to$ $\exists !f': G\slash H \to G'$ injective $\to$ $\exists\lambda: G\slash H \to Im(f)$ an isomorphism

$H \leq G$, N the minimal $N \trianglelefteq G$ s.t. $H \leq N$, $H \subset ker(f)$, then $N \subset ker(f)$, $\exists ! f': G\slash N \to G'$

$H, K \trianglelefteq G$, $K \subset H$, then $K \trianglelefteq H$ $\to$ $(G\slash K)\slash (H \slash K) \cong G \slash H$

$H, K \leq G$, $H \subset N_K$ $\to$ $H \cap K \trianglelefteq H$, $HK = KH \leq G$, $\to$ $H\slash(H\cap K) \cong HK\slash K$

$H' \trianglelefteq G'$, $H = f^{-1}(H')$ $\to$ $H \trianglelefteq G$ $\to$ $\overline{f}: G\slash H \to G'\slash H'$ injective

\noindent
A \textit{tower} of subgroups of G is a sequence $G = G_0 \supseteq G_1 \supseteq G_2 ... \supseteq G_m$

Such a tower is normal if each $G_{i+1} \trianglelefteq G_i$ and abelian if each factor group is abelian

The preimage of a normal tower under a homomorphism is itself a normal tower

And similarly with the preimage of an abelian tower

Inserting finitely many subgroups into a tower yields a \textit{refinement} of that tower

A \textit{solvable} group has an abelian tower with $G_m = \{e\}$

\noindent
An abelian tower of finite G admits a cyclic refinement.

\noindent
$H \trianglelefteq G \to$ G is solvable $\leftrightarrow$ $H$ and $G\slash H$ are solvable

\noindent
A \textit{commutator} in G is an element of the form $xyx^{-1}y^{-1}$

The \textit{commutator subgroup} of G is the subgroup generated by its commutators

\noindent
A \textit{simple} group is a non-trivial group whose only normal subgroups are \{e\} and itself

An abelian group G is simple $\leftrightarrow$ G is cyclic and of prime order

\noindent
$U, V \leq G$, $u \trianglelefteq U$, $v \trianglelefteq V$, then we have the following:

$u(U \cap v \trianglelefteq u(U \cap V)$ and $(u \cap V)v \trianglelefteq (U \cap V)v$ with isomorphic factor groups, that is,

$u(U \cap V)\slash u(U \cap v) \cong (U \cap v)v\slash(u \cap V)v$

\noindent
Two towers $G = G_1 \supseteq G_2 \supseteq \cdots \supseteq G_r$, $G = H_1 \supseteq H_2 \supseteq \cdots \supseteq H_s$ are \textit{equivalent} if:

$r = s$ and $\exists i \mapsto i'$ such that $G_i\slash G_{i+1} \cong H_{i'}\slash H_{i'+1}$

\noindent
Theorem (Schreier): Given a group G and two towers of that group.

If they are normal and end with the trivial group they have equivalent refinements

\noindent
$G = G_1 \supseteq G_2 \supseteq \cdots \subseteq G_r = \{e\}$ normal, each $G_i/G_{i+1}$ simple, $G_i \neq G_{i+1}$ for $1 \leq i \leq r - 1$

Then any normal tower of G with these properties is equivalent to this tower.

\subsection{1.4: Cyclic groups}

A group G is \textit{cyclic} if $\exists a \in G$ such that $\forall x \in G$, $x = a^n$ for some $n \in \mathds{Z}$

Such an a is the \textit{generator} of G.

If $a^m = e$ and $m > 0$ m is an \textit{exponent} of a.

Such is an \textit{exponent of G} if it is an exponent of a $\forall a \in G$.

\noindent
Let G be a group, $a \in G$, $f: \mathds{Z} \to G$ defined $f(n) = a^n$ and $H = ker(f)$

If the kernel is trivial, a has \textit{infinite period} and generates an infinite cyclic subgroup

With a nontrivial kernel, its \textit{period} d is the smallest positive element of the kernel

\noindent
G a finite group, order $> 1$, $a \in G$, $a \neq e$, then the period of a divides n.

\noindent
G cyclic: every subgroup of G is cyclic, and for f a homomorphism on G, $Im(f)$ is cyclic

\noindent
\textbf{Page 24}

\subsection{1.5: Operations of a group on a set}



\section{9/1}


\end{document}


