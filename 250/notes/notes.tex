\documentclass[12pt]{article}

\usepackage{mathptmx}
\usepackage{mathpazo}

\usepackage{amsfonts}
\usepackage{amsmath}
\usepackage{amssymb}
\usepackage[version-1-compatibility]{siunitx}
\usepackage{fixltx2e}
\usepackage{multirow}
\usepackage{dsfont}
\newcommand{\inv}{^{-1}}

\topmargin -0.5in %topmargin=1in+\topmargin
\textheight 9in % default is letter % so 11-2=9in
\textwidth 6.5in % and 8.5-2in=6.5in
\oddsidemargin 0in %left margin = % 1in + oddsidemargin
\footskip 1cm % pagenumber to text

\setcounter{secnumdepth}{0}

\title{\normalsizMath 250A}
\date{\normalsize Fall 2015}

\begin{document}

\noindent
\textbf{Math 250A, Fall 2015}

\section{8/27}

\noindent
A group G acts on a set S:

$G \times S \to S$

$(g, s) \mapsto g \cdot s$

$e \cdot s = s$

$(g g') \cdot s = g \cdot (g' \cdot s)$

\noindent
Alternatively,

$\phi: G \to Perm(S)$ is a homomorphism

$(\phi(g))(s) = g \cdot s$

\noindent
Examples

trivial action: $(\forall g)$ $g \mapsto e_{Perm(S)}$

G acting on self by left/right translation, conjugation

G acting on the set of subgroups of G by conjugation: $g \cdot H = gHg^{-1} = \{ghg^{-1} | h \in H\}$

normal subgroup $N \trianglelefteq G$: all $g \in G$ fix $N$ under conjugation

V vector space over a field K, GL(V) acts on V by $L \cdot v = L(v)$\\

\noindent
The orbit of s, $O(s) := \{g \cdot s | g \in G\}$

constitutes an equivalence relation on S

\noindent
The stabilizer (isotropy group) of $s \in S$, $G_s := \{g \in G |g \cdot s = s\}$

$G_s$ is closed under inverses: $g \in G_s \to g \cdot s = s \to g^{-1}gs = g^{-1}s \to s = g^{-1}s$

\noindent
There exists a natural bijection $\alpha: G/G_s \to O(s)$, $gG_s \mapsto g \cdot s$

well-defined: $g_1G_s = g_2G_s \to \exists g \in G_s, g_1 = g_2g$, $\alpha(g_1G_s) = g_1s = g_2 g s = g_2 s = \alpha(g_2G_s)$

injective: $\alpha(g_1G_s) = g_1 \cdot s = g_2 \cdot s = \alpha(g_2G_s) \to g_2^{-1}g_1 \cdot s = s$, $g_2^{-1}g_1 \in G_s$, so $g_1G_s = g_2G_s$\\

\noindent
Action under conjugation:

the conjugacy classes of a set are the orbits of the action

$O(g) = \{g\} \leftrightarrow g \in Z(G)$ the center of the group

in a permutation group, $\sigma (a_1, a_2, a_3, ... a_k) \sigma^{-1} = (\sigma a_1, \sigma a_2, \sigma a_3, ... \sigma a_k)$

\section{9/1}

\noindent
Let $\Sigma$ be a set of representative elements of the orbits of S.

The index of a subgroup H is $(G : H) = \#(G/H)$

For finite G, $(G:H) = \frac{\#G}{\#H}$ ($g \not \in H$, $\exists$ natural bijection $H \to gH$ (check text))

$\#S = \sum_{s \in \Sigma}\#O(s) = \sum_s(G:G_s)$

defines a 'mass formula' $\#S = (\sum_s\frac{1}{\#(G_s)})(\#G)$

\noindent
Let G act on a subgroup H by left translation.

$\#H_s = \#H$ and from the above $\#G = (G:H) \cdot \#H$. 

this is a statement of Lagrange's Theorem, $(G:H) = \frac{\#G}{\#H}$.

\noindent
The kernel of the action $K = \bigcap_{s \in S}G_s$, which is just the kernel of $G \xrightarrow{\phi} Perm(S)$.

\noindent
We can relate the stabilizers of points in the same orbit.

Let $s' = gs$.

Assume $x \in G_s$.  

Since $x \in G_s$, $(gxg^{-1})gs = g(xs) = gs$.

Hence $gxg^{-1} \in G_{gs}$, so $gG_sg^{-1} \subset G_{gs}$.

Apply this relation with $g \to g^{-1}$ and $s \to gs$:

Assume $x \in G_{gs}$.

Then $(g\inv xg)(s) = (g\inv)(xgs) = (g\inv gs) = s$.

So $g\inv G_{gs}g \subset G_s \to G_{gs} \subset gG_sg\inv$

\noindent
Thus, $gG_sg\inv = G_{gs} = G_{s'}$.

The stabilizer of $s'=gs$ is a conjugate of the stabilizer of $s$.\\


\noindent
p : prime

\noindent
p-group: a finite group G, $\#G = p^n, n \geq 1$\\

\noindent
``A p-group has a non-trivial center''

Recall: the center $Z(G) = Z = \{g \in G | gs = sg \forall s\in G\}$.

Since $gs = sg \to s = gsg^{-1}$, will be useful to consider action on self by conjugation.

\noindent
G a p-group, S a finite set.  Then $\#O(s) = \frac{\#G}{\#G_s} = \frac{p^n}{p^k}$.

\noindent
Two cases:

1) $\#O(s) = 1$, s is fixed by G, $s \in S^G$ (set of fixed points of S)

2) ($k < n$), thus $\#O(s)$ is divisible by p.

\noindent
$\#S =$ sum of \# of elements in the orbits $\equiv_{mod p} \#$ of orbits of size 1 $= \#(S^G)$.

Take S = G, with action $g: s \mapsto gsg^{-1}$.  Then $S^G = Z(G)$.

$\#Z(G) \equiv_{mod p} \#(S^G) \equiv_{mod p} \#S = \#G = p^n \equiv_{mod p} 0$.

Thus, the order of the center is divisible by p, and must be non-trivial.\\

\noindent
$H \leq G$ a finite group, $(G : H) = p$, the smallest prime dividing \#G $\to H \trianglelefteq G$

Let $S = G\slash H$; $\#(S) = (G : H) = p$, and let G act on S by left translation.

This induces $\varphi: G \to S_P$; recall $\#S_p = p!$

The stabillizer of H, $G_H = \{x \in G | xH = H\} = H$.

By inspection, we can see that $G_{gH} = gHg^{-1}$.

Let $K = \bigcap_{g \in G}gHg^{-1}$, the largest normal subgroup contained in H.

Note that $K = ker(\varphi)$ induced above; by the First Isomorphism Theorem $\varphi(G) \leq S_p$.

$(G : K) = \#(G\slash K) = \#(\varphi(G))$, which divides $\#(S_p) = p!$

Further, since $K \leq H \leq G$, $(G : K) = (G : H)(H : K)$.

Since $(G : K)$ divides $p!$ and $(G : H)$ divides p, $(H : K)$ divides $(p - 1)!$.

But p is the smallest prime dividing $\#G$, so $(H : K)$ = 1, $K = H$ and H is normal.\\

\noindent
A familiar embedding of a group into a larger group; ``Cauchy's Theorem''

$G \hookrightarrow Perm(G)$ by letting G act on itself by left-translation.

Its kernel $K = \{g \in G | gs=s \forall s\} = \{e\}$ (consider $s = e$), hence is an injection.

Since an injection, an embedding.

\noindent
Recall $S_n \subset$ group of $n \times n$ invertible matrices. $\sigma \mapsto M(\sigma)$ a permutation matrix.

Need to be careful in the construction to ensure $M(\sigma\tau)=M(\sigma)M(\tau)$!

E.g. $\sigma = (1 3 2)$ does $M(\sigma)$ have 1 in the 1st column, 3rd row?

Or in the 1st row, 3rd column?  One of these yields $M(\sigma\tau) = M(\tau)M(\sigma)$.

\noindent
G finite of order n; V the vector space of functions $G \xrightarrow{f} \mathds{Z}$; note $V \cong \mathds{Z}^n$

Linear maps $V \to V$ correspond to $n \times n$ matrices over $\mathds{Z}$:  $GL(V) \approx GL(n, \mathds{Z})$.

Similarly, invertible linear maps correspond to $n \times n$ invertible matrices over $\mathds{Z}$.

We can embed G in $GL(n, \mathds{Z})$ by using a left action of G on $GL(n, \mathds{Z}) = \{\phi: V \to V\}$

Recall that $V = \{f : G \to \mathds{Z}\}$.

This left action takes the form $L_g \mapsto \phi$ where $\phi(f(x)) = f(xg)$

$L_{gg'} = L_{g'} \circ L_{g}$ as desired? Verify for yourself.

\textbf{Check this over.}

$L_{gg'}(\varphi (x)) = \varphi(xgg') = L_{g'}(\varphi (xg)) = L_{g'} \circ L_g (\varphi(x))$

$g \mapsto L_g$ is a homomorphism $G \to GL(V)$

Using $\mathds{F}_p$ instead of $\mathds{Z}$, get $G \hookrightarrow GL(n, \mathds{F}_p)$, an embedding into a finite group.

\section{9/3}

\subsection{Sylow Theorems}

Lagrange: If $H \leq G$ then $\#(H)|\#(G)$.

$A_4$ with $n = 6$ gives the counterexample to the converse.

\noindent
Salvaging the converse: the case where $n = p^k$, p prime.

\noindent
(Sylow I): If $|G| = p^k \cdot r$, $(p, r) = 1$

$\exists H \leq G$ such that $|H| = p^k$

Such an H is called a p-Sylow subgroup of G

Generally assuming $k \neq 0$

\noindent
Example : $\mathds{Z}_{12}$

has 2-sylow subgroup $\{0, 3, 6, 9\}$ and 3-sylow subgroup $\{0, 4, 8\}$

\noindent
Example: $D_6$ generated by r, s subject to $rs = sr^{-1}$, $r^6 = e$, $s^2 = e$, has order 12

$\#(D_6) = 12$ so has 3-sylow subgroup $\{1, r^2, r^4\}$

Also has 2-sylow subgroups $\{1, r^3, s, r^3s\}$, $\{1, r^3, rs, r^4s\}$, $\{1, r^3, r^2s, r^5s\}$

\noindent
Example: G = $GL_n(\mathds{F}_p)$, $n \times n$ linear transformations in $\mathds{F}_p$, equal to $Aut(\mathds{F}_p^n)$

\noindent
The order of $|G|$:  

Asserting linear independence in each vector of an $n \times n$ matrix

$|G| = (p^n - 1)(p^n-p)(p^n-p^2)\cdots(p^n-p^{n-1}) = p^{1 + 2 + 3 + \cdots + n - 1}\cdot r = p^{\frac{n^2 - n}{2}} \cdot r$

$(p, r) = 1$

Consider P the set of $n \times n$ upper triangular matrices with 1's on the diagonal.

Then $|P| = p^{1 + 2 + 3 + \cdots + n - 1} = p^{\frac{n^2 - n}{2}}$, and P is a p-Sylow subgroup.

\noindent
Theorem: (Sylow I) p-Sylow subgroups always exist.

\noindent
Proof Sketch:

Suppose $|H| = p^k \cdot r$, $(p, r) = 1$, $k > 0$

Show $\exists G$, $H \leq G$, where G has a p-Sylow subgroup

Show that if G has a p-Sylow subgroup and $H \leq G$, then H has a p-Sylow subgroup

\noindent
Proof:

By Cayley's theorem, if $|H| = n$, then $H \leq S_n$.

(H acts on itself by left translates.  This yields an embedding into $S_n$.)

Additionally $S_n \leq GL_n(\mathds{F}_p)$ mapping to permutation matrices.

Alternatively, consider $V \cong \mathds{F}_p^n$, the vector space of functions $\varphi : G \to \mathds{F}_p$.

Embed H into $GL(V)$ by this action: $g \in H \mapsto$ automorphism taking $\varphi(x)$ to $\varphi(xg)$.

(Recall end of previous lecture).

We know that $GL_n(\mathds{F}_p)$ has p-Sylow subgroups.  (from the lower triangular matrices)

Let $G = GL_n(\mathds{F}_p)$.

Let P be a p-Sylow subgroup of G.  Consider G acting on the set of cosets of P.

Now, $Stab(gP) = gPg^{-1}$. (guest lecturer notation for stabilizer)

Similarly, letting H act on $G \slash P$, $Stab(gP) = (gPg^{-1} \cap H)$

This intersection is a p-group.

Want to choose $g \in G$ such that $gPg^{-1} \cap H$ is a p-Sylow subgroup.

If $(H : (gPg^{-1} \cap H))$ is coprime to p, then $gPg^{-1} \cap H$ is a p-Sylow subgroup.

By Orbit-Stabilizer, $(H : (gPg^{-1} \cap H)) = O(gP)$.

Note this is an orbit of $G/P$ induced by the action of the group H.

Since P is a p-Sylow subgroup of G, $|G\slash P| \not\equiv_{mod p} 0$.

The sum of the orbits is $|G \slash P|$.

Hence there must be some orbit with size coprime to p.\\

\noindent
Corollary: All p-subgroups of H are contained in a conjugate of P.

Let $J \leq H$ be a p-subgroup.  Then $J \cap gPg\inv$ is a p-Sylow subgroup of $J$ for some $g \in G$.

So since $J$ is a p-group $J \cap gPg\inv = J$, i.e. $J \subset gPg\inv$.

(since a p-group can't contain a proper p-Sylow subgroup by definition)\\

\noindent
Corollary: (Sylow II) All p-Sylow groups are conjugate.

\noindent
Proof:

Let $H \leq G$ and $P \leq G$ be p-Sylow subgroups.

By the preceding corollary, $H \subset gPg\inv$ for some $g \in G$.

Since $|H| = |P| = |gPg^{-1}|$, $H \cap gPg\inv = H$.\\

\noindent
Corollary: Every p-subgroup of G is contained in a p-Sylow of G.

By the above, each is contained in a conjugate of P, said conjugate being a p-Sylow.\\

\noindent
The p-Sylow subgroups in G are all conjugate, so that:

If P is a p-Sylow of G then $G/N(P)$ is the set of p-Sylows in G.

Where $N(P)$ is the normalizer of P.

So there are $(G: N(P))$ p-Sylows in total.\\

\noindent
Lemma: If a finite p-group $\Gamma$ acts on a set X, then $\#(X) \equiv_{mod p} \#(X^{\Gamma})$

($X^{\Gamma}$ the fixed points of X under $\Gamma$).

\noindent
Proof:

$\#X = \sum_i \#Orb(x_i) = \sum_i \frac{|\Gamma|}{|Stab(x_i)|} \equiv_{mod p} \#X^{\Gamma}$

Each $\frac{|\Gamma|}{|Stab(x_i)|} \equiv_{mod p} 1$ if $x_i$ fixed, else $\frac{|\Gamma|}{|Stab(x_i)|} \equiv_{mod p} 0$.\\

\noindent
Let $Syl_p(G)$ describe the p-Sylow subgroups of G and $n_p$ denote its cardinality.

\noindent
Theorem: (Sylow III) If $|G|= p^k \cdot r$, $k > 0$ then $n_p \equiv_{mod p} 1$.  Further, $n_p | r$.

\noindent
Proof:

Let P act on $Syl_p(G)$ by conjugation.

By the lemma, $\#Syl_p(G) = n_p \equiv_{mod p} (Syl_p(G))^P$.

Suppose Q is fixed under the group action.  Then $pQp\inv = Q$ $\forall p \in P$.

Then $P \leq N(Q)$; similarly $Q \leq N(Q)$.

P, Q are p-Sylow subgroups of N(Q); therefore P, Q are conjugate in N(Q).

However, $Q \trianglelefteq N(Q)$ so that Q is equal to all its conjugates in N(Q), and $P = Q$.

Hence P is the only fixed Sylow-p subgroup so $(Syl_P(G))^P \equiv_{mod p} 1$.

G acts on $Syl_p(G)$ as only one orbit since all p-Sylows in G are conjugate.

$(G : P) = n_p$, $n_p = |G| = p^k \cdot r$, $n_p | p^k \cdot r$, but $n_p \nmid p$, so $n_p | r$.

\section{9/8}

\subsection{Review of Sylow Theorems}

Prove existence by showing existence in a larger known subgroup.

And then that contained subgroups must have their own Sylow p-subgroups.

$O(s) = S = \{$p-Sylows$\}$

$O(s) = G\slash G_s = G \slash N(P)$

The number of p-Sylows is notated $n_p = (G : N(P))$

\noindent
$P, Q$ p-Sylows and $P \subset N(Q)$ then $P = Q$

reason: $PQ \leq G$ a subgroup of G

$HK$ not necessarily a group, but will be if one normalizes the other

ie $H \subset N(K)$

\noindent
Theorem $n_p \equiv_{mod p} 1$

Consider the action of $P$ on $S$ by conjugation

Take $x \in P$ and $x: Q \mapsto xQx\inv$

The number of fixed points is 1, since $P$ fixes only itself

\noindent
A simple group has

more than one element

no non-trivial proper normal subgroups

(kind of like a prime number)

\noindent
G finite abelian

G simple $\leftrightarrow$ G cyclic of prime order (simple easy exercise)

\subsection{continuing...}

\noindent
non-sporadic finite simple groups

$A_n (n \leq 5)$

recall the alternating groups $A_n$ are the even permutations on $\{1, \cdots, n\}$

Lie groups over finite fields, e.g. $\{\pm \begin{pmatrix} 1 & 0 \\ 0 & 1 \\ \end{pmatrix}\} \subset SL(2, \mathds{Z}|p\mathds{Z})$

P = projective; $PSL(2, \mathds{Z}|p\mathds{Z}) = SL(2, \mathds{Z}|p\mathds{Z})$

\noindent
Simple groups of order $\leq 60$.

(a) There are no non-abelian simple groups of order $< 60$

(b) If G is simple of order 60, then $G \cong A_5$.

($\#A_n = \frac{n!}{2}$)

\noindent
G simple of order 60.

$H < G$ simple (finite), H proper, $(G: H) = n \geq 2$

G acts on $G\slash H$ by left translation.

The action is transitive (for each pair $xH, yH$, $\exists$ permutation taking one to the other)

Therefore, this action is non-trivial.

$\pi : G \to Perm(G\slash H) = S_n$

$ker(\pi) \neq G$ and is a normal subgroup $\to$ the kernel is trivial.

$\pi: G \hookrightarrow S_n$ and in fact $\pi: G \hookrightarrow A_n$ (if $\#G > 2$)

\noindent
Why? because $G \cap A_n \trianglelefteq G$

If $G \subset S_n$.

Then $G \to S_n/A_n = \{\pm 1\}$ by the sign map, kernel is $G \cap A_n$.

Recall $sgn: S_n \to \{\pm 1\}$ $sgn(\sigma) = (-1)^t$ given t, num of transpositions

$G/(G \cap A_n) \hookrightarrow S_n/A_n = \{\pm 1\}$

$(G : G \cap A_n) = 1$ or $2$.

If G is simple then this cannot be 2 (would be normal subgroup), so =1.

And $G \hookrightarrow A_n$ for that $A_n$.

\noindent
G simple, order 60.

H a proper subgroup of G, index n.  (consider small values of n)

If $n = 2$ then H is normal in G, a contradiction.

(smallest prime dividing the order of a group)

If $n = 3$ or $n = 4$: $G \hookrightarrow A_3, A_4$ but their orders are too small (3, 12)

If $n = 5$: $G \hookrightarrow A_5$ and they are equal in cardinality $\to$ done.

Remaining case: $n = 15$.

What is $n_5$, the number of 5-Sylow subgroups.

$n_5 | 60\slash 5 = 12$, $n_5 = (G : N(P))$ $n_5$ divides the index

Also, $n_5 \equiv_{mod 5} 1$.

Thus $n_5 = 1$ or $n_5 = 6$.

If $n_5 = 1$ then only one 5-Sylow subgroup of G, must be normal.

This is impossible since G is simple.

Then $n_5 = 6$: tells you there are lots of elements of order 5 in G.

There is no overlap (excepting at the identity) between 5-Sylows.

Hence the number of elements of order 5 is $6 \cdot 4 = 24$

Elements of order 5 in $A_5$ are 5-cycles (a b c d e).

Need to take all strings of length 5: 120, and divide out by rotations 5.

Thus we get $120 \slash 5 = 24$ (check).

Consider $n_2$ the number of 2-Sylow subgroups.

Then $n_2$ divides $60/4 = 15$, and $n_2 \neq 1$ because of simplicity.

Also, $n_2 = (G : N(P_2))$, and this can't be 3 since G has no subgroup of index 3.

If $n_2 = 5$ then $N(P_2)$ is the desired index-5 subgroup $\to$ done.

From divisibility $n_2 = 1, 3, 5, 15$.

Elminate 1 by simplicity, 3 since the index is too small, 5 works, consider 15.

\noindent
Considering the situation where there are 15 2-Sylow subgroups (of order 4).

These are groups like the Klein 4-group (no elements of order 4).

There are 2 2-Sylow subgroups P and Q where $P \cap Q$ has order 2.

Prove by counting.

Taking intersection, must be proper else they would be the same.

Hence $P \cap Q$ has order 1 or 2.

If there is utterly no overlap, there are $15 \cdot 3 + 1 = 46$ elt's of 2-Sylows.

And these do not have order 5.  But there are 24 elements of order 5.  Too many.

\noindent
Now we know that some of these 2-Sylow subgroups have non-trivial overlap.

Consider $N(P \cap Q)$ for some such intersection, will be a subgroup of G.

Cannot be all of G, G is simple. (would make $P \cap Q$ normal)

$N(P \cap Q)$ contains P and Q since both are abelian.

Each are normal subgroups of $N(P \cap Q)$, so its order is divisible by 4.

Hence could have order 12, 20, or 60 (divisible by 4, divides 60).

Its index cannot be 1 (G is simple) cannot be 3 ($A_n$ too small), = 5.

QED (\textbf{revisit why}).

\noindent
Jordan-H\"{o}lder theorem

Website reference.

\noindent
G finite non-trivial.  Is G simple?  $\{e\} \subset G$, $G/\{e\}$ simple.

Not simple $G \supset G_1 \supset (e)$, $G_1 \trianglelefteq G$, $G_1$, $G/G_1$ smaller than G.

Keep going until 'end', using principle of string induction.c

\noindent
Proposition: $\exists G = G_0 \supset G_1 \supset G_2 \supset \cdots \supset G_n$, $G_{i + 1} \trianglelefteq G_i$, $G_i/G_{i+1}$ simple.

A \textit{normal tower} or \textit{composition series}, the simple quotients are the \textit{constituents}.

Obtain a successive extension of simple groups.

\noindent
Main point.

$N = p_1\cdots p_n$

$\{p_1, p_2, \cdots, p_n\}$ a set where order doesn't count but multiplicity does.

Gauss's theorem: (FTA) each prime decomposition of N yields the same set.

\noindent
Similarly, given $G$ and $G_i/G_{i+1} = Q_i$ and $\{Q_0,\cdots , Q_{n-1}\}$.

Order not mattering, multiplicity matters, up to isomorphism.

Theorem: Each composition yields the same multiset.

Theorem of ``Camille Jordan and some guy named H\"{o}lder.''

\section{9/10}

\subsection{Jordan-H\"{o}lder Theorem.}

$G = G_0 \supset G_1 \supset G_2 \supset \cdots \supset G_n$

$G_{i + 1} \trianglelefteq G_i$, $G_i/G_{i + 1} = Q_i$ simple.

\noindent
Statement of the theorem:

The ``set'' (multiplicity matters) $\{Q_0, \cdots, Q_{n - 1}\}$ is independent of the filtration.

Order doesn't count, $Q_i$ up to isomorphism.

\noindent
Proof strategy: by induction.

If G has a filtration with n quotients, then all filtrations have n quotients.

And all filters have the same set of quotients.

\noindent
Question, can two different groups have the same reduction?

Answer: yes.  $S_3 \supset A_3 \supset \{e\}.$ Quotients $\mathds{Z}/2\mathds{Z}$ and $\mathds{Z}/3\mathds{Z}$.

Also $\mathds{Z}/6\mathds{Z} \supset 3\mathds{Z}/6\mathds{Z} \supset 6\mathds{Z}/6\mathds{Z}$, same quotients but radically different structure.

``Knowing the building blocks does not confer knowledge of the building''.

\noindent
Demonstrating the existence of such a filtration for a group $G \neq \{e\}$.

Similar to the proof of prime decompositions.

If it is simple, then the filtration is $G \supset \{e\}$, done.

If G is not simple, $G \supset N \supset \{e\}$, and $G/N, N$ smaller than G.

Strong induction.  $\overline{G} = G/N$, then $\overline{G} \supset \overline{G_1} \supset \cdots$ and similarly for $N \supset H_1 \supset \cdots$

Note there is a correspondence b/t subgroups of G con't N and subgroups of $G/N$

$G \supset L \supset N$, $L/N \subset G/N$ and $\pi: G \to G/N$, $\pi^{-1}(K) \subset G$ and $K \subset G/N$.

\noindent
Base case $n = 1$, $G \supset \{e\}$, $G/\{e\}$ simple and G simple.

\noindent
Supposing $G \supset G_1 \supset \cdots \supset G_n \supset \{e\} = G_{n + 1}$ and $G \supset G_1' \supset \cdots \supset G_m' \supset \{e\} = G_{m + 1}'$.

? $m = n$, $\{G_i / G_{i + 1}\} = \{G_j'/G_{j + 1}'\}$ ... If $G_1' = G_1$, then done by induction.

Assume $G_1, G_1'$ are distinct.  Then $G_1 \cap G_1'$ is smaller than $G_1$ or $G_1'$.

Also, $G_1G_1'$ is a subgroup since its factors are normal by hypothesis.

Indeed, it is also a normal subgroup since $G_1$ and $G_1'$ are invariant under conjugation.

Additionally, $G_1G_1'$ is of size larger than $G_1$ and $G_1'$.  Thus it must be equal to G.

Can map $G_1'/(G_1 \cap G_1') \to G_1G_1'/G_1$.  Kernel is exactly $G_1 \cap G_1'$, hence injection.

This defines $G_1'/(G_1 \cap G_1') \hookrightarrow G/G_1$.  Symmetrically, $G_1/(G_1 \cap G_1') = G/G_1'$.

Have $G_1 \supset \cdots \supset G_n \supset \{e\} = G_{n + 1}$.

Take $G_1 \supset G_1 \cap G_1' = H \supset H_1 \supset H_2 \supset \cdots \supset H_k \supset \{e\}$, a Jordan-H\"{o}lder filtration of $G_1$.

Obtained by induction.

Note $G_1/H = G/G_1'$ is the first quotient of this filtration.

By induction, these two filtrations have the same length.

The constituents of $G_1$ are the constituents of H, with $G_1/H = G/G_1'$ appended.

Constituents: $G/G_1$ + constituents of $G_1$ = $G/G_1$ + $G/G_1'$ + constituents of H.

Have $G \supset G_1' \supset H \supset H_1 \supset \cdots \supset H_k = \{e\}$, same length as $G_1' \supset G_2' \supset \cdots \supset G_m' = \{e\}$.

Have related two different filtrations that have are unrelated, by a common filtration,

which depends on the intersection of these two filtrations.

\subsection{Free Groups}

S a set, define the free abelian group on $S$, $\mathds{Z}^S = \mathds{Z}\langle S\rangle = \{\sum_{s \in S} n_s \cdot s | n_s \in \mathds{Z}\}$.

Where all but finitely many of the $n_s$ are 0.

$S = \{1, \cdots, n\}$, $\mathds{Z}^S = \mathds{Z}^n = \{(c_1, \cdots, c_n) | c_i \in \mathds{Z}\}$

$$\sum_{i = 0}^\infty n_i x^i = \sum_{i = 0}^\infty n_i \cdot i \in \mathds{Z}\langle S \rangle$$

where $n_i = 0$ for $i >> 0$.

\noindent
``To map $\mathds{Z}\langle X\rangle$ to A in the world of abelian groups is to map S to A in the world of sets.''

$S \to \mathds{Z}\langle S \rangle$ a set map, $s \in S \mapsto 1 \cdot s$.

Given $f: \mathds{Z} \langle S \rangle A$ homomorphism.

And in fact, $F: Hom(\mathds{Z} \langle X\rangle, A) \to Maps(S, A)$, F is a bijection.

\noindent
These elements of the free abelian group are ``formal sums''.

That is, an $f: S \to \mathds{Z}$.

\noindent
Let $f: \mathds{Z}\langle S\rangle \to A$, $f(\sum n_ss) = \sum_{s \in S}n_sf(s)$

\noindent
An abelian group A is free of finite rank if $A \cong \mathds{Z}^n$ for some $n \geq 0$ ($\mathds{Z} = \mathds{Z}\langle \emptyset \rangle = 0$).

Define $rank(A) = n$.  If $\mathds{Z}^m \cong A \cong \mathds{Z}^n$ then n = m.

Why? Take positive integer $> 1$, e.g. 2.  Then $\mathds{Z}^n/2\mathds{Z}^n \cong \mathds{Z}^m/2\mathds{Z}^m$.

LHS has $2^n$ elts and RHS has $2^m$ elts so $n = m$. \\

\noindent
A subgroup of a free abelian group of rank n is a free abelian group of rank $\leq n$.

\noindent
Proof: by induction on n.

$n = 0$: $A = (0) = B$.

$n = 1$: $A = \mathds{Z} \supset B$.  What are the subgroups of $\mathds{Z}$? $(0), (t) = t\mathds{Z}, t \geq 1$.

Proof by division algorithm: $\mathds{Z} \supset B \neq 0$, $t = $smallest positive integer in B.

Division algorithm ensures that all elements are multiples of t.

$B \subset \mathds{Z}^n \xrightarrow{\pi} \mathds{Z}$.

$\pi: (c_1, \cdots, c_n) \mapsto c_n \in \mathds{Z}$.

\noindent
Cases:

(1) $\pi(B) = (0), B \subset \mathds{Z}^{n-1}$, free of rank $\leq n - 1$

(2) $\pi(B) = t\mathds{Z}, t \geq 1$

$B \xrightarrow{\pi|_B} t\mathds{Z} \xrightarrow{surj.} 0$

$ker(\pi)|_B = C$ free of rank $\leq n - 1$.

Choose $b \in B$ such that $\pi(b) = t$.

$C \subset \mathds{Z}^{n-1}: C = ker(\pi)|_B$, free of rank $\leq n - 1$.

$C = B \cap \mathds{Z}^{n - 1}$

$C \subset B$, $\mathds{Z} \cdot b \subset B$

\textbf{Missing (pf in Lang)}

\noindent
Simple linear algebra.

$a_1, \cdots, a_n \in A$ corresponds to a homomorphism $\mathds{Z}^n \to A$, $(c_1, \cdots, c_n) \mapsto \sum_{i = 1}^n c_ia_i$.

These are linearly independent if f is 1-to-1, and these span/generate A if $f$ is onto.

A is finitely generated if A is spanned by $a_1, \cdots, a_n$ for some $n \geq 0$, $a_i \in A$

A is finitely generated iff A is a quotient of $\mathds{Z}^n$ for some n.

\noindent
Corollary: a subgroup of a finitely generated abelian group is again finitely generated.

$\mathds{Z}^n \xrightarrow{f} A$ finitely generated, have $B \subset A$, $f^{-1}(B) \leq \mathds{Z}^n$, and $f^{-1}(B) \cong \mathds{Z}^k$, $k \leq n$.\\

\noindent
A finitely generated, torsion-free.

I.e. given $a \in A$ and $n \cdot a = 0$, $n \geq 1$, then $a = 0$.

Statement: A is free and of finite rank.

\noindent
Proof: Take a finite set of generators S in which take T lin indep and large as possible.

take $T = a_1, \cdots, a_k$ and $S = a_1, \cdots, a_k, \cdots, a_m$

$\sum_{1}^{k + 1} c_ka_k = 0$, $c_{k+1} \neq 0$

$B = span\{a_1, \cdots, a_k\} \cong \mathds{Z}^k$.

$a_{k + 1}, \cdots,a_m$: some multiple lies on B.

$N \geq 1$; $N \cdot A \subset B$.

Th: NA free, $N: A \to NA$ A torsion free.

Multiplication on A by a positive integer is injective.

A is isomorphic to NA by the mutliplication by n, since NA is free, A is free.

\section{9/15}

Abelian group, finitely generated.

\noindent
Last week:

free group has to do with some correspondence to a $\mathds{Z}^n$

subgroups of free finitely generated abelian groups are free and finitely generated

subgroups of finitely generated abelian groups are finitely generated

finitely generated, torsion free abelian group is a free abelian group

recall torsion free: for all $n \geq 1$, mult by n, $n \cdot A$ is injective

opposite A torsion: for all $a \in A$, $\exists n \geq 1$ such that $n \times a = 0$

\noindent
Example of a torsion abelian group: $\mathds{Q}/\mathds{Z}$

element $p/q \mod{\mathds{Z}}, q \geq 1, p \in \mathds{Z}$; $q \times \frac{p}{q} = 0$ in $\mathds{Q}/\mathds{Z}$

\noindent
finitely generated abelian groups up to isomorphism

A is a direct sum of a free part $\mathds{Z}^r$ and a torsion part (a direct sum of cyclic groups)

\noindent
Direct product of sets $A_i$ indexed by $S$: $$\bigoplus_{i \in S}A_i = \{f: S \to \cup_{i \in S}A_i : f(i) \in A_i\}$$

where for all but finitely many i, $f(i) = 0$

this is equivalent to the direct product when S is finite

\noindent
\textbf{Image 1}: a map from a $\bigoplus_{i \in S}A_i$ to B is determined by the mappings from the $A_i$

The direct sum is a coproduct.

\noindent
\textbf{Image 2}: a map into a $\prod_{i \in S}A_i$ is determined by the mappings into the $A_i$

The direct product is a product (in the categorical sense).

\noindent
$S$ countably infinite, $A_i = \mathds{Z}/2\mathds{Z}$

$\bigoplus_{i \in s}A_i$ is countable, but $\prod_{i \in S}A_i$ is not

\noindent
Categories: products, coproducts, morphisms

$Mor(?, B) = \prod Mor(A_i, B)$ $? =$ co-product

The coproduct of sets is disjoint union.

\noindent
Abelian group A and subgroups X and Y

we have inclusions from each into A

$X \times Y = X \oplus Y \xrightarrow{h} A$, $(x, y) \mapsto x + y$

$h$ is injective if every $a \in A$ is of the form $x + y$

$h$ is one-to-one $\leftrightarrow$ you can't write $x + y = x' + y'$ unless $x = x'$, $y = y'$

If true, say A is the direct sum of its submodules X and Y.

\noindent
Suppose A, $X \subset A$, $A/X$ is free (f.g. free): then X has a complement Y in A, $A \cong X \oplus A/X$

$A \xrightarrow{\pi} A/X$

$Y \subset A$, $\pi|_Y$ is an isom $Y \to A/X$.

$\pi|_Y$ inj $\leftrightarrow Y \cap X = (0).$

$\pi|_Y$ surjective: given $a + X \in A/X$ we can find $y \in Y$ s.t. $y + X = a + X$

$x = y \cdot a \in X$

$a = y \cdot x, x\in X, y\in Y$

$A/X$ free, say $\cong \mathds{Z}^r$

To map $A/X$ to $A$ is to choose images in $A$ of the generators of $A/X$ corresponding to the unit vectors of $\mathds{Z}^r$.

There is a unique homomorphism $s: A/X \to A$ so that $s(q_i) = a_i$ for $i = 1, \cdots, r$

$(\pi \cdot s)(q_i) = \pi(a_i) = q_i$

$\pi \circ s = id_{A/X}$

$Y =$ image of $S \subset A$.

$\pi|_Y$ surjective.  $\pi(s(q)) = q$ for all $q \in A/X$

$\pi|_Y$ is 1-1.  $/pi(s(q_0)) = 0$ but $s(q_0) = q_0$ so equals 0.

\noindent
A a finitely generated abelian group

$X = A_{tors} = \{a \in A | na = 0$ for some $n \geq 1\}$.

X f.g., tors $\to$ X finite abelian group.

$A/X$ torsion free, f.g. $\to$ $A$ free $\approx \mathds{Z}^r$

\noindent
$A \approx \mathds{Z}^r \oplus A_{tors}$.  $A_{tors} = ???$

it is a finite abelian group, let $B = A_{tors}$

p prime, $B_p = \{b \in B | p^t \cdot b = 0$ for some $t \geq 0$\}.

$B_P \subset B$.

$\bigoplus_p B_p \xrightarrow{\iota} B$

Proposition: $\iota$ is an isomorphism. (formal proof in Lang's book)

\noindent
Proof essence:

suppose $60 \cdot b = 0$, $60 = 4 \cdot 3 \cdot 5 = 12 \cdot 5$

$(12, 5) = 1$

$1 = r5 + s12 = 25 - 24$

$b = r \cdot 5 \cdot b + s \cdot 12 \cdot b$

$12x = 0$, $5y = 0$

Every element can be written as a sum of terms killed by a power of a prime

\noindent
$A = \mathds{Z}^r \oplus (\bigoplus_p B_p)$

\noindent
$\mathds{Z}^n \approx F \xrightarrow{\varphi} A$ $A$ finitely generated (by n elements)

$Ker(\varphi) = X \subset F$.

? understand A ! understand X inside F.

\noindent
Elementary division theorem

There exists a basis of $F \approx \mathds{Z}^n$ s.t. ... $X = \bigoplus_{i \leq r}0 \oplus a_1\mathds{Z} \oplus a_2\mathds{Z} \oplus \cdots \oplus a_{n - r}\mathds{Z}$, $a_i \geq 1$

$X \subset \mathds{Z}^n$

$a_1 | a_2 | a_3 | \cdots | a_{n - r}$, increasing multiplicatively

$A = F/X = \mathds{Z}^r \oplus \mathds{Z}/a_1\mathds{Z} \oplus \mathds{Z}/a_2\mathds{Z} \oplus \cdots$, $a_i | a_{i + 1}$

A a finite abelian group $\to$ A is a direct sum of cyclic groups

\noindent
p prime, $\#A = p^4 = a_1a_2a_3\cdots$

A is direct sum of cyclic groups of p-power order.

$A \approx \mathds{Z}|p^i \oplus \mathds{Z}|p^j \oplus \mathds{Z}|p^k \oplus \mathds{Z}|p^l$ at most

$i \leq j \leq k \leq l$, $i + j + k + l = 4$, $i, j, k, l, \geq 1$

\end{document}
