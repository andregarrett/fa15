\documentclass[12pt]{article}

\usepackage{mathptmx}
\usepackage{mathpazo}

\usepackage{amsfonts}
\usepackage{amsmath}
\usepackage{amssymb}
\usepackage[version-1-compatibility]{siunitx}
\usepackage{fixltx2e}
\usepackage{multirow}
\usepackage{dsfont}
\newcommand{\inv}{^{-1}}

\topmargin -0.5in %topmargin=1in+\topmargin
\textheight 9in % default is letter % so 11-2=9in
\textwidth 6.5in % and 8.5-2in=6.5in
\oddsidemargin 0in %left margin = % 1in + oddsidemargin
\footskip 1cm % pagenumber to text

\setcounter{secnumdepth}{0}

\title{Math 250A}
\date{\normalsize Fall 2015}


\begin{document}
\maketitle

\section{8/27}

\subsection{Group Action}
\noindent
A group G acts on a set S:

$G \times S \to S$

$(g, s) \mapsto g \cdot s$

$e \cdot s = s$

$(g g') \cdot s = g \cdot (g' \cdot s)$

\noindent
Alternatively,

$\phi: G \to Perm(S)$

$\phi$ is a homomorphism (gives the corresponding properties)

$(\phi(g))(s) = g \cdot s$

\subsection{Examples of Group Actions}

\noindent
The trivial action:

$G \to Perm(S)$ where $g \mapsto e_{Perm(S)}$

\noindent
G acting on self by left/right translation, conjugation

\noindent
G acting on the set of subgroups of G by conjugation:

$g \cdot H = gHg^{-1} = \{ghg^{-1} | h \in H\}$

\noindent
Normal subgroup $N \trianglelefteq G$

G acting on N, $g \cdot n := gng^{-1} \in N$

\noindent
$G = S_3$ where S is the set of subgroups of G of order 2.

S = \{\{1, (1 2)\}, \{1, (1 3)\}, \{1, (2 3)\}\}

\noindent recall
$\sigma (a_1, a_2, a_3, ... a_k) \sigma^{-1} = (\sigma a_1, \sigma a_2, \sigma a_3, ... \sigma a_k)$

\noindent
V vector space over a field K

G = GL(V) = group of invertible linear maps $V \to V$

e.g. if $V = K^n$ then G = GL(n, K)

G acts on V (rather simply) by $L \cdot v = L(v)$

\subsection{Orbits and Stabilizers}

\noindent
Given G acting on S by $G \times S \to S$ there is an obvious relation on S:

s, s': $s \thicksim s' \leftrightarrow \exists g \in G, s' = gs $

the orbit of s is just the equivalence class of s under this relation

i.e., $G \cdot s = \{g \cdot s | g \in G\}$

\noindent
The conjugacy classes of s are the orbits of S under the group action of G by conjugation

the orbit of s, $O(s) = \{s\} \leftrightarrow s = gsg^{-1} \forall g$

$\leftrightarrow (\forall g)gs = sg$

$\leftrightarrow s \in Z(G)$ the center of the group

\noindent
Example, for $G = S_3$

the orbit of 1 is \{1\}

the orbit of (1 2) = \{(1 2), (1 3), (2 3)\}

the orbit of (1 2 3) = \{(1 2 3), (1 3 2)\}

\noindent
Stabilizer (isotropy group) of a given element $s \in S := G_s$

$G_s = \{g \in G |g \cdot s = s\}$

stabilizer is closed under inverses: $g \in G_s \to g \cdot s = s \to g^{-1}gs = g^{-1}s \to s = g^{-1}s$

\subsection{large stabilizer$\leftrightarrow$ small orbit}

\noindent
there exists a natural bijection $\alpha: G/G_s \to O(s)$ defined $gG_s \mapsto g \cdot s$


\noindent
well-definition:

if $g_1G_s = g_2G_s$
then $\exists g \in G_s, g_1 = g_2g$
and $\alpha(g_1G_s) = g_1 \cdot s = g_2 g s = g_2 s = \alpha(g_2G_s)$

\noindent
injectivity:

if $\alpha(g_1G_s) = g_1 \cdot s = g_2 \cdot s = \alpha(g_2G_s)$
then $g_2^{-1}g_1 \cdot s = s$, $g_2^{-1}g_1 \in G_s$ and $g_1G_s = g_2G_s$

\section{Lang 1.1-1.4}

\subsection{1.1: Monoids}

\noindent
A \textit{monoid} is a set with associative binary operation and unit element.

Abelian $\leftrightarrow$ commutative

\noindent
A \textit{submonoid} is a subset of a monoid with identity and closure under the operation

Such a submonoid is, itself, a monoid

\subsection{1.2: Groups}

\noindent
A \textit{group} is a monoid with inverses for each element

\noindent
The \textit{permutation group} of S is the set of all bijections $S \to S$ (with composition as product)

\noindent
A direct product of groups has product defined componentwise

\noindent
A \textit{subgroup} of a group is a subset closed under composition and inverse

\noindent
$S \subset G$ \textit{generates} G if $\forall g \in G, g = \prod s_i$, where $s_i \in S$ or $s_i^{-1} \in S$

$G = \langle S \rangle$

\noindent
The \textit{group of symmetries of the square} is a non-abelian group of order 8

generated by $\sigma$, $\tau$ such that $\sigma^4 = \tau^2 = e$ and $\tau \sigma \tau^{-1} = \sigma^3$

\noindent
The \textit{quaternions} are a non-abelian group of order 8

generated by i, j where defining $k = ij$, $m = i^2$

$i^4 = j^4 = k^4 = e$, $i^2 = j^2 = k^2 = m$, and $ij = mji$

\noindent
A \textit{monoid-homomorphism} $f: G \to G'$ satisfies $f(xy) = f(x)f(y)$ and $f(e_G) = e_{G'}$

If G and G' are groups, f is a group homomorphism ($f(x^{-1}) = f(x)^{-1}$ is implied)

\noindent
An \textit{isomorphism} is a bijective homomorphism.

An \textit{automorphism} or \textit{endomorphism} of G is an isomorphism $\varphi: G \to G$

\noindent
The group \textit{Aut(G)} is the set of all automorphisms of G

\noindent
The \textit{kernel} of a homomorphism $f: G \to G'$ is $\{g \in G : f(g) = e_{G'}\}$

the kernel and the image f are subgroups of their respective groups

\noindent
An \textit{embedding} is a homomorphism $f: G \to G'$ where $G \cong Im(f)$.

\noindent
Fact: A homomorphism with trivial kernel is injective.

Forward is obvious.

Supposing trivial kernel: $f(x) = f(y) \leftrightarrow f(x)[f(y)]^{-1} = e \leftrightarrow f(xy^{-1}) = e \leftrightarrow xy^{-1} = e$

\noindent
For G a group, and $H, K \leq G$ such that $H \cap K = e$, $HK = G$, and $xy=yx$ $\forall x \in H$ $\forall y \in K$

The map $H \times K \to G$ defined $(x, y) \mapsto xy$ is an isomorphism

This generalizes to finitely many such subgroups by induction

\noindent
A \textit{left coset} of H in G ($H \leq G$) is $aH = \{ax : x \in H\} \leq G$

$x \mapsto ax$ gives bijection between cosets of H, are all of equal cardinality

The \textit{index} of H in G (G : H) is the number of cosets of H in G (right or left)

The \textit{order} of G is the index (G : 1) of its trivial subgroup

\noindent
For any subgroup H of G, G is the disjoint union of its cosets in H

\noindent
For $H \leq G$, $(G : H)(H : 1) = (G : 1)$, holding if at least two are finite

If (G : 1) is finite, the order of H divides the order of G.

\noindent
Given:

$H, K \leq G$, $K \subset H$

$\{x_i\}$ a set of coset representatives of K in H

$\{y_i\}$ a set of coset representatives of H in G

\noindent
Then:

$\{y_jx_i\}$ is a set of coset representatives of K in G.

\noindent
Therefore the above can be generalized to $(G : K) = (G : H)(H : K)$

\noindent
Conclusion: groups of prime order are cyclic.

\noindent
$J_n = \{1, ... , n\}$, $S_n = Perm(J_n)$

$\tau \in s_n$ is a \textit{transposition} if $\exists r \neq s \in J_n$, $\tau(r) = s$, $\tau(s) = r$, $\tau(k) = k$ $\forall k \neq r, s$

The set of transpositions generate $S_n$

Consider $H \leq S_n$ those which leave $n$ fixed. Then $H \cong S_{n-1}$.

Now if $\sigma_i \in S_n$ for $1 \leq i \leq n$ are defined with $\sigma_i(n) = i$, $\{\sigma_i\}$ are coset reps for H

Hence $(S_n : 1) = n(H : 1) = n!$. 

\subsection{1.3: Normal subgroups}

\noindent
For H the kernel of $f: G \to G'$ a group-homomorphism, $xH = f^{-1}(f(x)) = Hx$

Such a relation is equivalent to e.g. $xH \subset Hx$ and $H \subset xHx^{-1}$

A subgroup $H \trianglelefteq G$ (satisfying $xHx^{-1} = H$ $\forall x \in G$) is termed \textit{normal}

H is normal $\leftrightarrow$ H is the kernel of some homomorphism

\noindent
The \textit{factor group} of G by $H \trianglelefteq G$ is the group of cosets, denoted $G\slash H$

$f: G \to G\slash H$ defined $x \mapsto xH$ is the canonical map for H

\noindent
The \textit{normalizer} $N_S$ of $S \subset G$ is $\{x \in G | xSx^{-1} = S \}$

The normalizer of H is the largest subgroup of G in which H is normal

\noindent
The \textit{centralizer} $Z_S$ of S is $\{x \in G|xyx^{-1} = y$ $\forall y \in S\}$

The centralizer of G is called its \textit{center}; its elements commute with all others in G

\noindent
The \textit{special linear group} is the kernel of the determinant (a homomorphism)

\noindent
G is the \textit{semidirect product} of N and H if $G=NH$ and $H\cap N = \{e\}$

\noindent
An \textit{exact} sequence $G' \xrightarrow{f} G \xrightarrow{g} G''$ satisfies $Im(f) = Ker(g)$.

Can extend to larger sequences as long as each triple satisfies the above

\noindent
Some canonical homomorphisms, given $f: G \to G'$

$H = ker(f)$ $\to$ $\exists !f': G\slash H \to G'$ injective $\to$ $\exists\lambda: G\slash H \to Im(f)$ an isomorphism

$H \leq G$, N the minimal $N \trianglelefteq G$ s.t. $H \leq N$, $H \subset ker(f)$, then $N \subset ker(f)$, $\exists ! f': G\slash N \to G'$

$H, K \trianglelefteq G$, $K \subset H$, then $K \trianglelefteq H$ $\to$ $(G\slash K)\slash (H \slash K) \cong G \slash H$

$H, K \leq G$, $H \subset N_K$ $\to$ $H \cap K \trianglelefteq H$, $HK = KH \leq G$, $\to$ $H\slash(H\cap K) \cong HK\slash K$

$H' \trianglelefteq G'$, $H = f^{-1}(H')$ $\to$ $H \trianglelefteq G$ $\to$ $\overline{f}: G\slash H \to G'\slash H'$ injective

\noindent
A \textit{tower} of subgroups of G is a sequence $G = G_0 \supseteq G_1 \supseteq G_2 ... \supseteq G_m$

Such a tower is normal if each $G_{i+1} \trianglelefteq G_i$ and abelian if each factor group is abelian

The preimage of a normal tower under a homomorphism is itself a normal tower

And similarly with the preimage of an abelian tower

Inserting finitely many subgroups into a tower yields a \textit{refinement} of that tower

A \textit{solvable} group has an abelian tower with $G_m = \{e\}$

\noindent
An abelian tower of finite G admits a cyclic refinement.

\noindent
$H \trianglelefteq G \to$ G is solvable $\leftrightarrow$ $H$ and $G\slash H$ are solvable

\noindent
A \textit{commutator} in G is an element of the form $xyx^{-1}y^{-1}$

The \textit{commutator subgroup} of G is the subgroup generated by its commutators

\noindent
A \textit{simple} group is a non-trivial group whose only normal subgroups are \{e\} and itself

An abelian group G is simple $\leftrightarrow$ G is cyclic and of prime order

\noindent
$U, V \leq G$, $u \trianglelefteq U$, $v \trianglelefteq V$, then we have the following:

$u(U \cap v \trianglelefteq u(U \cap V)$ and $(u \cap V)v \trianglelefteq (U \cap V)v$ with isomorphic factor groups, that is,

$u(U \cap V)\slash u(U \cap v) \cong (U \cap v)v\slash(u \cap V)v$

\noindent
Two towers $G = G_1 \supseteq G_2 \supseteq \cdots \supseteq G_r$, $G = H_1 \supseteq H_2 \supseteq \cdots \supseteq H_s$ are \textit{equivalent} if:

$r = s$ and $\exists i \mapsto i'$ such that $G_i\slash G_{i+1} \cong H_{i'}\slash H_{i'+1}$

\noindent
Theorem (Schreier): Given a group G and two towers of that group.

If they are normal and end with the trivial group they have equivalent refinements

\noindent
$G = G_1 \supseteq G_2 \supseteq \cdots \subseteq G_r = \{e\}$ normal, each $G_i/G_{i+1}$ simple, $G_i \neq G_{i+1}$ for $1 \leq i \leq r - 1$

Then any normal tower of G with these properties is equivalent to this tower.

\subsection{1.4: Cyclic groups}

A group G is \textit{cyclic} if $\exists a \in G$ such that $\forall x \in G$, $x = a^n$ for some $n \in \mathds{Z}$

Such an a is the \textit{generator} of G.

If $a^m = e$ and $m > 0$ m is an \textit{exponent} of a.

Such is an \textit{exponent of G} if it is an exponent of a $\forall a \in G$.

\noindent
Let G be a group, $a \in G$, $f: \mathds{Z} \to G$ defined $f(n) = a^n$ and $H = ker(f)$

If the kernel is trivial, a has \textit{infinite period} and generates an infinite cyclic subgroup

With a nontrivial kernel, its \textit{period} d is the smallest positive element of the kernel

\noindent
G a finite group, order $> 1$, $a \in G$, $a \neq e$, then the period of a divides n.

\noindent
G cyclic: every subgroup of G is cyclic, and for f a homomorphism on G, $Im(f)$ is cyclic

\noindent
Proposition:

(i) An infinite cyclic group has exactly two generators (if $a$ is one, $a^-1$ is the other)

(ii) G finite cyclic of order n, x a generator; the set of generators is $\{x^v| gcd(v, n) = 1\}$

(iii) G cyclic, a and b two generators: $\exists f \in Aut(G)$, $f(a) = b$

(iii) conversely, if $f \in Aut(G)$, $f(a)$ is some generator of G

(iv) G cyclic of order n, d positive divisor of n $\to \exists ! H \leq G$, $\#H = d$

(v) $G_1$, $G_2$ cyclic, $\#G_1 = m$, $\#G_2 = n$.  If $gcd(m, n) = 1$, then $G_1 \times G_2$ is cyclic.

(vi) G finite abelian, noncyclic $\to \exists p$ prime and $H \leq G$, $H \cong C \times C$, C cyclic of order p

\section{Lang 1.5-1.6}

\subsection{1.5: Operations of a group on a set}

\noindent
An \textit{action/operation} of G on S is $\pi: G \to Perm(S)$ and S is called a G-set

Written with a product notation, this has properties $x(ys) = (xy)s$ and $es = s$

\noindent
Conjugation has inverse, so action yields a homomorphism $G \to Aut(G)$

Its kernel is the \textit{center} of G.

Elements of its image are called \textit{inner automorphisms}.

Subsets A and B are conjugate if for some $x \in G$, $B = xAx\inv$

\noindent
Another example of an action is left-translation.

Note that the image in $Perm(S)$ under this action does not consist of homomorphisms.

\noindent
Given two G-sets and $f$ between them, if $f(xs) = xf(s)$ then f is a \textit{morphism} of G-sets

\noindent
The $x \in G$ such that $xs = s$ for a given s is the \textit{isotropy group} or \textit{stabilizer} of s

This forms a subgroup of G.

\newpage

\section{9/1}

\subsection{Group Actions $\to$ Sylow theorems}

Recall:

the stabilizer $G_s = \{g \in G | g \cdot s = s\}$

the orbit $O(s) = \{g \cdot s | g \in G\}$

$G\slash G_s \cong O(s)$ and $\#(G\slash G_s) = \#O(s)$

\noindent
Let $\Sigma =$ set of representatives for $s \sim s' \leftrightarrow O(s) = O(s')$

$\#S = \sum_{s \in \Sigma}\#O(s) = \sum_s(G:G_s)$

G finite $(G:G_s) = \frac{\#G}{\#G_s}$

Mass formula $\#S = (\sum_s\frac{1}{\#(G_s)})(\#G)$

\noindent
A subgroup H of G acted upon by G has orbits, cosets, and trivial stabilizer.

Hence from the above $\#H_s = \#H$, and $\#G = (G:H) \cdot \#H$.

This is a statement of Lagrange's Theorem, $(G:H) = \frac{\#G}{\#H}$.

\noindent
We can relate the stabilizers of points in the same orbit.

$G_s' = G_{g \cdot s} = gG_sg^{-1}$

See $(gxg^{-1})s' = (gxg^{-1})gs = g(xs)$ 

The stabilizer of $s'$ is a conjugate of the stabilizer of $s$.

\noindent
The kernel of the action $$K = \{g\in \bigcap_{s \in S}G_s\}$$

This is just the kernel of $G \xrightarrow{\phi} Perm(S)$.

\noindent
Assume $x \in G_s$.  Claim $gxg^{-1} \in G_{gs}$, showing $gG_sg^{-1} \subset G_{gs}$

Since $x \in G_s$, $(gxg^{-1})s' = (gxg^{-1})gs = g(xs) = gs$.

Applying this relation with $g \to g^{-1}$ and $s \to gs$, $G_{gs} \subset gG_sg^{-1}$

\subsection{Applications}

p : prime

\noindent
p-group: a finite group G, $\#G = p^n, n \geq 1$

\noindent
``A p-group has a non-trivial center.''

(Recall: the center $Z(G) = Z = \{g \in G | gs = sg \forall s\in G\}$).

Since $gs = sg \to s = gsg^{-1}$, will be useful to consider action on self by conjugation.

G a p-group, S a finite set.  Then $\#O(s) = \frac{\#G}{\#G_s} = \frac{p^n}{p^k}$.

Two cases: 1) $\#O(s) = 1$ s is fixed by G, $s \in S^G$ (set of fixed points of S)

2) ($k < n$), thus $\#O(s)$ is divisible by p.

$\#S =$ sum of \# of elements in the orbits $\equiv_{mod p} \#$ of orbits of size 1 $= \#(S^G)$.

Take S = G, with action $g: s \mapsto gsg^{-1}$.  Then $S^G = Z(G)$.

Thus, $\#Z(G) \equiv_{mod p} p^n \equiv_{mod p} 0$. Center has order divisible by p.

\noindent
$H \leq G$ a finite group, $(G : H) = p$, p the smallest prime dividing \#G $\to H \trianglelefteq G$

Let $S = G\slash H$; $\#(S) = (G : H) = p$, and let G act on S by left translation.

This induces $\varphi: G \to S_P$; recall $\#S_p = p!$

The stabillizer of H, $G_H = \{x \in G | xH = H\} = H$.

By inspection, we can see that $G_{gH} = gHg^{-1}$.

Let $K = \bigcap_{g \in G}gHg^{-1}$, the largest normal subgroup contained in H.

Note that $K = ker(\varphi)$ induced above; by the First Isomorphism Theorem $\varphi(G) \leq S_p$.

$(G : K) = \#(G\slash K) = \#(\varphi(G))$, which divides $\#(S_p) = p!$

Further, since $K \leq H \leq G$, $(G : K) = (G : H)(H : K)$.

Since $(G : K)$ divides $p!$ and $(G : H)$ divides p, $(H : K)$ divides $(p - 1)!$.

But p is the smallest prime dividing $\#G$, so $(H : K)$ = 1, $K = H$ and H is normal.

\noindent
A familiar embedding of a group into a larger group; ``Cauchy's Theorem''

$G \hookrightarrow Perm(G)$ by letting G act on itself by left-translation.

Its kernel $K = \{g \in G | gs=s \forall s\} = \{e\}$ (consider $s = e$), hence is an injection.

Since an injection, an embedding.

\noindent
Recall $S_n \subset$ group of $n \times n$ invertible matrices. $\sigma \mapsto M(\sigma)$ a permutation matrix.

Need to be careful in the construction to ensure $M(\sigma\tau)=M(\sigma)M(\tau)$!

E.g. $\sigma = (1 3 2)$ does $M(\sigma)$ have 1 in the 1st column, 3rd row?

Or in the 1st row, 3rd column?  One of these yields $M(\sigma\tau) = M(\tau)M(\sigma)$.

\noindent
G finite of order n; V the vector space of functions $G \xrightarrow{f} \mathds{Z}$; note $V \cong \mathds{Z}^n$

Linear maps $V \to V \leftrightarrow n \times n$ matrices over $\mathds{Z}$; this is $GL(V) \approx GL(n, \mathds{Z})$.

Similarly, invertible linear maps correspond to $n \times n$ invertible matrices over $\mathds{Z}$.

We can embed G in $GL(n, \mathds{Z})$ by using a left action of G on $GL(n, \mathds{Z}) = \{\phi: V \to V\}$

Recall that $V = \{f : G \to \mathds{Z}\}$.

This left action takes the form $L_g: f(x) \mapsto f(xg)$

\textbf{Verify for yourself} that $L_{gg'} = L_g \circ L_{g'}$ and $g \mapsto L_g$ is a homomorphism $G \to GL(V)$

Using $\mathds{F}_p$ instead of $\mathds{Z}$, get $G \hookrightarrow GL(n, \mathds{F}_p)$, an embedding into a finite group

\section{9/3}

\subsection{Sylow Theorems}

Lagrange: If $H \leq G$ then $\#(H)|\#(G)$.

$A_4$ with $n = 6$ gives the counterexample to the converse.

\noindent
Salvaging the converse: the case where $n = p^k$, p prime.

\noindent
(Sylow I): If $|G| = p^k \cdot r$, $(p, r) = 1$

$\exists H \leq G$ such that $|H| = p^k$

Such an H is called a p-Sylow (Sylow-p) subgroup of G

Generally assuming $k \neq 0$

\noindent
Example : $\mathds{Z}_{12}$

has 2-sylow subgroup $\{0, 3, 6, 9\}$ and 3-sylow subgroup $\{0, 4, 8\}$

\noindent
Example: $D_6$ generated by r, s subject to $rs = sr^{-1}$, $r^6 = e$, $s^2 = e$, has order 12

$\#(D_6) = 12$ so has 3-sylow subgroup $\{1, r^2, r^4\}$

Also has 2-sylow subgroups $\{1, r^3, s, r^3s\}$, $\{1, r^3, rs, r^4s\}$, $\{1, r^3, r^2s, r^5s\}$

\noindent
Example: G = $GL_n(\mathds{F}_p)$, $n \times n$ linear transformations in $\mathds{F}_p$, equal to $Aut(\mathds{F}_p^n)$

\noindent
The order of $|G|$:  

Asserting linear independence in each vector of an $n \times n$ matrix

$|G| = (p^n - 1)(p^n-p)(p^n-p^2)\cdots(p^n-p^{n-1}) = p^{1 + 2 + 3 + \cdots + n - 1}\cdot r = p^{\frac{n^2 - n}{2}} \cdot r$

$(p, r) = 1$

Consider P the set of all lower triangular matrices in $n \times n$.

Then $|P| = p^{1 + 2 + 3 + \cdots + n - 1} = p^{\frac{n^2 - n}{2}}$, and P is a Sylow subgroup.

\noindent
Theorem: (Sylow I) p-Sylow subgroups always exist.

\noindent
Proof Sketch:

Suppose $|H| = p^k \cdot r$, $(p, r) = 1$, $k > 0$

Show $\exists G$, $H \leq G$, where G has a p-Sylow subgroup.

Show that if G has a p-Sylow subgroup and $H \leq G$, then H has a p-Sylow subgroup

\noindent
Proof:

By Cayley's theorem, if $|H| = n$, then $H \leq S_n$.

(H acts on itself by left translates.  This yields an embedding into $S_n$.)

Additionally $S_n \leq GL_n(\mathds{F}_p)$ mapping to permutation matrices.

We know that $GL_n(\mathds{F}_p)$ has p-Sylow subgroups.

Let P be a p-Sylow subgroup of G.  Consider G acting on the set of cosets of P.

Now, $Stab(gP) = gPg^{-1}$.

Similarly, letting H act on $G \slash P$, $Stab(gP) = (gPg^{-1} \cap H)$

This intersection is a p-group.

Want to choose $g \in G$ such that $gpg^{-1} \cap H$ is a p-Sylow subgroup.

If $(H : (gPg^{-1} \cap H))$ is coprime to p, then $gPg^{-1} \cap H$ is a p-Sylow subgroup.

By Orbit-Stabilizer, $(H : (gPg^{-1} \cap H)) = O(gP)$.

$|G\slash P| \not\equiv 0$ (mod p), and the sum of the orbits is $|G \slash P|$

Hence there must be some orbit with size coprime to p.

\noindent
Corollary of proof, using fact that these have the form $gPg^{-1} \cap H$.

\noindent
Statement of the corollary: (Sylow II) All p-Sylow groups are conjugate.

\noindent
Proof:

Let $H \leq G$, $P \leq G$ p-Sylows.  Then $H \cap gPg^{-1}$ is a p-Sylow of H for some $g \in G$.

Since H is a p-group $H \cap gPg^{-1} = H$ i.e. $H \subset gPg^{-1}$. \textbf{I don't know why this is true.}

Then $|H| = |P| = |gPg^{-1}|$, so $H \cap gPg\inv = H$.

\noindent
Corollary of Proof of Sylow I:

If $|G| = p^k \cdot r$, then $\exists H_i \subset G$ such that $|H_i| = p^i$, for $0 \leq i \leq k$.

\textbf{Proof left to student: not sure what it is.}

\noindent
Let $Syl_p(G)$ describe the p-Sylow subgroups of G and $n_p$ denote its cardinality.

\noindent
Theorem (Sylow III) If $|G|= p^k \cdot r$, $k > 0$ then $n_p \equiv 1 (p)$ $n_p | r$.

\noindent
Lemma: If $\Gamma$ acts on X, X a set, $\Gamma$ a p-group (finite)

Then $\#X \equiv \#Fix_\Gamma(X) (mod p)$, where $Fix_\Gamma(X)$ is the set of elements of x fixed by all of $\Gamma$

\noindent
Proof:

$\#X = \sum_i \#Orb(x_i) = \sum_i \frac{|\Gamma|}{|Stab(x_i)|} \equiv \#Fix_\Gamma(X) (mod p)$

Each $\frac{|\Gamma|}{|Stab(x_i)|} = 1$ if $x_i$ fixed, else 0

\noindent
Proof of the Theorem:

Let $Syl_p(G)$ act on itself by conjugation.

By the lemma, $\#Syl_p(G) = n_p \equiv Fix_p(Syl_p(G)) (mod p)$.

Suppose Q is fixed.  Then $pQp\inv = Q$ $\forall p \in P$.

Then $P \leq N(Q)$; similarly $Q \leq N(Q)$.

P, Q are p-Sylow subgroups of N(Q); therefore P, Q are conjugate in N(Q).

However, $Q \trianglelefteq N(Q)$ so that $P = Q$.

Further, P is the only such Sylow-p subgroup that works so $Fix_p(Syl_p(G)) = 1 (mod p)$

G acting on $Syl_p(G)$ as only one orbit since all p-Sylows in G are conjugate.

Stab. $n_p = |G| = p^k \cdot r$, $n_p | p^k \cdot r$, but $n_p \nmid r$, so $n_p | r$.

\section{9/8}

\subsection{Review of Sylow Theorems}

Prove existence by showing existence in a larger known subgroup.

And then that contained subgroups must have their own Sylow p-subgroups.

$O(s) = S = \{$p-Sylows$\}$

$O(s) = G\slash G_s = G \slash N(P)$

The number of p-Sylows is notated $n_p = (G : N(P))$

\noindent
$P, Q$ p-Sylows and $P \subset N(Q)$ then $P = Q$

reason: $PQ \leq G$ a subgroup of G

$HK$ not necessarily a group, but will be if one normalizes the other

ie $H \subset N(K)$

\noindent
Theorem $n_p \equiv 1$ (mod p)

Consider the action of $P$ on $S$ by conjugation

Take $x \in P$ and $x: Q \mapsto xQx\inv$

The number of fixed points is 1, since $P$ fixes only itself

\noindent
A simple group has

more than one element

no non-trivial proper normal subgroups

(kind of like a prime number)

\noindent
G finite abelian

G simple $\leftrightarrow$ G cyclic of prime order (simple easy exercise)

\subsection{continuing...}

\noindent
non-sporadic finite simple groups

$A_n (n \leq 5)$

recall the alternating groups $A_n$ are the even permutations on $\{1, \cdots, n\}$

Lie groups over finite fields, e.g. $\{\pm \begin{pmatrix} 1 & 0 \\ 0 & 1 \\ \end{pmatrix}\} \subset SL(2, \mathds{Z}|p\mathds{Z})$

P = projective; $PSL(2, \mathds{Z}|p\mathds{Z}) = SL(2, \mathds{Z}|p\mathds{Z})$

\noindent
Simple groups of order $\leq 60$.

(a) There are none of order $< 60$ (HW)

(b) If G is simple of order 60, then $G \cong A_5$.

($\#A_n = \frac{n!}{2}$)

\noindent
G simple of order 60.

$H < G$ simple (finite), H proper, $(G: H) = n \geq 2$

G acts on $G\slash H$ by left translation.

The action is transitive (for each pair $xH, yH$, $\exists$ permutation taking one to the other)

Therefore, this action is non-trivial.

$\pi : G \to Perm(G\slash H) = S_n$

$ker(\pi) \neq G$ and is a normal subgroup $\to$ the kernel is trivial.

$\pi: G \hookrightarrow S_n$ and in fact $\pi: G \hookrightarrow A_n$ (if $\#G > 2$)

\noindent
Why? because $G \cap A_n \trianglelefteq G$

If $G \subset S_n$.

Then $G \to S_n/A_n = \{\pm 1\}$ by the sign map, kernel is $G \cap A_n$.

Recall $sgn: S_n \to \{\pm 1\}$ $sgn(\sigma) = (-1)^t$ given t, num of transpositions

$G/(G \cap A_n) \hookrightarrow S_n/A_n = \{\pm 1\}$

$(G : G \cap A_n) = 1$ or $2$.

If G is simple then this cannot be 2 (would be normal subgroup), so =1.

And $G \hookrightarrow A_n$ for that $A_n$.

\noindent
G simple, order 60.

H a proper subgroup of G, index n.  (consider small values of n)

If $n = 2$ then H is normal in G, a contradiction.

(smallest prime dividing the order of a group)

If $n = 3$ or $n = 4$: $G \hookrightarrow A_3, A_4$ but their orders are too small (3, 12)

If $n = 5$: $G \hookrightarrow A_5$ and they are equal in cardinality $\to$ done.

Remaining case: $n = 15$.

What is $n_5$, the number of 5-Sylow subgroups.

$n_5 | 60\slash 5 = 12$, $n_5 = (G : N(P))$ $n_5$ divides the index

Also, $n_5 \equiv 1$ (mod 5).

Thus $n_5 = 1$ or $n_5 = 6$.

If $n_5 = 1$ then only one 5-Sylow subgroup of G, must be normal.

This is impossible since G is simple.

Then $n_5 = 6$: tells you there are lots of elements of order 5 in G.

There is no overlap (excepting at the identity) between 5-Sylows.

Hence the number of elements of order 5 is $6 \cdot 4 = 24$

Elements of order 5 in $A_5$ are 5-cycles (a b c d e).

Need to take all strings of length 5: 120, and divide out by rotations 5.

Thus we get $120 \slash 5 = 24$ (check).

Consider $n_2$ the number of 2-Sylow subgroups.

Then $n_2$ divides $60/4 = 15$, and $n_2 \neq 1$ because of simplicity.

Also, $n_2 = (G : N(P_2))$, and this can't be 3 since G has no subgroup of index 3.

If $n_2 = 5$ then $N(P_2)$ is the desired index-5 subgroup $\to$ done.

From divisibility $n_2 = 1, 3, 5, 15$.

Elminate 1 by simplicity, 3 since the index is too small, 5 works, consider 15.

\noindent
Considering the situation where there are 15 2-Sylow subgroups (of order 4).

These are groups like the Klein 4-group (no elements of order 4).

There are 2 2-Sylow subgroups P and Q where $P \cap Q$ has order 2.

Prove by counting.

Taking intersection, must be proper else they would be the same.

Hence $P \cap Q$ has order 1 or 2.

If there is utterly no overlap, there are $15 \cdot 3 + 1 = 46$ elt's of 2-Sylows.

And these do not have order 5.  But there are 24 elements of order 5.  Too many.

\noindent
Now we know that some of these 2-Sylow subgroups have non-trivial overlap.

Consider $N(P \cap Q)$ for some such intersection, will be a subgroup of G.

Cannot be all of G, G is simple.

Moreover, $P, Q \leq N(P \cap Q)$.  Now, $N(P \cap Q)$ cannot contain $P$ or $Q$.

Therefore, its index cannot be 1 (G is simple) cannot be 3, $A_n$ too small, = 5.

\noindent
Jordan-H\"{o}lder theorem

Website reference.

\noindent
G finite non-trivial.  Is G simple?  $\{e\} \subset G$, $G/\{e\}$ simple.

Not simple $G \supset G_1 \supset (e)$, $G_1 \trianglelefteq G$, $G_1$, $G/G_1$ smaller than G.

Keep going until 'end', using principle of string induction.c

\noindent
Proposition: $\exists G = G_0 \supset G_1 \supset G_2 \supset \cdots \supset G_n$, $G_{i + 1} \trianglelefteq G_i$, $G_i/G_{i+1}$ simple.

A \textit{normal tower} or \textit{composition series}, the simple quotients are the \textit{constituents}.

Obtain a successive extension of simple groups.

\noindent
Main point.

$N = p_1\cdots p_n$

$\{p_1, p_2, \cdots, p_n\}$ a set where order doesn't count but multiplicity does.

Gauss's theorem: (FTA) each prime decomposition of N yields the same set.

\noindent
Similarly, given $G$ and $G_i/G_{i+1} = Q_i$ and $\{Q_0,\cdots , Q_{n-1}\}$.

Order not mattering, multiplicity matters, up to isomorphism.

Theorem: Each composition yields the same multiset.

Theorem of ``Camille Jordan and some guy named H\"{o}lder.''

\section{9/10}

\end{document}
