\documentclass[12pt]{article}

\usepackage{mathptmx}
\usepackage{mathpazo}

\usepackage{amsfonts}
\usepackage{amsmath}
\usepackage{amssymb}
\usepackage[version-1-compatibility]{siunitx}
\usepackage{fixltx2e}
\usepackage{multirow}
\usepackage{dsfont}

\topmargin -0.5in %topmargin=1in+\topmargin
\textheight 9in % default is letter % so 11-2=9in
\textwidth 6.5in % and 8.5-2in=6.5in
\oddsidemargin 0in %left margin = % 1in + oddsidemargin
\footskip 1cm % pagenumber to text

\setcounter{secnumdepth}{0}

\title{CS 188}
\date{\normalsize Fall 2015}


\begin{document}
\maketitle

\section{8/27}

\subsection{Introduction to AI}
\noindent
\textit{Rationality} is defined in terms of achieving maximum utility by some pre-defined metric or set of goals/intentions.  Rationality depends only on the usefulness of the choice reached rather than on any aspect of the process that led to that choice.  For example, a rational process for playing tic-tac-toe could be formed simply by creating a table for all game states; this would not have consist of any decision process whatsoever.

\noindent
AI, economics, statistics, operations research, etc. assume utility to be \textbf{exogenously specified}.  The difficulty of competent machines is a question of value misalignment.

\end{document}


