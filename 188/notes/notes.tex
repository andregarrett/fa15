\documentclass[12pt]{article}

\usepackage{mathptmx}
\usepackage{mathpazo}

\usepackage{amsfonts}
\usepackage{amsmath}
\usepackage{amssymb}
\usepackage[version-1-compatibility]{siunitx}
\usepackage{fixltx2e}
\usepackage{multirow}
\usepackage{dsfont}
\usepackage{indentfirst}

\topmargin -0.5in %topmargin=1in+\topmargin
\textheight 9in % default is letter % so 11-2=9in
\textwidth 6.5in % and 8.5-2in=6.5in
\oddsidemargin 0in %left margin = % 1in + oddsidemargin
\footskip 1cm % pagenumber to text

\setcounter{secnumdepth}{0}

\title{CS 188}
\date{\normalsize Fall 2015}


\begin{document}
\noindent
CS 188, Fall 2015\\

\subsection{Intelligent Agents}

\noindent
Rationality: achieving maximum utility by some pre-defined metric or set of goals/intentions.

depends on the usefulness of the choice and not on the process that led to that choice

e.g., rational process for playing tic-tac-toe by tabulating game states

has no decision process, but acts rationally

\noindent
An agent has sensors which take in percepts and actuators which effect actions

a percept is a set of perceptual inputs at a fixed point in time

a percept sequence is composed of percepts

\noindent
An agent function is $f: P^* \to A$ where $P^*$ is the percept sequence and $A$ the set of actions

the agent program is the specific architecture which implements this function

not all agent functions can be implemented by some agent program

e.g. halting problems, NP-hard problems, ``too-large'' problems (e.g. chess)

\noindent
Performance measure: an objective criterion for the success of behavior of an agent

\noindent
Rational agent: maximizes expected performance measure by its actions

given the prior knowledge and percept sequence available to it

e.g. vacuum world: 2 squares, could be dirty or clean

action function: suck if dirty, move if clean

under the measure of most clean squares/time period, this is rational

if we seek to minimize movements as well, this is irrational

\noindent
Autonomy: the ability to function beyond the prior knowledge of the designer\\

\noindent
The task environment: performance measure, environment, actuators, and sensors

partially vs. fully observable (perceive all aspects relevant to choice of action)

stochastic vs. deterministic (next state a function only of current state, action)

strategic: deterministic but for the actions of other agents

episodic vs. sequential (current decision could affect future decisions)

static vs. dynamic (environment can change while agent deliberates)

semidynamic: environment doesn't change with time but performance score does

semidynamic e.g. chess with a clock

discrete vs. continuous (can apply to state, time, percepts/actions)

single vs. multiagent: e.g. competitive multiagent, cooperative multiagent\\

\noindent
Agent structure

the agent program only takes in the current percept

the agent function maps from the entire percept history\\

\noindent
A simple reflex agent uses only the current percept in its decision process

responds via condition-action rules

requires full observability to work effectively

\noindent
A state-based reflex agent maintains internal state using the percept history

needs what is called a model of the environment

knowledge about how it evolves, how actions affect it

also called a model-based reflex agent

\noindent
A goal-based agent seeks to achieve states which are considered favorable

Unlike reflex agents, makes use of foresight in its decision process

\noindent
A utility-based agent evaluates future states using a utility function

Seeks to maximize utility of future states

\noindent
A learning agent can be broken down roughly into four components

A learning element is responsible for improving the agent

A critic evaluates current performance and informs the learning element

(uses a fixed performance standard)

A performance element selects actions

A problem generator suggests exploratory actions

\noindent
Two of these are reflexive, the next two are planning-based.

Planning agents predict consequences of actions using a transition model.

\noindent
Agents can vary in the degree to which they deliberate

one extreme: carefully construct a complex plan

another: start with a simple plan and rapidly correct as complications arise

\subsection{Search Problems}

\noindent
We consider problem-solving agents, which set out to achieve some desirable state

an uninformed search algorithm has no idea of where to look for solutions

relies only upon the problem definition

\noindent
An agent plan in a search problem will first formulate, then search, then execute

\noindent
Problem formulation:

simplifies environment, abstracts away unnecessary information

helps organize the behavior of the agent

\noindent
Components of a well-defined problem:

(1) Initial state

(2) Possible actions: i.e., a successor function $f: \{$states$\} \to \{($action, consequent state$)\}$

These first two implicitly define a state space, and a state space graph

State space graph has states as nodes and actions as edges

(3) Goal test

(4) Path cost: (e.g. induced by edge costs of a state space graph)

A path is a sequence of actions.

\noindent
A solution is a path from an initial state to a goal state.

An optimal solution has lowest path cost.

\noindent
Environments we are considering are static, observable, discrete, and deterministic.\\

\noindent
Generalized tree search

envision the search space as a tree (states can be revisited)

nodes have: state, parent-node, action leading to that node, path-cost, and depth

the root of the search tree corresponds to the initial state of the problem

uses a fringe, initialized to contain only the root node

the fringe contains nodes which have been generated but not yet expanded

\noindent
At each node in the process:

perform goal test on the node in question, if succeeds, return corresponding solution

expand node by i.e. applying successor function to generate new nodes

if no such nodes exists, declare failure

choose which node to analyze next according to some search strategy

\noindent
Evaluation of a search strategy.

Completeness: whether or not it finds a solution, if a solution exists.

Optimality: if it finds a solution, whether or not such a solution is of lowest cost.

Complexity: space (nodes stored at any given time) or time (total nodes explored)

\noindent
Factors often used to evaluate the complexity of a given algorithm.

Branching factor $b$: the maximum number of successors of any node.

Depth $d$: the minimum depth among goal nodes.

Maximum path length $m$: can be infinite.\\

\noindent
Types of uninformed (aka blind) search strategies

contrast: informed search strategies take advantage of heuristics

cannot solve searches of exponential complexity for all but the smallest problems

\noindent
Breadth-first search uses a queue (FIFO) as its fringe.

Satifies completeness (if branching factor finite) but not optimality.

Identical space and time complexity (holds all nodes in state until goal is found)

Exponential complexity: $\mathcal{O}(b^{d + 1})$

In practice, spacial complexity too hard: very difficult to accrue enough memory

\noindent
Uniform-cost search orders its fringe by path cost.

As long as each step cost $> \epsilon> 0$, satisfies completeness \textit{and} optimality.

Again, same time and space complexity

Given $\epsilon$ minimum action cost, $C*$ optimal solution cost, complexity is $\mathcal{O}(b^{\lceil C*/\epsilon \rceil})$

\noindent
Depth-first search uses a stack (LIFO) as its fringe

Node storage is not exponential, space complexity is $\mathcal{O}(bm)$.

Backtracking search: nodes store set of successors, search expands only one at a time.

Backtracking search can reduce the space complexity to $\mathcal{O}(m)$.

Worst-case time complexity $\mathcal{O}(b^m)$ where potentially $m \gg d$ and even $m = \infty$.

Complete if $m < \infty$, but not optimal.

\noindent
Depth-limited search

specifies a depth limit $l$, and performs a depth-first search up to that limit

if $l < d$, will be incomplete, and if $d < l$, can be non-optimal

has a time complexity of $\mathcal{O}(b^l)$ and a space complexity of $\mathcal{O}(bl)$.

the diameter of the state space: min number of actions between any two states

is a great depth limit, but, we don't necessarily know the diameter, a priori

\noindent
Depth-first search by iterative deepening

apply depth-limited search with $l$ ranging over $\mathds{N}$

like DFS, has good memory (spatial complexity) at $\mathcal{O}(bd)$

like BFS, if $b$ is finite will be complete, and if path cost corresponds to depth, optimal

repeated state generation not costly: most nodes in the bottom level

generated nodes: $d(b) + (d - 1)b^2 + \cdots + (1)b^d$ (factor of $d$ is node repetition)

hence the time complexity is $\mathcal{O}(b^d)$

better than BFS ($\mathcal{O}(b^{d+1})$): IDS generates no nodes beyond the solution depth

\noindent
When search space is large and solution depth unknown, depth-first IDS is generally best.

\noindent
Bidirectional search run simultaneous searches from initial state and goals

Concludes when the searches meet.

Time and complexity $\mathcal{O}(b^{d/2})$, is reduced.

However, requires effective computation of predecessors: often non-trivial/impossible.\\

\noindent
Avoiding repetition of states: Graph Search

adds a closed list to tree search: a list of all nodes which have been expanded

the term open list is sometimes used to refer to the fringe

current nodes which match a node on the closed list are discarded

possibly suboptimal, if search methods can reach nodes at a non-optimal cost first

thus uniform cost graph search is optimal but IDS graph search may not be

the closed list increases space requirements, possibly to unfeasibility

behavior of closed list is such that the memory used is proportional to the runtime\\

\noindent
Partial Information Search

\noindent
Sensorless Problems

know consequences of actions and the possible states

can coerce world into a particular state, with some cleverness

uses a belief state: a set of states currently regarded as possible

a solution is a path to a belief state in which all its members satisfy the goal test

this approach can be analogously applied to nondeterministic problems

\noindent
Contingency Problems

agent can obtain new information from sensors after performing actions

solution: use an unfixed action sequence, with dependencies on percepts received

the action at each node will depend on all percepts received up to that point

agent can act before finding a guaranteed plan

approach referred to as interleaving search and execution

does not need to account for \textit{all} contingencies, simply the ones that occur

\subsection{Informed Search and Exploration}

\noindent
Informed search uses problem-specific knowledge beyond the definition of the problem

\noindent
Approach: select nodes for expansion based on an estimate of distance to goal

Use a priority queue ordered by some evaluation function $f$

Called best-first search: since we order by nodes which seem best

A heuristic $h: \{$nodes$\} \to \mathds{R}^+$ estimates cost of cheapest path to a goal node\\

\noindent
Greedy best-first search uses $f = h$

susceptible to false starts, dead ends

same defects as DFS: not optimal, incomplete, worst-case time/space complexity $\mathcal{O}(b^m)$

$m$ the maximum depth of search space

\noindent
A* search uses $f = g + h$ where $g$ is the path cost to a node

hence $f$ is the estimated cost of cheapest solution through n

for tree search, if $h$ is admissible (never overestimates), A* will be optimal

for graph search, need to ensure the first generated path is optimal

this shall occur if $h$ is additionally consistent/monotone

consistent $h$ satisfies, for nodes $n, n'$ and action $n \xrightarrow{a} n'$, $h(n) \leq c(a) + h(n')$

satisfies a triangle inequality; i.e., $h$ must be a metric on the state space

key consequence of a consistent $h$: $f$ is always increasing along a path

note: consistency implies admissibility

A* is optimally efficient for a given $h$: $\not \exists$ an optimal, more efficient algorithm

however, nodes in goal contour search space still increase exponentially, 

unless $|h - h^*| \leq \mathcal{O}(log(h^*(n)))$ where $h*$ is the true cost

generally $|h - h^* = \mathcal{O}(h^*(n))$ at best

thus, it's often impractical to insist on finding an optimal solution

A* keeps all generated nodes in memory $\to$ impractical for large-scale problems\\

\noindent
Improvements on A* with respect to space complexity

\noindent
Can try iterative deepening A* (IDA*) using $f = g + h$ rather than $d$ as the cutoff

but this incurs substantial overhead

\noindent
Recursive best-first search

tracks the f-value of the current best alternative path

winds back to this alternative path once $f$ considered exceeds that stored $f$

optimal if $h$ is admissible, with space complexity $\mathcal{O}(bd)$

time complexity can vary, depends on $h$, frequency of path changes

can potentially explore a state multiple times (typical tree-search problem)

stores only the value of $f$ and $\mathcal{O}(bd)$ nodes

\noindent
MA* (memory-bounded A*) and SMA* (simplified MA*) make use of all memory

description of SMA*: keep expanding until memory is full

once memory is full, replace the worst (by $f$) node in memory, breaking ties by age

SMA* is complete if there is enough memory to hold the shortest path to a goal

practically, probably best for a graph state space and non-uniform path costs

if too much switching, problems that A* would solve become intractable for SMA*

time $\leftrightarrow$ space tradeoff

\noindent
Learning to search better

metalevel learning algorithm analyzes current method, seeks improvements\\

\noindent
Admissible heuristic creation

often solutions to relaxed problems, where new actions are available

\noindent
A heuristic $h$ dominates $h'$ if $h \geq h'$ on the state space

higher admissible heuristics are stronger

trade-off between computation on heuristic efficacy and in searching

taking the maximum over a set of admissible heuristics can be useful

\subsection{Local Search}

\noindent
In which the path does not matter, only the achievement of the goal state

also useful to solve pure optimization according to an objective function

\noindent
state space lanscape: manifold of the objective function on the state space

seek global extrema: completeness and optimality defined as before\\

\noindent
hill-climbing search (greedy local search):

take any actions that improve the situation

foiled by local extrema, ridges, plateaux on the manifold

dealing with shoulders: allow sideways moves, but need to limit, because of plateaus

random-restart hill climbing guaranteed to eventually find goal state

\noindent
simulated annealing allows bad moves occasionally via thermodynamic principles

randomly chooses an action, always goes ahead with improvements

if a downgrade, accept it with probability dictated by a Boltzmann weighting

start at high temperature, and slowly lower T

the algorithm is guaranteed to find a global optimum with probability $\to 1$

\noindent
local beam search runs k copies of a local search algorithm

at each step, generate all the successors, and choose the k best successors

to keep diversity among states high, can use stochastic beam search

stochastic beam search introduces some randomness, probability $\propto$ value

\noindent
genetic algorithms: successors are generated as combinations of parent states

k randomly generated states, use evaluation function as a judge of fitness

crossovers and mutations generate new states

works well in some cases, but not others

\subsection{Uncertain Search}

\noindent
contingency plans to account for nondeterminism, stochasticity

branch for each possible result following a given action, form a conditional plan

\noindent
and-or search trees:

or level on actions (only need 1 to work)

and level on nondeterminism (need a plan to work for all branches)

contingent solution cuts down on the tree by selecting actions

all the leaves need to be goals

can implement and-or search as corecursion with an and and an or function

\noindent
cyclic solution

all leaves are goal states

every point in the plan has a path to every leaf

\noindent
belief states to account for partial observability

new transition model: predict based off of the action, update based off of the percepts

\subsection{Adversarial Search}

\noindent
Formally define game as a search problem with:

initial state

successor function from states to (move, state) pairs

terminal test

utility function or objective function (i.e. a score)

these first two yield a game tree

\noindent
an strategy is considered optimal relative to an infallible opponent

\noindent
the minimax value of a node is its utility assuming mutual infallibility

algorithm computes best decision from current state, using recursion

\noindent
$\alpha$-$\beta$ pruning

$\alpha$ is the value of the current best for max

$\beta$ is the value of the current best for min

if choosing better node ordering for pruning possible, $\mathcal{O}(b^d) \to \mathcal{O}(b^{d/2})$

random order for pruning leads to approximately $\mathcal{O}(b^{3d/4})$

a transposition table is a hash table of previously seen positions

\noindent
evaluation function: a heuristic for when depth is large

a weighted linear (or nonlinear) function of features

quiescent position: unlikely to exchibit wild swings in value in near future

it is best to expand until you are at quiescent states

horizon effect: minimax avoids a very good move for the opponent

even when it must eventually happen

better algorithms can see if there's a horizon effect present, accept it and move forward

forward pruning is also possible, but dangerous (could prune best paths)

\noindent
impact of chance $\to$ chance nodes

expectiminimax: use expected value over chance nodes

decisions invariant under positive linear transformations of the evaluation function

by contrast, any isotone transformation preserves the decisions of minimax

\subsection{Constraint Satisfaction}

\noindent
constraint satisfaction problems start to upack the black box:

search and game-playing are highly abstracted atomic representations

constraint satisfaction problems, propositional logic are factored representations

even more complex are structured representations such as first-order logic

\noindent
CSP breaks state down into variables $X_i$, each of which takes values from a domain $D$

Goal test represented by a set of constraints upon allowed values

expressible implicitly $Val(X) \neq Val(Y)$ or explicitly (enumerate all possibilities)

explicit expression e.g. $(X, Y) \in \{(1, 2), (2, 3), \cdots \}$

\noindent
Binary CSP: each constraint relates at most two variables

any non-binary CSP can be converted to a binary CSP

can then be represented by a graph, vertices are variables, edges are constraints

\noindent
Characterizing a finite CSP: $n$ variables, $d$ the maximum domain size

$\mathcal{O}(d^n)$ complete assignments exist

generally no better than exponential time, in the worst case\\

\noindent
Search formulation for solving a CSP

initial state: $\{\}$, action assigns a value to a variable

continue until all variable assigned and all constraints satisfied

prior methods (e.g. BFS, DFS) highly inefficient

\noindent
backtracking search is a depth-first search with the following alterations:

variable assignment is commutative: apply assignment to variables in a fixed order

reducing branching factor from $nd$ to $d$

only considers values that do not conflict with previous assignments

\noindent
improvements to backtracking

smallest domain first: variable ordering by minimum remaining values (MRV)

break ties by degree heuristic: choose variable with most ties

goal: fail as quickly as possible, to eliminate large sections of the tree

least restrictive assignments: value ordering by least constraining value (LCV)

\noindent
making inferences about the domains, using filtering

forward checking eliminates values of adjacent variables that violate a constraint

\noindent
better inferences: maintain arc consistency to detect failures earlier

consistent arc $(X, Y)$: $\forall x$ $\in dom(X)$, $\exists y \in dom(Y)$ satisfying constraints

repeatedly check arc consistency to pare down the domains

\noindent
can apply algorithm separately to connected components, independent

breaking down greatly reduces complexity $\mathcal{O}(d^n) \to \mathcal{O}(\frac{n}{c}d^c) = \mathcal{O}(n)$\\

\noindent
Tree-Structured CSPs have graph representations which are trees

Can be solved in $\mathcal{O}(nd^2)$ time, no longer exponential

choose any ordering, pick a root, make a linear chain

apply arc consistency (working backwards)

make assignments (moving forwards)

worst case: check d values against each of the d values, n checks up, n checks down

\noindent
simplify CSPs down to tree-structured CSPs if possible to reduce runtime

set a value for some variables first if helps to reduce down to a tree-structure

conditioning: instantiate a variable to affect the rest of the CSP favorably

cutset conditioning: instantiate variables such that the remaining graph is a tree

can then compute residual CSPs for each of the possible cutset value assignments

\noindent
local search for CSPs: min-conflicts algorithm

while not solved, randomly select any conflicted variable

value selection: heuristic by minimum resulting conflicts

extremely good for problems dense in solutions

CPSs are very good at dealing with randomly-generated CSPs

although exists a critical ratio $R = \frac{constraints}{variables}$ at which becomes very very bad

\subsection{Logic}

\noindent
Built off of a knowledge base: a set of sentences in some formal language

Add sentences to the knowledge base

Apply a process of inference to determine actions

Inference engine independent of the knowledge on which it acts

Inference algorithm/reasoning allows for universality: can act on any knowledge

Reduces to a question of considering the knowledge base

\noindent
Syntax: rules for allowable sentences

\noindent
Semantics: possible worlds, truth relation between sentences and worlds

\noindent
E.g. Propositional Logic: possible worlds are assignments of TF to variables

Semantics: $\alpha \wedge \beta$ is true in a world iff $\alpha$ is true and $\beta$ is true

\noindent
E.g. First-order logic

Syntax $\forall x \exists y P(x, y) \wedge \neg Q(Joe, f(x)) \to f(x) = f(y)$

Possible world:

Objects $o_1, o_2, o_3$; $P$ holds for $\langle o_1, o_2 \rangle$; $Q$ holds for $\langle o_3 \rangle$; $f(o_1) = o_1$; $Joe = o_3$; etc.

Semantics: $\phi(\sigma)$ is true if $\sigma=o_j$ and $\phi$ holds for $o_j$

\noindent
Entailment: $\alpha \models \beta$, $\alpha$ entails $\beta$ or $\beta$ follows from $\alpha$

the $\alpha$-worlds are a subset of the $\beta$-worlds [models($\alpha$) $\subset$ models($\beta$)]

the entailment of $\beta$ makes it at most as strong as $\alpha$, and possibly weaker

\noindent
A proof is a demonstration of entailment

\noindent
Method 1: check in every possible world, that if $\alpha$ is true then $\beta$ is true too

Semi-decidable: if cannot be proven in this fashion, has no way of indicating such

\noindent
Method 2: exhibit a sequence of applications of inference rules taking $\alpha$ to $\beta$

\noindent
Sound inference algorithm: everything that it claims to prove is entailed

\noindent
Complete inference algorithm: everything that is entailed can be proven\\

\noindent
Propositional logic

Have a set of symbols, with distinguished symbols 'True' and 'False'

Sentences are generated by the set of symbols under $\neg, \wedge, \vee, \to, \leftrightarrow$

Semantics: symbols have truth values, can recurse over syntax to evaluate sentences

\noindent
Forward chaining: given $X_1 \wedge X_2 \wedge \cdots X_n \to Y$ and $X_1, X_2, \cdots, X_n$, infer $Y$

knowledge base only contains definite clauses (of the above form)

therefore, cannot deal with disjunctions; no reasoning by cases

forward chaining algorithm: have table counting number of symbols in each premise

iterate over symbols, decrement count for each clause that has that symbol in premise

allows for $\mathcal{O}(n)$ where $n$ is the size of the knowledge base

sound (since Modus Ponens is sound) and complete for definite-clause KBs\\

\noindent
Simple model checking for entailment: recursive enumeration of all worlds

go through all possible worlds (sets of TF assignments) to all symbols

for all worlds where KB is true, make sure that $\alpha$ is true

shows when $\alpha$ is entailed, woefully inefficient ($\mathcal{O}(2^n)$ time, linear space)

\noindent
A sentence is satisfiable if it is true in at least one world

SAT solvers take a sentence in conjunctive normal form and determines satisfiability

using a SAT (satisfiability) solver to check entailment

if $\alpha \models \beta$, then $\alpha \to \beta$ in all worlds

hence $\neg(\alpha \to \beta)$ is false in all worlds

hence $\alpha \wedge \neg \beta$ is false in all worlds

if can show that $\alpha \wedge \neg \beta$ is unsatisfiable then $\alpha \models \beta$

analogous to an proof by contradiction

\noindent
DPLL SAT solver: backtracking search over models with:

early termination: stop immediately if all clauses are satisfied, any clause is falsified

conj. normal form will be e.g. $(A \vee B) \wedge (A \vee \neg C)$; this has 2 clauses

pure literals: symbol always has same sign in to-go clauses, just assign it the value

e.g. $(A \vee B) \wedge (A \vee \neg C) \wedge (C \vee \neg B)$ then set $A$ to be true

unit clauses: clause left with single literal, set symbol to satisfy clause

DPLL efficient enough to solve up to 100 variables

Tricks to improve efficiency:

order variables and values (just like CSPs)

divide into pieces if you can see that two sections don't depend on each other

cache unsolvable subcases as extra clauses to avoid redoing them

with these improvements, can solve problems with ten million variables

\subsection{Logical Agents}

\noindent
Knowledge-based agent:

percepts added to knowledge base, after converted to some logical sentences

figures out what its next action shall be, performs action

then it adds to knowledge base the fact that it has performed this action

\noindent
Initial knowledge possessed by the agent

Sensor model: how the current percept is generated from the current state

transition model: how the next-state determined by action, current state

initial conditions: initial state

domain constraints: certain conditions that are generally satisfied

\noindent
Set the knowledge, and then the SAT-solver does all the work



\end{document}

