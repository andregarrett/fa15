\documentclass[12pt]{article}

\usepackage{mathptmx}
\usepackage{mathpazo}

\usepackage{amsfonts}
\usepackage{amsmath}
\usepackage{amssymb}
\usepackage[version-1-compatibility]{siunitx}
\usepackage{fixltx2e}
\usepackage{multirow}
\usepackage{dsfont}
\usepackage{indentfirst}

\topmargin -0.5in %topmargin=1in+\topmargin
\textheight 9in % default is letter % so 11-2=9in
\textwidth 6.5in % and 8.5-2in=6.5in
\oddsidemargin 0in %left margin = % 1in + oddsidemargin
\footskip 1cm % pagenumber to text

\setcounter{secnumdepth}{0}

\title{CS 188}
\date{\normalsize Fall 2015}


\begin{document}
\maketitle

\section{8/27}

\textit{Rationality} is defined in terms of achieving maximum utility by some pre-defined metric or set of goals/intentions.  Rationality depends only on the usefulness of the choice reached rather than on any aspect of the process that led to that choice.  For example, a rational process for playing tic-tac-toe could be formed simply by creating a table for all game states; this would not have consist of any decision process whatsoever.

AI, economics, statistics, operations research, etc. assume utility to be \textbf{exogenously specified}.  The difficulty of competent machines is a question of value misalignment.

\section{9/1}

\textit{Sensors} are preceptors of the environment and \textit{actuators} are methods by which it manipulates the environment through actions.

An \textit{agent function} maps from percept histories to actions.

$$f: P^* \to A$$

An \textit{agent program} I runs on some machine M to implement f:

$$f = Agent(I, M)$$

Not all agent functions can be implemented by some agent program.  E.g. halting problems, NP-hard problems, chess (combinatorically large amounts of information needed to process)

Use a \textit{performance measure} to evaluate effectiveness.  Care must be taken to ensure that the performance measure accurately measures the execution of the task designed.  A performance measure is a measure on the \textit{environment}.

A \textit{rational agent} maximizes the expected value of the performance measure.  Rationality depends on prior knowledge of environment, action, previous percepts.

PEAS model: performance measure, environment, actuators, sensors.

Environment characteristics: observable (fully/partial), number of agents, deterministic/stochastic, static/dynamic, discrete/continuous, known/unknown (e.g. don't know the dynamics of the system, but still try to maximize utility)

\noindent
Effects of the environment on agent design:

partially observable $\to$ agent requires memory

multi-agent $\to$ randomness may be necessary

static $\to$ can utilize time to implement a rational decision

continuous time $\to$ will have some continuously operating controller

\noindent
Agent types (increasing generality/complexity):

simple reflex

state-based reflex agents

goal-based agents

utility-based agents

\noindent
Two of these are reflexive, the next two are planning-based.



\end{document}


