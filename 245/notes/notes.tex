\documentclass[12pt]{article}

\usepackage{mathptmx}
\usepackage{mathpazo}

\usepackage{amsfonts}
\usepackage{amsmath}
\usepackage{amssymb}
\usepackage[version-1-compatibility]{siunitx}
\usepackage{fixltx2e}
\usepackage{multirow}
\usepackage{dsfont}

\topmargin -0.5in %topmargin=1in+\topmargin
\textheight 9in % default is letter % so 11-2=9in
\textwidth 6.5in % and 8.5-2in=6.5in
\oddsidemargin 0in %left margin = % 1in + oddsidemargin
\footskip 1cm % pagenumber to text

\setcounter{secnumdepth}{0}

\title{Math 245A}
\date{\normalsize Fall 2015}


\begin{document}
\maketitle

\section{Chapter 2}

\subsection{2.2 Groups}

A group G is a 4-tuple $G = (|G|, \mu, \iota, e)$ with

underlying set $|G|$

law of composition $\mu$

inverse function $\iota$

neutral element e

\noindent
(Exercise 2.2:1) A homomorphism from a group G to a group H is a function $\phi : G \to H$ satisfying the following for $a, b \in G$:

$\phi(e_G) = e_H$

$\phi(\iota_G(a)) = \iota_H(\phi(a))$

$\phi(\mu_G(a, b)) = \mu_H(\phi(a), \phi(b))$

\noindent
A more common representation of a group uses symbols $G = (|G|, \cdot, ^{-1}, e)$

\noindent
(2.2.1) The conditions for a 4-tuple to be a group are as follows

$(\forall x, y , z \in |G|)\;(x \cdot y) \cdot z =  x \cdot (y \cdot z)$

$(\forall x \in |G|)\;e \cdot x = x = x \cdot e$

$(\forall x \in |G|)\;x^{-1} \cdot x = e = x \cdot x^{-1}$

\noindent
(2.2.2) We may also say that a set $|G|$ with a map $|G| \times |G| \to |G|$ constitudes a group if

$(\forall x, y , z \in |G|)\;(x \cdot y) \cdot z =  x \cdot (y \cdot z)$

there exists $e \in |G|$ such that $(\forall x) e \cdot x = x = x \cdot e$ and $(\forall x \in |G|)(\exists y \in |G|) y \cdot x = e = x \cdot y$

\noindent
(2.2.1) consists of identities (universally quantified equations) and (2.2.2) does not

note: universal quantification is a "for all" quantification

\noindent
(Exercise 2.2:2)

(i)

(ii)

\noindent
(Exercise 2.2:3)

\noindent

\subsection{2.3 Indexed Sets}

An \textit{I-tuple} of elements of X, $(x_i)_{i \in I}$ is formally defined as an $f: I \to X$

The set of all functions from I to X is denoted $X^I$

\subsection{2.4 Arity}

\noindent
The \textit{arity} of an operation is, e.g., 1 if unary, 2 if binary, etc.

An I-ary operation on S is a map $S^I \to S$

Group: a set, a binary operation, a unary operation, and a distinguished element

\noindent
Can think of the identity as a 0-ary/zeroary operation of the structure

$S^0$ has exactly one map, $\emptyset \to S$, so a map $S^0 \to S$ is determined by one element

Note these are not strictly identical since one is a map and the other the element itself

But they are in 1-to-1 correspondence and give equivalent information

\subsection{2.5 Group-theoretic terms}

\noindent
A \textit{group-theoretic relation} in $(\eta_i)_I$ is an equation $p(\eta_i) = q(\eta_i)$ holding in G

p and q are are \textit{group-theoretic terms} which we formally define

\noindent
The terms in the elements of X under the formal group operations $\mu, \iota, e$ form a set T:

given with functions $symb_T: X \to T$, $\mu_T: T^2 \to T$, $\iota_T: T \to T$, and $e_T: T^0 \to T$

such that each map is one-to-one, its images disjoint, and T is the union of those images

and T is generated by $symb_T(X)$ under the aforementioned operations

that is, T has no proper subset containing $symb_T(X)$ and closed under those operations

\noindent
We can represent these terms, for groups, using strings of symbols

We need full parentheses notating order of operations to ensure disjoint images

\noindent
A set-theoretic approach dispenses with strings and allows for infinite arities

\noindent
For the example of a group, we would have (using ordered pair, 3-tuple, etc.):

for $x \in X$, $symb_T(x) := (*, x)$

for $s, t \in T$, $\mu_T(s, t) := (\cdot, s, t)$

for $s \in T$, $\iota_T(s) := (^{-1}, s)$

and $e_T = (e)$

and by set theory, no element can be written as such an n-tuple in more than one way

\subsection{2.6 Evaluation}

\noindent
Given a set map $f: X \to |G|$ for a group G

\noindent
Recursive evaluation of $s_f \in |G|$ given an X-tuple of symbols $s \in T = T_{X, \cdot, -1, e}$

if $s= symb_T(x)$ for some $x \in X$, then $s_f := f(x)$

$s = \mu_T(t, u) \to s_f = \mu_G(t_f, u_f)$, assuming that given $t, u \in T$ we know $t_f, u_f \in |G|$

similarly, $s = \iota_T(t) \to s_f = \iota_G(t_f)$, assuming we know $t_f$ given t

finally $s = e_T \to s_f = e_G$

\noindent
Varying $f$ in addition to $T$ gives an evaluation map $(T_{X, \cdot, ^{-1}, e}) \times |G|^X \to |G|$

\noindent
Alternatively, fixing $s \in T$ gives a map $s_G: |G|^X \to |G|$

these represent substitution into s

these $s_G$ are the \textit{derived n-ary operations} (aka \textit{term operations}) of G

distinct terms can induce the same derived operation

e.g. $(x \cdot y) \cdot z = x \cdot (y \cdot z)$ in general or others for certain groups

\noindent
Examples of derived operations on groups

conjugation $\xi^{\eta}\ = \eta^{-1}\xi\eta$ (binary)

commutator $[\xi, \eta] = \xi^{-1}\eta^{-1}\xi\eta$ (binary)

squaring (unary)

$\delta$ (Exercise 2.2:2)

$\sigma$ (Exercise 2.2:3)

\subsection{First Class Question}

The last example of a derived operation on groups cited the trivial ``second component'' function, $p_{3, 2}(\xi, \eta, \zeta) = \eta$ induced by $y \in T_{\{x, y, z\}, ^{-1}, \cdot, e}$.  I wasn't entirely sure how this derived operation would be represented as an element of $T_{\{x, y, z\}, ^{-1}, \cdot, e}$.  Would $p_{3, 2}$ be the element $(*, y)$ (in the set-theoretic notation)?

\subsection{Terms in other families of operations}

\noindent
An \textit{$\Omega$-algebra} is a system $A = (|A|, (\alpha_A)_{\alpha \in |\Omega|})$

here $|A|$ is some set, and for each $\alpha \in |\Omega|$, $\alpha_A : |A|^{ari(\alpha)} \to |A|$

note that often people will use $n(\alpha)$ (rather than $ari(\alpha)$) for the arity of an operation $\alpha$

e.g. for a group, $|\Omega| = \{\mu, \iota, e\}$, $ari(\mu) = 2$, $ari(\iota) = 1$, and $ari(e) = 0$

\section{Lecture 8/28}

\subsection{Operations, terms, algebra}

revisit the difficulty with distinguished elements as a zeroary operation when dealing with the empty set

the idea behind distinguishing between terms is e.g. to have distinct objects that we can compare

$(x \cdot y) \cdot z \neq x \cdot (y \cdot z)$ as terms, allowing

$(x \cdot y) \cdot z = x \cdot (y \cdot z)$ to be a useful statement about groups

\noindent
set-theoretic approach, infinite arity

$(\mu, s, t)$

$(\mu, (s, t))$

$\alpha_T: T^X \to T$ using $(\alpha, (S_X)_{x \in X})$

$X$ here shall be some cardinal

\subsection{Next reading: free groups}

\noindent
$x, y, z \in G$ and $\xi, \eta, \zeta \in H$

when can we have a homomorphism $G \to H$

if and only if the relations that hold in G hold in H for the corresponding elements

\subsection{Exercises in today's reading}

\noindent
2.7:3

can't have $s( , , , , ,...) = s'( , , , , ...) = s''( , , , , ...)$ where the s'' term is the same as the s term

\noindent
2.2:2 and 2.2:3

$\delta_G(x, y) = xy^{-1}$ and $\sigma_G(x, y) = xy^{-1}x$

$G = \mathds{Z}$ knowledge of the identity

$x *+ y = (x - 1) + (y - 1) + 1$


\end{document}


