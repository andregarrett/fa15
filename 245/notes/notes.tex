\documentclass[12pt]{article}

\usepackage{mathptmx}
\usepackage{mathpazo}

\usepackage{amsfonts}
\usepackage{amsmath}
\usepackage{amssymb}
\usepackage[version-1-compatibility]{siunitx}
\usepackage{fixltx2e}
\usepackage{multirow}
\usepackage{dsfont}

\topmargin -0.5in %topmargin=1in+\topmargin
\textheight 9in % default is letter % so 11-2=9in
\textwidth 6.5in % and 8.5-2in=6.5in
\oddsidemargin 0in %left margin = % 1in + oddsidemargin
\footskip 1cm % pagenumber to text

\setcounter{secnumdepth}{0}

\title{Math 245A}
\date{\normalsize Fall 2015}


\begin{document}
\maketitle

\section{Chapter 2}

\subsection{2.2 Groups}

A group G is a 4-tuple $G = (|G|, \mu, \iota, e)$ with

underlying set $|G|$

law of composition $\mu$

inverse function $\iota$

neutral element e

\noindent
A more common representation of a group uses symbols $G = (|G|, \cdot, ^{-1}, e)$

\noindent
(2.2.1) The conditions for a 4-tuple to be a group are as follows

$(\forall x, y , z \in |G|)\;(x \cdot y) \cdot z =  x \cdot (y \cdot z)$

$(\forall x \in |G|)\;e \cdot x = x = x \cdot e$

$(\forall x \in |G|)\;x^{-1} \cdot x = e = x \cdot x^{-1}$

\noindent
(2.2.2) We may also say that a set $|G|$ with a map $|G| \times |G| \to |G|$ constitudes a group if

$(\forall x, y , z \in |G|)\;(x \cdot y) \cdot z =  x \cdot (y \cdot z)$

there exists $e \in |G|$ such that $(\forall x) e \cdot x = x = x \cdot e$ and $(\forall x \in |G|)(\exists y \in |G|) y \cdot x = e = x \cdot y$

\noindent
(2.2.1) consists of identities (universally quantified equations) and (2.2.2) does not

note: universal quantification is a "for all" quantification

\noindent
(Exercise 2.2:2)

(i)

(ii)

\noindent
(Exercise 2.2:3)

\noindent

\subsection{2.3 Indexed Sets}

An \textit{I-tuple} of elements of X, $(x_i)_{i \in I}$ is formally defined as an $f: I \to X$

The set of all functions from I to X is denoted $X^I$

\subsection{2.4 Arity}

\noindent
The \textit{arity} of an operation is, e.g., 1 if unary, 2 if binary, etc.

An I-ary operation on S is a map $S^I \to S$

Group: a set, a binary operation, a unary operation, and a distinguished element

\noindent
Can think of the identity as a 0-ary/zeroary operation of the structure

$S^0$ has exactly one map, $\emptyset \to S$, so a map $S^0 \to S$ is determined by one element

Note these are not strictly identical since one is a map and the other the element itself

But they are in 1-to-1 correspondence and give equivalent information

\subsection{2.5 Group-theoretic terms}

\noindent
A \textit{group-theoretic relation} in $(\eta_i)_I$ is an equation $p(\eta_i) = q(\eta_i)$ holding in G

p and q are are \textit{group-theoretic terms} which we formally define

\noindent
The terms in the elements of X under the formal group operations $\mu, \iota, e$ form a set T:

given with functions $symb_T: X \to T$, $\mu_T: T^2 \to T$, $\iota_T: T \to T$, and $e_T: T^0 \to T$

such that each map is one-to-one, its images disjoint, and T is the union of those images

and T is generated by $symb_T(X)$ under the aforementioned operations

that is, T has no proper subset containing $symb_T(X)$ and closed under those operations

\noindent
We can represent these terms, for groups, using strings of symbols

We need full parentheses notating order of operations to ensure disjoint images

\noindent
A set-theoretic approach dispenses with strings and allows for infinite arities

\noindent
For the example of a group, we would have (using ordered pair, 3-tuple, etc.):

for $x \in X$, $symb_T(x) := (*, x)$

for $s, t \in T$, $\mu_T(s, t) := (\cdot, s, t)$

for $s \in T$, $\iota_T(s) := (^{-1}, s)$

and $e_T = (e)$

and by set theory, no element can be written as such an n-tuple in more than one way

\subsection{2.6 Evaluation}

\noindent
Given a set map $f: X \to |G|$ for a group G

\noindent
Recursive evaluation of $s_f \in |G|$ given an X-tuple of symbols $s \in T = T_{X, \cdot, -1, e}$

if $s= symb_T(x)$ for some $x \in X$, then $s_f := f(x)$

$s = \mu_T(t, u) \to s_f = \mu_G(t_f, u_f)$, assuming that given $t, u \in T$ we know $t_f, u_f \in |G|$

similarly, $s = \iota_T(t) \to s_f = \iota_G(t_f)$, assuming we know $t_f$ given t

finally $s = e_T \to s_f = e_G$

\noindent
Varying $f$ in addition to $T$ gives an evaluation map $(T_{X, \cdot, ^{-1}, e}) \times |G|^X \to |G|$

\noindent
Alternatively, fixing $s \in T$ gives a map $s_G: |G|^X \to |G|$

these represent substitution into s

these $s_G$ are the \textit{derived n-ary operations} (aka \textit{term operations}) of G

distinct terms can induce the same derived operation

e.g. $(x \cdot y) \cdot z = x \cdot (y \cdot z)$ in general or others for certain groups

\noindent
Examples of derived operations on groups

conjugation $\xi^{\eta}\ = \eta^{-1}\xi\eta$ (binary)

commutator $[\xi, \eta] = \xi^{-1}\eta^{-1}\xi\eta$ (binary)

squaring (unary)

$\delta$ (Exercise 2.2:2)

$\sigma$ (Exercise 2.2:3)

\subsection{Class Question \#1}

end of Section 2.6:
Unimportant

The last example of a derived operation on groups cited the trivial ``second component'' function, $p_{3, 2}(\xi, \eta, \zeta) = \eta$ induced by $y \in T_{\{x, y, z\}, ^{-1}, \cdot, e}$.  I wasn't entirely sure how this derived operation would be represented as an element of $T_{\{x, y, z\}, ^{-1}, \cdot, e}$.  Would $p_{3, 2}$ be the element $(*, y)$ (in the set-theoretic notation)?

\subsection{Terms in other families of operations}

\noindent
An \textit{$\Omega$-algebra} is a system $A = (|A|, (\alpha_A)_{\alpha \in |\Omega|})$

here $|A|$ is some set, and for each $\alpha \in |\Omega|$, $\alpha_A : |A|^{ari(\alpha)} \to |A|$

note that often people will use $n(\alpha)$ (rather than $ari(\alpha)$) for the arity of an operation $\alpha$

e.g. for a group, $|\Omega| = \{\mu, \iota, e\}$, $ari(\mu) = 2$, $ari(\iota) = 1$, and $ari(e) = 0$

\section{Lecture 8/28}

\subsection{Operations, terms, algebra}

revisit the difficulty with distinguished elements as a zeroary operation when dealing with the empty set

the idea behind distinguishing between terms is e.g. to have distinct objects that we can compare

$(x \cdot y) \cdot z \neq x \cdot (y \cdot z)$ as terms, allowing

$(x \cdot y) \cdot z = x \cdot (y \cdot z)$ to be a useful statement about groups

\noindent
set-theoretic approach, infinite arity

$(\mu, s, t)$

$(\mu, (s, t))$

$\alpha_T: T^X \to T$ using $(\alpha, (S_X)_{x \in X})$

$X$ here shall be some cardinal

\subsection{Next reading: free groups}

\noindent
$x, y, z \in G$ and $\xi, \eta, \zeta \in H$

when can we have a homomorphism $G \to H$

if and only if the relations that hold in G hold in H for the corresponding elements

\subsection{Exercises in today's reading}

\noindent
2.7:3

can't have $s( , , , , ,...) = s'( , , , , ...) = s''( , , , , ...)$ where the s'' term is the same as the s term

\noindent
2.2:2 and 2.2:3

$\delta_G(x, y) = xy^{-1}$ and $\sigma_G(x, y) = xy^{-1}x$

$G = \mathds{Z}$ knowledge of the identity

$x *+ y = (x - 1) + (y - 1) + 1$

\section{Chapter 3}

\subsection{3.1 Motivation}



\subsection{3.2 The logician's approach: construction from terms}

\subsection{3.3 Free groups as subgroups of big enough direct products}

\subsection{3.4 The classical construction: groups of words}


\subsection{Class Question \#2}

near 3.3.1
Important

The question concerns the set of all groups G (I'll call it X) whose underlying sets $|G|$ are subsets of S, some countably infinite set.  I wanted to clarify for myself why for any countable group H we can find an isomorphism from one of these groups to H.  Is it sufficient to justify the statement by declaring that X contains all countable groups up to isomorphism and hence for some $G' \in X$, $G'$ is isomorphic to H?  For some reason this feels like incomplete justification to me, and there may be some set-theoretic considerations that may need to be explicated more clearly.

\section{Lecture 8/31}

\subsection{Free Groups: the motivation}

\noindent
factor-set: given a set and an equivalence relation, the set of equivalence classes

\subsection{... as subgroups of big products}

\noindent
If G generated by an X-tuple of elements then has cardinality $\leq max(card(X), \aleph_0)$


\section{Chapter 4.1-4.5}

\subsection{4.1 The subgroup and normal subgroup of G generated by $S \subset |G|$}


$\langle S \rangle$ contains S and is contained in every subgroup which contains S

$\forall x \in \langle S \rangle$ $x = e$ or $x = \prod s_i$, $s_i \in S$ or $s_i^{-1} \in S$

$\langle S \rangle$ is also the image of the map into G of the free group F on S induced by the inclusion map $S \to |G|$

There is additionally a least \textit{normal} subgroup of G containing S.

\subsection{4.2 Imposing relations on a group. Quotient groups}

Quotient groups: homomorphisms causing certain elements to fall together.

Satisfies e.g. $(\forall i \in I) f(x_i) = f(y_i) \leftrightarrow (\forall i \in I) f(x_iy_i^{-1}) = e$

A set of elements annihilated by a group homomorphism form a normal subgroup.

Leads to $q: G \to G\slash N$, where $N$ is this normal subgroup.

We have a quotient map and a quotient group.

\noindent
This map has the universal property desired:

For every homomorphism $h: G \to K$ satisfying the above, $\exists ! g : N \to K$, s.t. $h = g \circ q$.

\noindent
This construction \textit{imposes the relations} $x_i = y_i (i \in I)$ on G, forming $G \slash (x_i = y_i | i \in I)$.

\noindent
For G a group, a G-set is a pair $S = (|S|, m)$, $|S|$ a set and $m: |G| \times |S| \to |S|$, satisfying

$(\forall s \in |S|, g, g' \in |G|)$ $g(g's) = (gg')s$

$(\forall s \in |S|)$ $es = s$

That is, a set on which G acts by permutations.

\noindent
A homomorphism $S \to S'$ of G-sets (for G fixed) is a map $a: |S| \to |S'|$ satisfying

$(\forall s \in |S|, g \in |G|)$ $a(gs) = g a (s)$

\noindent
The set of left cosets of H in G is $|G \slash H|$ and a typical coset $[g] = gH$.

Then $|G\slash H|$ is the underlying set of a left G-set $G\slash H$ by $g[g'] = [gg']$

\subsection{4.3 Groups presented by generators and relations}

Let X be a set, T the set of all group-theoretic terms in X, and $R \subset T \times T$.

$\exists$ a universal example of a group with X-tuples of elements satisfying the relations R.

\noindent
That is, there is a pair $(G, u)$ with G a group and $u: X \to |G|$ satisfying:

$(\forall (s, t) \in R)$ $s_u = t_u$

such that for any group H and X-tuple v of elements in H satisfying $$(\forall (s, t) \in R) \; s_u = t_u$$

$\exists ! f: G \to H$ a homomorphism, such that $v = fu$

The pair $(G, u)$ is determined up to canonical isomorphism by these properties

The group G is generated by u(X)

\noindent
Proving the existence of such a universal construction.

Two ways: construction from terms and subgroups of direct products.

In these approaches, apply the additional conditions to the group axioms.

\noindent
A proof that builds upon prior constructions:

Let $(F, u_F)$ be the free group on X and N be a normal subgroup.

Define N such that it is generated by $\{s_{u_F}t_{u_F}^{-1} | (s, t) \in R\}$

Take a canonical map $q: F \to F \slash N$.

Then the pair $(F\slash N, q \circ u_F)$ has the desired universal properties.

As a consequence of the universal properties of free and quotient groups.

\subsection{Class Question \#3}

3.3 proof of a universal group satisfying relations
Important

I am unsure about the maps in the diagram describing the construction of the pair $(F\slash N, qu_F)$ using the free group on X and an appropriate quotient group.  The map $q: F \to G$ is a map between groups as indicated in the right-sided diagram, whereas it is composed to form a set map $u : X \to |G|$ as in the left-hand diagram.  What sort of distinctions between q as set-map and q as group-map do I need to consider here?  I feel that I might not be clearly expressing the source of my confusion, so I apologize for that, but maybe my question betrays some fundamental misunderstanding about the nature of free groups or quotient groups to be cleared up.

P.S. I do not yet see how to show X contains all countable groups up to isomorphism, but hope to spend some more time thinking about it.


\subsection{4.4 Abelian groups, free abelian groups, and abelianizations}

\subsection{4.5 The Burnside Problem}

\section{Chapter 4.6-4.9}

\subsection{Class Question \#4}

3.6.5 Proposition, use of the van der Waerden stratagem
Unimportant

What are the fundamental characteristics of groups of permutations on a set such as A that make them so useful for proofs that certain sets of reduced forms are composed of distinct elements?

\section{9/4}



\end{document}


