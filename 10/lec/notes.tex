\documentclass[12pt]{article}

\usepackage{mathptmx} % Times font

\usepackage{amsfonts}
\usepackage{amsmath}
\usepackage[version-1-compatibility]{siunitx}
\usepackage{fixltx2e}
\usepackage{multirow}

\topmargin -0.5in %topmargin=1in+\topmargin
\textheight 9in % default is letter % so 11-2=9in
\textwidth 6.5in % and 8.5-2in=6.5in
\oddsidemargin 0in %left margin = % 1in + oddsidemargin
\footskip 1cm % pagenumber to text

\setcounter{secnumdepth}{0}

\begin{document}

\noindent
Near Eastern Studies 10 Lecture, Fall 2015

\section{8/28}

\noindent
nomads

Bedouin: Arab word which refers specifically to camel/desert nomads

marginalized, small percentage of the population

\noindent
sedentarization: movement away from nomadic life, most often involuntary

governments prefer sedentarism to nomadism

tribe loyalty, general disregard for boundaries

several governments forcibly settle nomads down (Jordan, Syria, Iran, etc.)

last half-century: sharp decline in nomadism

\noindent
city: an area that moved away from agriculture

``parasite of the village'' (reliant for food)

dominate politics and culture

\noindent
irregular housing: squatter settlements ringing the outside of a big city

often called bidonville (French: metal can, tin ... house-building material)

essentially synonymous with the word ``slum''

generally integrated into the city

``cities don't make people poor, cities attract poor people'' $\to$ opportunity

driving urbanization: lower infant mortality assoc. with high birth rates

people in the city though have lower birth rates

\noindent
large number of young people and also a great difficulty with unemployment

difficult to measure, though: informal/underground economy significant

governments lie about everything, you never know what to believe

figures from the Middle East are rarely verifiable

\noindent
there is also ``underemployment''

referring to a large number of people working but with nothing to do

especially prevalent in civil/govt orgs: buying loyalty with sinecures

but also evident elsewhere; man paid to turn on the faucet in the hotel

\noindent
privatization trending upward in the Middle East

most countries (esp. Syria, Algeria) historically feature govt. intervention in industry

example Egypt privatizing wine industry

initially, unemployment spikes with the shift

additionally, crony capitalism prevalent (industries to those in power at dirt-cheap prices)

expectation gap; ability to observe those who are more affluent than oneself

\newpage
\section{8/31}

\textit{Thanks Lakshay}

\noindent
Resources in the Middle East\\

General terms, under resourced and not evenly spread out. Main resource is oil, unrenewable so will run out and what happens then. No one knows when the oil will run out. Technology has improved so they are able to extract oil they wouldn't be able to before. US might become an exporter due to fracking, good for short term! 

Second natural resource are phosphates - Morocco is the largest producer. Morocco and Tunisia get income from tourism.

Theres an obvious disparity between each country in the ME, Qatar has an average income of 88000 and yemen only has 1000. The wealth in even the rich countries is distributed poorly. To sustain democracy you need a stable middle class which is often missing in these countries. 

Most important resource is water. Agriculture uses up a lot of water. Most countries are semi-arid, so they don't get any rain. Aquifers are rocks which store water underground, this is Saudis main source of water. Some aquifers are recharged every year by rain some times they aren't and this is called fossil water since it's unrenewable. Turkey is building dams to trap water and use it for hydroelectricity and farming etc. However this stops water getting to Syria. Turkey and Syria have dam systems which would working together cause Iraq to lose 80\% of its water. River Nile is life blood of Egypt. Very worried about the fact that it runs through Ethopia and so they'll feel the knock on affect. Desalinization is another option. Heat up the water, and then condense the water.

Prime agricultural land is now being turned into housing. Egypt used to be an exporter but now is an importer. North Africa used to be the granary of the Roman Empire now it has to export its wheat. 

Some countries practice remittance economies. They send money back home to their families. 3 million Egyptians work outside their home country. The gulf countries have a small populations so need more workers and Germany after WW2 (same with Britain and France). Immigrants don't have skill set and can't speak the language, they aren't treated well so they turn to drugs and petty crime so not good for economy. Very hard to get a job because of discrimination/alienation. In many places workers can't get citizenship, in Kuwait 80\% are immigrant workers. 

Spark for revolution in Tunisia was Dec 17, 2010. Local street vendor sets himself on fire in protest against unfair system set up there. Needed the system to be overthrown. Driven by anger and humiliation. It isn't religious ideology - positive when overthrowing dictator and crazy when it comes to ISIS etc. 

\newpage
\section{9/2}

Language:

a component of ethnicity; binding

they change

they are related to each other by common ancestry

\noindent
Proto-Indoeuropean

English Latin Russian Persian Lithuanian

Original location unknown, around 6000 B.C.

Writing begins around 3000 B.C.

No hard sources, but a construct of linguists

\noindent
English most closely related to Frisian, (spoken in regions of Netherlands)

Lesson: to be a language you don't need to have a country

\noindent
Some language families (number of: 20-40)

Semitic: Arabic, Hebrew, Aramaic, Akkadian

Altaic: Turkish, Mongolian

Sino-Tibetan: Tibetan, Chinese (several languages)

Indo-European: Persian, Pashto (Afghanistan)

Aramaic (Syria, Iraq, Iran) (Jesus)

60\% of people involved in NES 10 speak Arabic

\noindent
Decline in the number of languages

Key cause: urbanization

\noindent
Some languages appear unrelated to even Proto-IndoEuropean

e.g. Basque (Spanish-French mountains)

\noindent
Monogenesis of language theory (single common ancestor)

Not demonstrable

\noindent
Use of the same system of writing does not necessarily indicate related languages

\noindent
Very different forms of Arabic (African, Iraqi variaties, etc. very disparate)

The language stigmatized

\noindent
Some fear of the co-culturization

Convergence American-centric

\section{9/4}

\textit{thanks to Lakshay}

Judaism - practiced in mainly Israel and a little bit in iran, turkey. 
Christianity - practiced in egypt (10\%), 10\% of syria, 40\% in lebanon. Very few in other countries, but due to immigrant workers there are some christians. 
Islam - 90\% of the middle east is islamic. Split into sunni and shii but they are all muslims so the core practices are the same. Some of the differences: sunni's have different ideas about power and how to attain to it. 90\% of sunni, but shii are concentrated in some areas like iran, iraq, lebanon, saudi arabia (10\%). "They live where the oil is." Some interpretations have changed so much can we even call them shii, or even muslim? Example Druze: 3 days of ramadan (too excessive to do it for a month), don't need to visit mecca since god can be in your heart, use wine in there ceremonies. You can find Druze's mostly in Lebanon, so they manipulate lebanese politics. In israel they are even part of the army. Alawi's is another example of a changed sect, mostly find them in syria (two of the past presidents are from this group and the royal family is too causes tension as some people think they have shady islamic practices).  Yazidis are devil worshippers according to ISIS so they have been killing them, god is abstract in their religion and so doesn't do much for the planet, but the devil is his regent and deals with stuff on the ground. 

Ethnicities define by their cultural pasts. Maybe through their language, religion, common origin. Largest ethnic group in the ME are Arabs (60\%), we define arabs through self selection and mostly through linguistics, so through the language. However some people know the language, but aren't arab. 

Jew is another group, defined maybe through the practice of judaism, civilization and culture for some people. Hebrew is the national language of Israel and it is what the religious texts are written in. Yiddish is indo-european language with letters from hebrew (dying out though).

Persians are mostly found in Iranian, they make up about 60\% of the population. 

Pushto people live in afghanistan, dominant group in the nation. 

Kurds don't have a country of their own, no nation state, they live in iraq, iran and syria etc. Could be defined by the fact that they speak the language, but they do speak the language of the country they reside in.

Assyrians live mostly in Iraq they speak a language derived from aramaic, and speak languages from their own country. They are mostly christians. 

Turks are found in turkey, and speak turkish. They started out where mongolia is so there was a large migration, however along they way some took up roots in other places so since they kept their roots you can find turks in china etc. They are mostly sunni's and sometimes alevi's. 

Berbers live in north africa, and mostly in morocco. Arabs and berbers intermarry in north africa so a lot of people have berber blood. 6 variations of berbers spoken across multiple countries. There are still some nomads, living on the fringe of the sahara, in mountains and some are moving into the city. Growing sense that all berbers are forming one people unlike before. Name might come from latin so people don't like it. Amazigh is another term that the locals prefer. 

Armenians live in armenia, they are also christians. They also live in Syria and Iran. 

Local identity sticks in your head the most. Member of the nation state also. Now since most people live in nation states unlike 40-50 years ago they can define themselves in this way. Panarabism is the belief that all arabs are one big country and these splits were created by british and french. 

\section{9/9}

\textit{thanks to Lakshay}

90 million people in egypt, capital has 17 million people which is the biggest city in the ME. Nobel prize for literature first arab was from eygpt. Has the oldest university in the world. 90\% arab speaking muslims, and 10\% arab speaking christians (coptic church/cops - own pope, own services, own holidays). 

Strong feelings - part of the middle east but has its own history. Egypt is the gift of the Nile. Flows south to north, longest river in the world. Upper egypt is below Cairo and vice versa for lower egypt. The nile splits and becomes a delta. Best agricultural land in Africa.

Attacks on christians by extremist muslims. Many christians have left egypt. Modern history starts at 1882, broke away from the ottoman empire by the british. Essentially becomes a british colony. Made monarchy in egypt, only had 2 kings. In 1936 16 year old Farouk comes to the throne, a lot of hope about him, but he failed. Deposed in 1952, lots of silk clothes, walking sticks and pornography. Nasser who deposed farouk (exiled him didn't kill him) led egypt to independence, part of a group called free officers.  States egyptians are arabs, and articulates idea of arab unity. Took land from rich and gave it to the poor. Fixed the prices of commodities and food, government subsidizes bread. Took control of economy and nationalized a lot of stuff. Nationalisation causes lots of problems: over employment, too much beaucracy, unproductive etc. Non aligned movement steer a course between the US + allies and Russia + allies. Both countries had a target list where targets were ranked. He was a gifted public speaker, cared about his people. Lived and died (of a heart attack) in the same house. He had no vices according to the CIA so there was nothing they could blackmail him with, too clean as a leader. Followed by Anwar Sadat, signed Camp David accord and made peace with Israel. Angered a lot of arabs, since he sold out the arab cause but put egypt first. Became more autocratic, and then assassinated in 1981. Followed by Hosni Mabarak. He was a dictator, lackey of the US, and made people very unhappy.  

Very few resources in egypt, and a lot of the money flowing into egypt is remittances. Tourism is another factor but that waxes and wanes depending on the situation. Foreign aid from US, 2nd most amount, spent a lot on pointless military expenses so the money comes back to the US. Corruption removes some of the money too. 

In June 2010 policeman beats up man for witnessing a drug deal. January 2011, there was a call for a demonstration in Tahrir square. Thugs tried to break it up, killed 100s of people. After 2 weeks Mubarak steps down and retires. Public was very naive. June 2011, first free elections in 5000 years , won by Morsi. Part of the muslim brotherhood which was illegal, but they provided social services which the government didn't at the time. Starts to lose its credibility, and so is ousted by the military who is now in control el-Sisi. They wiped out camps where people were staying, bloodiest day in modern egypt. 

\section{9/11}

missing

\newpage
\section{9/14}

Iraq is very much a new state; formerly under the Ottoman Empire\\

Three administrative units, north (Mosul), middle (Baghdad), south (Basra)

Ottoman collapse $\to$ control of the British

British combined the three provinces and joined together, named Iraq

British created a very inorganic state, provinces didn't join well

Mosul mostly Kurdish, around Baghdad mostly Sunni, around Basra mostly Shiite

Shiites majority, but British created a Sunni-dominated state

Why? Sunnis near Baghdad, and Ottomans had be Sunnis

1946 665,000 Baghdad almost all Sunni Arab

1964 2 million Arab many more Shiites come in

Good agriculture, good oil source

'Independent' in '32: next period, many coups, untrackable, Saddam Hussein 1970-2006

perhaps the most coercive police state in the Middle East

1980-1988 Iraq-Iran War (much like WWI trench warfare)

pointless war; can't understand motivation, all based on Saddam Hussein's dictate

``we Arabs against those Persians''

Iran seems to begin winning, US begins to help Saddam with weapons, aerial reconnoitering

war devastating to the Iraqi economy, net loss of \$100 billion to the surplus, becomes deficit

Saddam then attacks Kuwait (is part of Basra province, so could be considered part of Iraq)

an argument that he was eliminating an artificial boundary by doing so

but was more or less self-interested, to get the oil (Oct 8, 1990)

Jan 16, 1991, US sends in military in Gulf War I, US returns Kuwait to Kuwaiti

but Saddam Hussein remained in power

assumed that US intervened to prop up Saudi Arabia

US invasion prompts a new use of Islam as a tool for his govt, increasingly allied w/Muslims

9/11 $\to$ Gulf War II, WMD, linkage b/t Saddam and Al Qaeda (false)

Al Qaeda composed of Sunni religious fanatics (Saddam a secularist)

Mar 20, 2003, May, Mission Accomplished sign

Problem: what's going to happen after the overthrow

inorganic state with strong fault lines, no democratic govt for 90 years

stirred up anti-American feeling

suppression of the Kurds (e.g. Saddam Hussein, 1988, campaign, for their oil)

Kurdish region more and more separated, safe, as good as independent

But have not demanded out-and-out independence, have just worked w/in system

``We won't break away from Iraq, but they might from us.''

Iraq needs to be rebuilt; but more like built for the first time, from the ground up

ISIS; extreme sects of Islam, extreme even for most Muslims

If going to invade, should have made a much larger long-term commitment

Marsh Arabs from the far south of Iraq, traditional lifestyle

Mostly Shiites took refuge there; some say they are Sumerian descendants

\section{9/16}

Syria: Alawis a minority ethnic group

Leader Hafez Hamah dictator who was Alawi, so have minority group leading

Had alliances with some Sunnis, Christians who bought into the state

Hafez dies, son Bashar takes place, not 40, technically not legal to be pres, change law

was not going to be President; first releases some prisoners

But later turns towards his father's course: imprisoning

Arab Spring in Tunisia, Egypt, and Bashar fights back and is able to maintain control

Thousands of Syrians killed since 2011

Tunisia basically had one corrupt family, pretty easy to throw out

Egypt just had Mubarak and his cronies, relatively easy to throw out

Syria somewhat different, have 10\% of population (Alawis) worried by change

More general support, more people buying into the state

Drought in Syria 2006-2010 which displaced around one million Syrians from country to city

Opposition to regime really started in those areas

Similar to the opposition which started in other areas

Bashar engaged in 'crony capitalism'

Privatizating industries, with control going into the hands of supporters

Lack of our intervention: don't know who the ``good guys'' are.

Danger of materiel being turned against us

Many others are involved: e.g. Iran helping the state considerably

Considers the Alawis to be brother Shiites

Iran a minority Shiite state, always concerned about Shiites

Russians are involved, their main/only Mediterranean port at Tartus in Syria

Is a civil war, dragging on for years, many people killed

``It will end, but it will not end soon''

Bashar was going to be an opthalmologist, appeared to be good

Perhaps a consequence of the danger of power, many cronies latching on to him

What if he had kept on that track of being a 'good guy' that he was on at the beginning

Instead, just becomes another mass murdering dictator

\newpage
\noindent
Lebanon: Beirut capital (Paris of the east)

perceived as western band of arab/muslim world and eastern edge of the christian world

Mountain ranges ring Lebanon, good for minority groups, to be able to settle in those mountains

Very strong ethnic-religious local identities (Sunnis, Jews, Christian Maronites, Shiites)

Economy driven by the port city of Beirut, which supports also Jordan, Syria, Iraq

A 'free-wheeling capitalist' state, many banks

Most democratic state in the Arab middle east; doesn't show up in the Arab Spring

Constitution states that all creeds are respected

Independent in 1943

Maronites: perceive themselves as an ethnicity, particular brand of Christianity

Maronites identify more with the Christian west than the Muslim east

1943: 2 politicians created the national pact, an oral, non-written agreement

Everything is divvied up by ethnicity: Presidency, prime minister, heads of department

Each religious and ethnic group always has a representative in that particular seat

Reinforces these local identities (Druze, Shiite, etc.)

Pact never had a provision for a change in demographics

Christian population dropped and the Muslim Shiites started to grow in proportion 

(poorer, higher birth rate); also some influx from Palestinians and other areas

Also an out-migration of Christians into europe

Really the only Christian state, but not reflecting population

1975-1990 civil war exacerbated by these difficulties

Very jumbled, many localized militia, no real front lines, scattered fighting

Framed as 'Christian-Muslim' war here in the west, but not really an accurate representation

Around 1982, Hezbollah more noticable, Shiite group

Names means ``people of god'', lots of support from Iran

Long-term goal: to establish a Muslim state in Lebanon

Support from people for providing a lot of social services to people that the gov't does not

Support not necessarily entirely religion-based

Similar to the situation in Egypt with the Muslim Brotherhood

Some redistribution of power since 1990; even number of seats in the assembly

Currently maybe about 30\% Christians now, 1932 census was around 55\%

Offices still apportioned by the old national pact

Underlying problem the self-identification by local/religious identity: 'confessional loyalty'

Confessional nature of the state $\to$ fair amount of cronyism

Images: very clear that declaration of religious identity important

(Image of doctors with all of there religious identifications present)

\section{9/18}

Jordan

arbitrarily made by divisions under the Ottoman Empire and the British

gained independence in 1946

kind in 1963-1989 Husain

related to family of Mohamed the Muslim prophet: religious cachet and legitimization

Jordan has phosphates

many Palestinians have ended up in Jordan

probably more Palestinians than native-born Jordanians

Has caused a certain amount of problems

Main body among Palestinians (PLO: Palestine Liberation Organization [overthrow Israel])

PLO would periodically launch attacks on Israel from Jordanian territory

Israeli responses hurt Jordan; worries King Husain, sees PLO as danger

Daily fighting between Palestinians and Jordanians; rift in state

Sept 16 1970: Husain's army attacks Pal. encampments inside Jordan: ``Black September''

Drove the PLO leadership out of Aman and into Lebanon

Husain actually tried to get Israel to help them out; Syria helped PLO, sent tanks

entreated Israel to send aerial attacks upon the Syrian tanks

``only example I can think of of Arab country asking for Israeli help''

Splinter group of PLO Black September attacked Israeli athletes in 1972, Munich Olympics

Jordan contested West bank with Israel for some time

1988 Jordan gave up formal claims became potential ground for Palest. state

Husain dies in 1989, replaced by a man named Abdallah; seems to be a good guy

Jordan is a constitutional monarchy

Arab spring has not hit: Abdallah working to help, solve economic problems

\noindent
Saudi Arabia

also a new state, created in 1932

contains Mecca and Medina (Mohamed born, buried)

has mostly been a backwater, though (no resources)

was never anybody's colony in any form

hundreds and hundreds of tribes, no structure

united into one state by the force of one man, (we call him) Ibn Saud

period of time up to 1932; many people killed in the course of this binding

state named after self: absolute monarchy since then

all power lies in the royal family

``when are you going to give rights to women?'' ``when we give them to men''

a ``democracy of 7000 princes''

many wives of Ibn Saud $\to$ often for political purposes

``united the country in bed''

only 1\% arable water, smallest out of any place in the Middle East

The world's largest country in the world without a river

Oil brings in a ton of money; Saudis spend a lot of money, esp. on military

largely a waste of money: not enough Saudis to man this equipment

army not very efficient

has not sent a single soldier to fight in an encounter with Israel

In the US, we get a lot of technological gains from inventing the stuff, spin-offs

Saudis however just buy the stuff and get little to no benefit

Free education, healthcare, home loans, etc.

Princes who are the direct descendents of Ibn Saud get lots of benefits

Princes get stipends, costly to the govt

Population has grown a fair amount, more people for the same amount of income

perception: saudi arabian $\to$ rich

perhaps half the people in saudi arabia are foreign workers: get no perks

live in slums, treated like indentured servants

come to Saudi Arabian b/c can make money there that they can't at home

99\% Muslim, most are Sunni are some are Shiites(15\%)

We forced them into Islam by the sword (Ibn Saud felt Shiites were apostates)

Did not consider Shiites to actually be Muslim

Have benefited the least from the political structure

Wahhabi interpretation of Islam; purge of folk beliefs, rigid interpretation

e.g. called Puritanical interpretation of Islam

women not allowed to leave w/o guardianship of a man, can't drive

very little opportunity for women to develop themselves

few work, mostly to teach other women or as nurses for other women

anecdotally issues with clinical depression, obesity among Saudi women

Taliban interpretation in Afghanistan is a spin-off of Wahhabism

So is Al Qaeda

This is the interpretation that is most visibly seen in the news

Media focus on exotic interpretations, although women are driving ``next door in Bahrain''

no freedom of religion in the Saudi state

one of the few where practice of any religion other than Islam is banned

In the strictest interpretation of Wahhabist Islam, Shiites are apostates

Saudi princes visiting other nations, gambling and womanizing

Other peoples feel that these princes give Islam a bad name

Gulf War I many American soldiers stationed in Saudi Arabia

Resented the fact that the Saudis with all their money couldn't defend themselves

Curiosity about whether or not liberal progression will happen

Still very repressive to women but also to men in some sense

In 9/11, 15 of the 19 hijackers were from Saudi Arabia; it's where Osama grew up

A question about what it is about the Saudi Arabian system

``Saudi youth have no guide'', aimless prosperity

Have used Saudi money to sort of buy off political opposition

\section{9/21}

\noindent
Maghrib: the countries Morocco, Algeria, Tunisia, Libya

Morocco, Algeria, Tunisia are former French colonies, Libya is a former Italian colony

Arab intermarriage with Berbers $\to$ Middle Eastern identity in Africa

Berbers: a people who self-identify, although scattered across Africa

\noindent
Tunisia gained freedom in 1956, becoming a one-party state with a president fro life

Suppressed population, birth control, abortion common.

Limited military and social spending.

Corruption and the ruling family: business, investment difficult.

Ben Ali deposed the old president.

Fruit vendor self-immolation sparked the first successful revolution in the modern Arab world.

First free elections, first democracy of the Arab Spring

Newsworthy moments: recent terrorist attacks on tourist locations.

\noindent
Morocco was a former Ottoman colony seized by France in 1912; it became independent in 1956.

Formed a monarchy.

Demographics: Berbers and Arabs.

Economy: tourism and phosphates. High unemployment, remittance-driven economy.

Like Morocco:Europe::Mexico:U.S.

European governments fear Moroccan immigrants.

King: Mohammed the 6th; claims to descendency from the prophet Mohammed.

Monarchy has reasonable popular support.

Small 2012 Arab Spring moment mostly affecting the Berbers.

\noindent
Libya also existed under the Ottomans until Italy took it in 1911.

It became independent in 1951.

Defined by the rule of Colonel Muammar Gaddafi

Dictator, interesting philosophies on Islam:

changed Muslim calendar, said Hajj not necessary but still recommended

Economy: oil-driven

Gaddafi: social projects for female education, health, jobs.

Man-made river system from the Nile aquifer which has been moderately successful.

Funded terrorism, assassinations.

Took up the helm of Nasser in the cause of Arab Unity

Eventually concluded that no such unity existed; proposes a United States of Africa

Awful human rights record.  Rebellion, execution in 2011.

No real state at the moment, govt. controls east, rebels/militia control west.

\noindent
Algeria taken by France from the Ottoman Empire in 1830; many French settle there.

Those French called ``colons''.  Algerians had to give up Islam to be a citizen.

This resulted in an 8-year civil war from 1954-1952.

French: the cause of ``defending the west'', 2 million soldiers sent.

Movie 1966: the Battle of Algiers; now a symbol for independence struggles worldwide

New independent government rickety, a one-party state.

Rulers: rich businessmen, military men, Arabs.

"Algeria is rich but algerians are poor".

High unemployment; many immigrants in France.

\section{9/23}

\noindent
Turkey

Istanbul: the bridge between 2 continents, a city of 13 million people.

Most are Turks; relatively new inhabitants of the region.  Melded, formed local identity.

Almost universally Muslim.

1600s control of the Middle East: Ottoman Empire.

A revitalization of the Ottoman area by young Turks to bring it to the 20th century.

Mustafa Kemal Ataturk: the founder.  Destroy the pure dictatorship of the Ottoman Empire.

Replace with a democratic Republic of Turkey; Ataturk first president.

Ataturk dies in 1933.

Kemalism: the movement to westernization and secularization of Turkey.

Ataturk thought that Islam was an impediment to forward progress, into the future.

Ataturk changed the language in 1928 to use English, rather than Arabic script.

He replaced Islamic Law with a Swiss Civil Code, including better women's rights.

He moved the capital to Ankara, minimized involvement with other Middle-Eastern countries.

Ataturk a 1935 title: ``father of the Turks''.

\noindent
Turkey more recently: have been attempting to join the EU, without success.

Argument against: culturally far from Europe, non-Christian.

Villages: somewhat distant from Kemalism, rural areas stronger tied to Islam.

Prime minister 2003 Erdogan: more religious, but also modernist, pragmatic.

Ergodan: privatization, imporved economy.

Poor treatment of Kurds (20\% of the population); 80\% of them are unemployed.

Denial of Kurdish independence; some Kurds carry out attacks against military, civil targets.

Conflict: government, ISIS, Kurds.

\noindent
Afghanistan: modern borders formed in 1907 by Russia and the British.

Mountainous, people fragmented $\to$ strong local identities.

Main ethnic group: Pashto, then Tajiks (25\%), Uzbeks (Turkish language)

1960 links to Soviet Union; Soviets driven out in 1989

In this time million Afghans were killed, a third fled; CIA funded anti-Soviet rebels

1994 Taliban: fundamentalism, strict Islam like the Saudis, killed many Shiites

Taliban cleaned up corruption in government, order appealed to the people, did not rebel.

2001 U.S. invasion of Afghanistan.

Economy: opium, source of 90\% of the heroin in the world, a stable source of income.

\section{10/12}

\noindent
talked a lot last Friday about Zoroastrianism\\

\noindent
religion is one of the defining markers of ethnicity in the Middle East

\noindent
rehash ancient near east and egypt

polytheism

no systematicity, lot of different gods, creation myths

no foundational book out of mesopotamia and egypt: Judaism quite different

\noindent
Judaism: will start mostly on the history here

\noindent
Important components

monotheism

God actually acts in history

the idea of a covenant between a people and their God\\

\noindent
Abraham (guess, 2000 B.C.)

born in an ancient city called Ur, southern Mesopotamia, ancient Sumer

moves with family to Harran (way north of Mesopotamia, south of Turkey)

Ur no longer exists, but Harran does

hears God speaking to him: orders him to leave Harran

first of the three patriarchs (others Isaac son, Jacob grandson)

Abrahamic religion: Christianity, Judaism, Islaam

\noindent
Jacob (grandson or great-grandson of Abraham)

``wrestling match with God''

Jacob name changed to Israel (God strives)

changing of a name not uncommon

descendents begin to be called Israelites

\noindent
Abraham's travel to Israel/Palestine, then called Canaan, inhabited by Canaanites

covenant between God and descendants of Abraham

promised land is between river of Egypt and the Euphrates

defined in different ways in different places in the Scriptures

the one just mentioned is the one of largest dimensions

covenant is a promise of land and of descendants

\noindent
promise ritualized by circumcision

practiced by many peoples, including the ancient Egypt (not Akkadians/Assyrians though)

usually at puberty, different here, done at 8 days of age

viewpoint: symbolizes finality, covenant is binding

becomes a mark of ethnicity, of distinction (sometimes called boundary-marking rituals)

other boundary-marking rituals: eating of certain foods (pork, etc.)

\noindent
maybe around 1300 B.C. there's a famine in the land

some descendants of Abraham in Egypt (Nile fruitful)

oppression and enslavement, work on royal building projects (e.g. military, storehouses)

pyramids had been built for thousands of years before their work

\noindent
Moses and the Israelite Exodus

parting of Red Sea: instantiation of God acting in history

unmentioned in Egyptian sources

could have been further north of Red Sea

miracle in a technical sense, bending the normal rules of physics

Ten Commandments delivered to Moses

God himself on two stone tablets; appear twice in the Scriptures

\noindent
Exodus 20:

numbers added to the text by medieval scholars, not part of the original text

no other Gods before me (requirement of monotheicity not explicit)

parallel with Zoroastrianism: hennotheism (one god above many) then rigid monotheism

no idols, appropriate use of God's name, Sabbath day

honor to parents

prohibit murder, adultery, stealing, false witness against neighbor

shall not covet neighbor's stuff (goes against human nature?)

\noindent
Joshua leads Israelites into battle against Canaanites

described as the conquest; described as a campaign

Canaanites defeated or assimilated into the Israelites

monarchy eventually established, state of Israel (around 1000 B.C.E.)

\section{10/14}

Hebrew scriptures

stories transferred orally for many years before being written

at some point canonized (90 A.D.) by scholars

first five books: Torah, core of the scriptures

\noindent
all we know about Abraham comes from Scripture

historicity of the account: how much is fact, how much is fiction?

what's important? historicity, ideology, moral, etc.

\noindent
the nation-state of Israel

Kings Saul, David, Solomon (970-930)

Solomon featured significantly in all three Abrahamic religions for his wisdom

capital in Jerusalem with a large temple

\noindent
under the Romans

conqured by Pompey

Romans never really understood the religion of the land

Romans participated in ``low-key'' polytheism; didn't care about religion of conquered

however, all had to acknowledge the Roman emperor as semi-divine, even if just pretend

Hebrews among most conquered would not at all show this devotion to emperor

Romans extracted large amounts of tax

Romans built many paved roads for military reasons, helped commerce/travel generally

Sadduces the Jews on the top, permissive towards Romans

below them Pharisees, focused on keeping the traditional law, emphasized scripture study

1 A.D. Messianism: idea that a Messiah, deliverer, will come to lead Jews out of troubles

Roman armies destroy temple in 70 A.D. in response to revolt, another revolt 135 A.D.

after this very few Jews remain in Jerusalem, esp. in south, some still in Galilee

this is the Diaspora (scattering)

council held to maintain and stabilize the scriptures

\noindent
ancient Israelite religion vs. Judaism

a relevant distinction? mainly maintained by Christians

rationale: destruction of the temple changed nature of the religion

initially centered around temple, rituals, sacrifice

changed to an emphasis on piety via study

Jews more likely to see a continuity in history

\noindent
Orthodox $\to$ conservative $\to$ reform $\to$ liberal

U.S. orthodox/conservative generally oppose redefinition of marriage

orthodox alone have women and men seated separately

rabbi: teacher, begin as a title of acclimation, now a process of certification

imam: leads Friday communal prayers, traditionally male, very rarely female (Islam)

\noindent
identification practices

calendar defines identity, Muslims, Christians, Jews, Zoroastrians

food taboos fall into the same sort of boundary-marking ritual

Judaism readily accepts converts; several examples in department, from Christianity

generally Jewish heritage passes through mother (Islam through father)

\noindent
Jews in the middle east prevalent up to fairly recently

but since around 1940s (e.g. Egypt) persecuted, attacked, mostly moved out of Middle East

\noindent
pronunciation of the name of God unsure

Yahweh (w sounding as v), but actual naming somewhat forbidden

so many circumlocutions: our lord

\section{10/16}

\noindent
Jesus Christ the Mashiach (Messiah)

name etymology: Yehoshua $\to$ Joshua $\to$ Jesus

birthplace Bethlehem, lived most of life in Nazareth, born to Joseph/Mary, Jewish carpenter

at 30 years old, John the Baptist (Hebrew: Yochanan ``Yahweh is gracious''), baptism

ruled by Herod the Great of the Romans

John spent much of time in Galilee in the north

Romans concerned with Jesus as a political power; crucifixion (punishment for serious crimes)

died around 30 A.D.

to Christians: miraculous elements in his birth, life, death

Mary virgin mother, healings, Lazarus, resurrection from dead (capital-'M' miracle)

spoke in Aramaic; 4 gospels in Greek

at the beginning, didn't perceive themselves as creating a new religion

parting of the ways: Judaism and Christianity

Saul Jewish name and Paul Greco-Roman Latin name

reasonably common to have two names for different use at different times

persecution of early Christian Jews as separatists; transformation, proselytization

apostle to the Gentiles (non-Jewish); key to growth of Christianity

Paul can be considered the most important figure to the future of Christianity

irreconcilable separation of Jews

\section{10/18}

\noindent
Romans

Violence; road-building; empire.

North Africa: the ``granary of the romans''.

New Rome: Constantinople $\to$ Istanbul (truncation, language development)

Advantages of the location at byzantium: strategic positioning, a crossroads.

Western Empire: Rome, Eastern Empire: Istanbul, the ``Byzantine Empire''

The Eastern Roman Empire controlled most of the Middle East at one point.

These are Greek-speaking Christians.

They contended with the Sasanian Empire, centered around Ctesiphon, near Baghdad.

Christianity: conversions of Zoroastrians in the Sasanian Empire.

Century of wars, Byzantine Empire hoping to support Christians in the Sasanian Empire.

\noindent
Islam

Predated by Jahiliyya, the ``age of ignorance'' (Arabic root jahili).

Word still used: describes societies that move away from Islam.

The two climates of the Arabian peninsula: Arabia Felix (near Yemen) and Arabia Deserta

Arabia Felix: prosperous, sufficiently humid, dams, dykes, rain

Frankincense used to cover the smell of creation, offered to Gods

Arabia Deserta: dry, inhabited by bedouins, nomadic, no states

Loyalty to tribes, oral culture, few monuments, some oral poetry not written down till Islam.

The East is a Poetic society (contrasst west prosaic).

In the Jahiliyya, some religion present, but not considered central to life.

Mecca: one of the few structured cities.

Built around a central well called Zun Zun.  Near a major trade route, gained from tax.

A site of pilgrimage: 360 shrines for Gods, many people can pray there.

Medina (Yathrib); converted and incorporated Jewish tribes from outside of the city.

Christianity made the most inroads in Medina.

\section{Islam, beginnings}

A diversity of opinion; many argue among themselves, differ in religious practices.

Islam: submission (to God), a complete social system and way of life.

\noindent
Mohammed

Born around 570 AD in Mecca, at the time a thriving commercial center.

There were Jews and Christians in Mecca at Mohammed's birth; Christian monks on the peninsula.

There were also the ``hanif'' who had a monotheistic faith about which little is known.

The most sacred place for Muslims, the kaaba (cube)

It's not a temple, it's a ``place for god to reside on earth''.

Life of Mohammed: very little from the Quran, more from later documents.

Were taken down a century or two after his life.

\noindent
Details of the life of Mohammed

Born into the Quraysh (little shark) tribe.

Sheperd $\to$ businessman $\to$ marries buisinesswoman boss $\to$ mountain cave seclusion

This was enabled by the wealth of wife.

610 AD visited by angel Gabriel, recited words from God (Quran)

Alive for 22 more years.  Quran completed after death; not written by him.

Quran: words of God mediated by Gabrial.

The Sura are chapters, they have Ayat (verses), organized by length.

Generally, the larger ayat are in the beginning of each sura.

To read the Quran, native Arabic speakers need classes.

Mohammed preaches these words, but not received well.

622 goes to Yathrib, a place with much fighting, but little influence as an outsider.

This is the Hijra (migration of Mohammed), an important event in the Muslim calendar.

Yathrib later called Medina (the ``city of the prophet'').

Medina is the first Muslim community, and from here, Islam begins to spread.

\section{After Mohammed}

Fatimah youngest daughter of Mohammed, the wife of Ali.

Ali: first male convert, his right-hand mand and pious.

First caliph after Mohammed's death was Abu Bakr, the first convert outside of Mohammed's family

Led prayers, handled tax duties, governmental duties

Some thought Ali should have been the first caliph

Shiat Ali (supporters of) $\to$ Shiites

Others Sunni (from the word sunna, meaning traditional)

Ali is the fourth caliph; 661 AD dies in political turmoil, stabbed by angry Persian.

Two sons: Hassan and Husayn.  Hassan avoids conflict, Husayn goes to battle in Karbala (Iraq).

Husayn and his followers die; a martyr for the struggle.

One of the most celebrated religious stories for the Shiites.

Shiites self-flagellation: want to suffer as Husayn did, fight his battle.

Shiites: should have been hereditary.

Ali is the first Imam (technically: prayer reader, for Shiites, ``first leader'')

Imams are said to have prophetic light, mystical understanding of Quran verses.

Twelvers: believe the 12th imam went into ``occultation'' at 5, still 'alive'.

Will come at the end of days with the title Mahdi.

Ayatollahs: Shiite.

Argument over the lines of Imams: 5er and 7er views.

\noindent
Shiite and Sunni

Lebanon 40\% Shiite, Bahrain 70\% Shiite, Saudis some, Iraq some.

Sunni and Shiites used to be divided by urbanization joined together.

Shiites: often bottom of the economic scale; downtrodden sympathetic to Shiite message

Shiites: taqiya, a taking up of other religion for self-protection

taqiya is a consequence of longstanding suppression

Wahhabis: Shiites are not Muslim in that they think of Ali almost as a God

Alawis: Ali is the closest to God on earth.

\end{document}


