\documentclass[12pt]{article}

\usepackage{mathptmx} % Times font

\usepackage{amsfonts}
\usepackage{amsmath}
\usepackage[version-1-compatibility]{siunitx}
\usepackage{fixltx2e}
\usepackage{multirow}

\topmargin -0.5in %topmargin=1in+\topmargin
\textheight 9in % default is letter % so 11-2=9in
\textwidth 6.5in % and 8.5-2in=6.5in
\oddsidemargin 0in %left margin = % 1in + oddsidemargin
\footskip 1cm % pagenumber to text

\setcounter{secnumdepth}{0}

\title{Near Eastern Studies 10: Lecture}
\author{Andre Garrett}
\date{\normalsize Fall 2015}


\begin{document}
\maketitle

\section{8/28}

\noindent
nomads

Bedouin: Arab word which refers specifically to camel/desert nomads

marginalized, small percentage of the population

\noindent
sedentarization: movement away from nomadic life, most often involuntary

governments prefer sedentarism to nomadism

tribe loyalty, general disregard for boundaries

several governments forcibly settle nomads down (Jordan, Syria, Iran, etc.)

last half-century: sharp decline in nomadism

\noindent
city: an area that moved away from agriculture

``parasite of the village'' (reliant for food)

dominate politics and culture

\noindent
irregular housing: squatter settlements ringing the outside of a big city

often called bidonville (French: metal can, tin ... house-building material)

essentially synonymous with the word ``slum''

generally integrated into the city

``cities don't make people poor, cities attract poor people'' $\to$ opportunity

driving urbanization: lower infant mortality assoc. with high birth rates

people in the city though have lower birth rates

\noindent
large number of young people and also a great difficulty with unemployment

difficult to measure, though: informal/underground economy significant

governments lie about everything, you never know what to believe

figures from the Middle East are rarely verifiable

\noindent
there is also ``underemployment''

referring to a large number of people working but with nothing to do

especially prevalent in civil/govt orgs: buying loyalty with sinecures

but also evident elsewhere; man paid to turn on the faucet in the hotel

\noindent
privatization trending upward in the Middle East

most countries (esp. Syria, Algeria) historically feature govt. intervention in industry

example Egypt privatizing wine industry

initially, unemployment spikes with the shift

additionally, crony capitalism prevalent (industries to those in power at dirt-cheap prices)

expectation gap; ability to observe those who are more affluent than oneself

\section{8/31}

\textit{Many thanks to Lakshay!}

\subsection{Resources in the Middle East}

General terms, under resourced and not evenly spread out. Main resource is oil, unrenewable so will run out and what happens then. No one knows when the oil will run out. Technology has improved so they are able to extract oil they wouldn't be able to before. US might become an exporter due to fracking, good for short term! 

Second natural resource are phosphates - Morocco is the largest producer. Morocco and Tunisia get income from tourism.

Theres an obvious disparity between each country in the ME, Qatar has an average income of 88000 and yemen only has 1000. The wealth in even the rich countries is distributed poorly. To sustain democracy you need a stable middle class which is often missing in these countries. 

Most important resource is water. Agriculture uses up a lot of water. Most countries are semi-arid, so they don't get any rain. Aquifers are rocks which store water underground, this is Saudis main source of water. Some aquifers are recharged every year by rain some times they aren't and this is called fossil water since it's unrenewable. Turkey is building dams to trap water and use it for hydroelectricity and farming etc. However this stops water getting to Syria. Turkey and Syria have damn systems which would working together cause Iraq to lose 80\% of its water. River Nile is life blood of Egypt. Very worried about the fact that it runs through Ethopia and so they'll feel the knock on affect. Desalinization is another option. Heat up the water, and then condense the water.

Prime agricultural land is now being turned into housing. Egypt used to be an exporter but now is an importer. North Africa used to be the granary of the Roman Empire now it has to export its wheat. 

Some countries practice remittance economies. They send money back home to their families. 3 million Egyptians work outside their home country. The gulf countries have a small populations so need more workers and Germany after WW2 (same with Britain and France). Immigrants don't have skill set and can't speak the language, they aren't treated well so they turn to drugs and petty crime so not good for economy. Very hard to get a job because of discrimination/alienation. In many places workers can't get citizenship, in Kuwait 80\% are immigrant workers. 

Spark for revolution in Tunisia was Dec 17, 2010. Local street vendor sets himself on fire in protest against unfair system set up there. Needed the system to be overthrown. Driven by anger and humiliation. It isn't religious ideology - positive when overthrowing dictator and crazy when it comes to ISIS etc. 

\section{9/2}

Language:

a component of ethnicity; binding

they change

they are related to each other by common ancestry

\noindent
Proto-Indoeuropean

English Latin Russian Persian Lithuanian

Original location unknown, around 6000 B.C.

Writing begins around 3000 B.C.

No hard sources, but a construct of linguists

\noindent
English most closely related to Frisian, (spoken in regions of Netherlands)

Lesson: to be a language you don't need to have a country

\noindent
Some language families (number of: 20-40)

Semitic: Arabic, Hebrew, Aramaic, Akkadian

Altaic: Turkish, Mongolian

Sino-Tibetan: Tibetan, Chinese (several languages)

Indo-European: Persian, Pashto (Afghanistan)

Aramaic (Syria, Iraq, Iran) (Jesus)

60\% of people involved in NES 10 speak Arabic

\noindent
Decline in the number of languages

Key cause: urbanization

\noindent
Some languages appear unrelated to even Proto-IndoEuropean

e.g. Basque (Spanish-French mountains)

\noindent
Monogenesis of language theory (single common ancestor)

Not demonstrable

\noindent
Use of the same system of writing does not necessarily indicate related languages

\noindent
Very different forms of Arabic (African, Iraqi variaties, etc. very disparate)

The language stigmatized

\noindent
Some fear of the co-culturization

Convergence American-centric

\end{document}


