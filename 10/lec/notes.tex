\documentclass[12pt]{article}

\usepackage{mathptmx} % Times font

\usepackage{amsfonts}
\usepackage{amsmath}
\usepackage[version-1-compatibility]{siunitx}
\usepackage{fixltx2e}
\usepackage{multirow}

\topmargin -0.5in %topmargin=1in+\topmargin
\textheight 9in % default is letter % so 11-2=9in
\textwidth 6.5in % and 8.5-2in=6.5in
\oddsidemargin 0in %left margin = % 1in + oddsidemargin
\footskip 1cm % pagenumber to text

\setcounter{secnumdepth}{0}

\title{Near Eastern Studies 10: Lecture}
\author{Andre Garrett}
\date{\normalsize Fall 2015}


\begin{document}
\maketitle

\section{8/28}

\noindent
nomads

Bedouin: Arab word which refers specifically to camel/desert nomads

marginalized, small percentage of the population

\noindent
sedentarization: movement away from nomadic life, most often involuntary

governments prefer sedentarism to nomadism

tribe loyalty, general disregard for boundaries

several governments forcibly settle nomads down (Jordan, Syria, Iran, etc.)

last half-century: sharp decline in nomadism

\noindent
city: an area that moved away from agriculture

``parasite of the village'' (reliant for food)

dominate politics and culture

\noindent
irregular housing: squatter settlements ringing the outside of a big city

often called bidonville (French: metal can, tin ... house-building material)

essentially synonymous with the word ``slum''

generally integrated into the city

``cities don't make people poor, cities attract poor people'' $\to$ opportunity

driving urbanization: lower infant mortality assoc. with high birth rates

people in the city though have lower birth rates

\noindent
large number of young people and also a great difficulty with unemployment

difficult to measure, though: informal/underground economy significant

governments lie about everything (hahahaha), you never know what to believe

figures from the Middle East are rarely verifiable

\noindent
there is also ``underemployment''

referring to a large number of people working but with nothing to do

especially prevalent in civil/govt orgs: buying loyalty with sinecures

but also evident elsewhere; man paid to turn on the faucet in the hotel

\noindent
privatization trending upward in the Middle East

most countries (esp. Syria, Algeria) historically feature govt. intervention in industry

example Egypt privatizing wine industry

initially, unemployment spikes with the shift

additionally, crony capitalism prevalent (industries to those in power at dirt-cheap prices)

expectation gap; ability to observe those who are more affluent than oneself

\section{8/31}

\end{document}


