\documentclass[12pt]{article}

\usepackage{mathptmx} % Times font

\usepackage{amsfonts}
\usepackage{amsmath}
\usepackage[version-1-compatibility]{siunitx}
\usepackage{fixltx2e}
\usepackage{multirow}

\topmargin -0.5in %topmargin=1in+\topmargin
\textheight 9in % default is letter % so 11-2=9in
\textwidth 6.5in % and 8.5-2in=6.5in
\oddsidemargin 0in %left margin = % 1in + oddsidemargin
\footskip 1cm % pagenumber to text

\setcounter{secnumdepth}{0}

\begin{document}

\noindent
Near Eastern Studies 10 Discussion, Fall 2015

\section{9/21}

Egypt: ``wasta'' is like cred, if you have it, you can get stuff done

it's all about who you know

the connections you have are important

if you don't have a connection with somebody, might need presents

``Cairo is like Athens on acid''

\noindent
One thing about Egypt, (sim. other Arab countries) people like to make you happy

people won't tell you outright if something is not doable

learned to tell by the expression on faces something isn't really doable but doesn't want to say

also, very hospitable, but in a different fashion from what we're used to

\noindent
In Egypt, as in many other countries, considered more desirable to be paler

\noindent
Egyptian military not too competent (thought Mexican tourists were ISIS, firebombed)

but other than that, Egypt is highly safe; ``Cairo is safer than San Francisco''

\noindent
Mubarak replaced by el-Sisi; as typical, likely a transition from dictatorship to dictatorship

(Abdel Fattah) el-Sisi seems to be good at putting people in prison

\noindent
Generally, Americans will be followed to keep tabs on the safety of the American

Mostly problematic because the police escorts, on a power trip, would harass Egyptian friends\\

\noindent
Today's lecture was about the Gulf states



\end{document}


